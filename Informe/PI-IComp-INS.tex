%%%%%%%%%%%%%%%%%%%%%%%%%%%%%%%%%%%%%%%%%
% Masters/Doctoral Thesis 
% LaTeX Template
% Version 2.5 (27/8/17)
%
% This template was downloaded from:
% http://www.LaTeXTemplates.com
%
% Version 2.x major modifications by:
% Vel (vel@latextemplates.com)
%
% This template is based on a template by:
% Steve Gunn (http://users.ecs.soton.ac.uk/srg/softwaretools/document/templates/)
% Sunil Patel (http://www.sunilpatel.co.uk/thesis-template/)
%
% Template license:
% CC BY-NC-SA 3.0 (http://creativecommons.org/licenses/by-nc-sa/3.0/)
%
%%%%%%%%%%%%%%%%%%%%%%%%%%%%%%%%%%%%%%%%%

%----------------------------------------------------------------------------------------
%	PACKAGES AND OTHER DOCUMENT CONFIGURATIONS
%----------------------------------------------------------------------------------------

\documentclass[
11pt, % The default document font size, options: 10pt, 11pt, 12pt
%oneside, % Two side (alternating margins) for binding by default, uncomment to switch to one side
spanish, % ngerman for German
singlespacing, % Single line spacing, alternatives: onehalfspacing or doublespacing
%draft, % Uncomment to enable draft mode (no pictures, no links, overfull hboxes indicated)
%nolistspacing, % If the document is onehalfspacing or doublespacing, uncomment this to set spacing in lists to single
%liststotoc, % Uncomment to add the list of figures/tables/etc to the table of contents
%toctotoc, % Uncomment to add the main table of contents to the table of contents
%parskip, % Uncomment to add space between paragraphs
%nohyperref, % Uncomment to not load the hyperref package
headsepline, % Uncomment to get a line under the header
% chapterinoneline, % Uncomment to place the chapter title next to the number on one line
% consistentlayout, % Uncomment to change the layout of the declaration, abstract and acknowledgements pages to \textit{match} the default layout
]{MastersDoctoralThesis} % The class file specifying the document structure

\usepackage[utf8]{inputenc} % Required for inputting international characters
\usepackage[T1]{fontenc} % Output font encoding for international characters
\usepackage{comment}
\usepackage{mathpazo} % Use the Palatino font by default
\usepackage{graphicx}
\usepackage[backend=bibtex,bibencoding=ascii,style=numeric,natbib=true]{biblatex} % Use the bibtex backend with the authoryear citation style (which resembles APA)
\usepackage{longtable}
\usepackage[table]{xcolor} 

\addbibresource{example.bib} % The filename of the bibliography

\usepackage[autostyle=true]{csquotes} % Required to generate language-dependent quotes in the bibliography
\usepackage{float}
% \usepackage{tabu}
\usepackage{booktabs}
% \usepackage[table,xcdraw]{xcolor}
\renewcommand*{\lstlistingname}{Código}
%----------------------------------------------------------------------------------------
%	MARGIN SETTINGS
%----------------------------------------------------------------------------------------

\geometry{
	paper=a4paper, % Change to letterpaper for US letter
	outer=2.5cm, % Inner margin
	inner=3.8cm, % Outer margin
	bindingoffset=.5cm, % Binding offset
	top=1.5cm, % Top margin
	bottom=1.5cm, % Bottom margin
	%showframe, % Uncomment to show how the type block is set on the page
}

%----------------------------------------------------------------------------------------
%	THESIS INFORMATION
%----------------------------------------------------------------------------------------

\thesistitle{Redes Neuronales para la determinación automática del control de una Red de Petri} % Your thesis title, this is used in the title and abstract, print it elsewhere with \ttitle
\supervisor{Dr. Ing \textsc{Micollini}, Orlando} % Your supervisor's name, this is used in the title page, print it elsewhere with \supname
\examiner{Ing. \textsc{Ventre}, Luis Orlando} % Your examiner's name, this is not currently used anywhere in the template, print it elsewhere with \examname
% \degree{Doctor of Philosophy} % Your degree name, this is used in the title page and abstract, print it elsewhere with \degreename
\authorA{\textsc{Izquierdo}, Agustina Nahir}
\matriculaA{37729473}
\emailA{agustina.izquierdo@mi.unc.edu.ar}

\authorB{\textsc{Navarro}, Matias Alejandro}
\matriculaB{38730347}
\emailB{matias.navarro@mi.unc.edu.ar}

\authorC{\textsc{Salvatierra}, Andrés}
\matriculaC{39611008}
\emailC{andressalvatierra@mi.unc.edu.ar}
% Your name, this is used in the title page and abstract, print it elsewhere with \authorname
% \addresses{} % Your address, this is not currently used anywhere in the template, print it elsewhere with \addressname

% \subject{Biological Sciences} % Your subject area, this is not currently used anywhere in the template, print it elsewhere with \subjectname
\keywords{Software Defined Networking (SDN), Administracion, Open Networking Operating System (ONOS)} % Keywords for your thesis, this is not currently used anywhere in the template, print it elsewhere with \keywordnames
\university{Universidad Nacional de Córdoba} % Your university's name and URL, this is used in the title page and abstract, print it elsewhere with \univname
% \department{Facultad de Ciencias Exactas Físicas y Naturales} % Your department's name and URL, this is used in the title page and abstract, print it elsewhere with \deptname
% \group{\href{http://researchgroup.university.com}{Research Group Name}} % Your research group's name and URL, this is used in the title page, print it elsewhere with \groupname
\faculty{Facultad de Ciencias Exactas Físicas y Naturales} % Your faculty's name and URL, this is used in the title page and abstract, print it elsewhere with \facname

\AtBeginDocument{
\hypersetup{pdftitle=\ttitle} % Set the PDF's title to your title
\hypersetup{pdfauthor=\authornameA} % Set the PDF's author to your name
\hypersetup{pdfkeywords=\keywordnames} % Set the PDF's keywords to your keywords
}

\begin{document}

\frontmatter % Use roman page numbering style (i, ii, iii, iv...) for the pre-content pages

\pagestyle{plain} % Default to the plain heading style until the thesis style is called for the body content

%----------------------------------------------------------------------------------------
%	TITLE PAGE
%----------------------------------------------------------------------------------------

\begin{titlepage}
	\begin{center}
				
		\resizebox{1\textwidth}{!}{\includegraphics{Figures/logo-unc-fcefyn.png}}%
				
		\vspace*{.06\textheight}
		{\scshape\LARGE \univname\par}\vspace{0.5cm} % University name
		{\scshape\LARGE \facname\par}\vspace{1.5cm}
		\textsc{\Large Proyecto Integrador}\\[0.5cm] % Thesis type
		\textsc{\Large Ingeniería en Computación}\\[0.5cm]
				
		\HRule \\[0.4cm] % Horizontal line
		{\huge \bfseries \ttitle\par}\vspace{0.4cm} % Thesis title
		\HRule \\[1.5cm] % Horizontal line
				 
		\begin{minipage}[t]{0.5\textwidth}
			\begin{flushleft} \large
				\emph{Autores:}\\
				\authornameA \\ % Author name - remove the \href bracket to remove the link
				\matriculanameA \\
				\hfill \break
				\authornameB \\
				\matriculanameB \\
				\hfill \break
				\authornameC \\
				\matriculanameC
				
			\end{flushleft}		
		\end{minipage}
		\begin{minipage}[t]{0.4\textwidth}
			\begin{flushright} \large
				\emph{Director:} \\
				\supname \\
				\hfill \break
				\emph{Co-director:} \\
				\examname
			\end{flushright}
		\end{minipage}\\[2cm]
				
		% \large \textit{A thesis submitted in fulfillment of the requirements\\ for the degree of \degreename}\\[0.3cm] % University requirement text
		% \textit{in the}\\[0.4cm]
		% \groupname\\\deptname\\[2cm] % Research group name and department name
				
		{\large {Septiembre 2020}}\\[4cm] % Date
		%\includegraphics{Logo} % University/department logo - uncomment to place it
				 
		\vfill
	\end{center}
\end{titlepage}

%----------------------------------------------------------------------------------------
%	DEDICATION
%----------------------------------------------------------------------------------------

\dedicatory{Para nuestras familias\ldots} 

% %----------------------------------------------------------------------------------------
% %	DECLARATION PAGE
% %----------------------------------------------------------------------------------------

% \begin{declaration}
% \addchaptertocentry{\authorshipname} % Add the declaration to the table of contents
% \noindent I, \authorname, declare that this thesis titled, \enquote{\ttitle} and the work presented in it are my own. I confirm that:

% \begin{itemize} 
% \item This work was done wholly or mainly while in candidature for a research degree at this University.
% \item Where any part of this thesis has previously been submitted for a degree or any other qualification at this University or any other institution, this has been clearly stated.
% \item Where I have consulted the published work of others, this is always clearly attributed.
% \item Where I have quoted from the work of others, the source is always given. With the exception of such quotations, this thesis is entirely my own work.
% \item I have acknowledged all main sources of help.
% \item Where the thesis is based on work done by myself jointly with others, I have made clear exactly what was done by others and what I have contributed myself.\\
% \end{itemize}
 
% \noindent Signed:\\
% \rule[0.5em]{25em}{0.5pt} % This prints a line for the signature
 
% \noindent Date:\\
% \rule[0.5em]{25em}{0.5pt} % This prints a line to write the date
% \end{declaration}

% \cleardoublepage

%----------------------------------------------------------------------------------------
%	QUOTATION PAGE
%----------------------------------------------------------------------------------------

% \vspace*{0.2\textheight}

% \noindent\enquote{\itshape Thanks to my solid academic training, today I can write hundreds of words on virtually any topic without possessing a shred of information, which is how I got a good job in journalism.}\bigbreak

% \hfill Dave Barry

%----------------------------------------------------------------------------------------
%	ABSTRACT PAGE
%----------------------------------------------------------------------------------------

\begin{abstract}
	\addchaptertocentry{\abstractname} % Add the abstract to the table of contents
	En respuesta al continuo incremento de los requerimientos en las redes de transporte, surgen no sólo avances en las capacidades de los dispositivos ópticos, sino también en los métodos de configuración en los mismos. La flexibilidad y la estandarización de las configuraciones pasan de ser una característica deseable, a ser un requerimiento funcional en los equipos. En las empresas de telecomunicaciones, donde dispositivos de diferentes características y fabricantes interactúan a la hora de brindar servicios, es crucial que  exista un protocolo estándar y un ente centralizado de administración. Históricamente esta comunicación se llevó a cabo a través del protocolo \textit{SNMP}, sin embargo las necesidades de las redes actuales involucran capacidades que exceden a este protocolo. 
	
	Este contexto da lugar al protocolo de configuraciones de red, también llamado \textit{NETCONF}, que se establece como un protocolo estandarizado por la \textit{Internet Engineering Task Force} (\textit{IETF}), que provee tanto funciones de control (ej. configuración del plano de datos) como así también funciones de administración (ej. acceso a información de monitoreo). Depende de \textit{Yet Another Next Generation} (\textit{YANG}) para describir las configuraciones de los dispositivos de manera estándar. Tanto \textit{NETCONF} como \textit{YANG}, se volvieron foco de interés de los operadores de red en busca de un estándar de configuración abierto.
	
	Al mismo tiempo, surge un nuevo paradigma de arquitectura de redes: Las Redes Definidas por Software (\textit{SDN} por sus siglas en inglés), una propuesta para solucionar la creciente complejidad en la administración de las redes. \textit{SDN} propone la separación del plano de datos y el plano de control, logrando de este modo un plano de control centralizado  facilitando no solo el control sino también la administración de las redes.

	En este proyecto integrador se busca, en primer lugar, adquirir en su totalidad los conocimientos involucrados con \textit{SDN} y administración de la configuración de los dispositivos de red. El objetivo será configurar un dispositivo óptico llamado muxponder, para ello se deberá adaptar el protocolo \textit{NETCONF} al equipo. Además, será necesario implementar un \textit{driver} en la interfaz \textit{southbound} del controlador \textit{SDN}. Así mismo, se genera una interfaz de usuario para la administración sencilla de los dispositivos presentes en la topología. Además, se desarrolla un ambiente para la verificación y validación de las aplicaciones. Finalmente, se brindan distintas ideas para mejorar lo aquí implementado y posibles vías para continuar trabajando en el ámbito de las redes definidas por software.

\end{abstract}

% \textbf{Palabras Clave:} \textit{\keywordnames}

%----------------------------------------------------------------------------------------
%	ACKNOWLEDGEMENTS
%----------------------------------------------------------------------------------------
 
\begin{acknowledgements}
	
	\addchaptertocentry{\acknowledgementname} % Add the acknowledgements to the table of contents

	% 	{\normalsize padres y hermanos \par}
	Muchas gracias a mi familia, por el apoyo incondicional a lo largo de todos estos años de estudio.  
	\bigskip
			
	% 	{\normalsize Hugo y Matt \par}
	Este proyecto no hubiera sido posible sin el soporte, la confianza, la supervisión y el duro empeño de mis directores, Hugo Carrer y Matthew Aguerreberry.
	\bigskip
			
	% 	{\normalsize amigos \par}
	Un especial agradecimiento a mis amigos y todas las personas que tuve el placer de conocer durante estos años de carrera.
	\bigskip
		
	% 	{\normalsize Fundación y gente que labura en la funda \par}
	Agradezco a la Fundación Fulgor y a la Fundación Tarpuy, y a todo su personal, por las oportunidades y enseñanzas compartidas.
	\bigskip
			
	% 	{\normalsize universisdad / profes de la facu \par}
	Finalmente, agradezco a la Facultad de Ciencias Exactas Físicas y Naturales de la Universidad Nacional de Córdoba por la oportunidad de realizar esta carrera de grado.

	\vspace*{\fill}
		
\end{acknowledgements}

%----------------------------------------------------------------------------------------
%	LIST OF CONTENTS/FIGURES/TABLES PAGES
%----------------------------------------------------------------------------------------

\hypersetup{
	linkcolor=black,
	citecolor=black,
	urlcolor=black
	}

\tableofcontents % Prints the main table of contents

\listoffigures % Prints the list of figures

\listoftables % Prints the list of tables

%----------------------------------------------------------------------------------------
%	ABBREVIATIONS
%----------------------------------------------------------------------------------------

\begin{abbreviations}{ll} % Include a list of abbreviations (a table of two columns)
		
	\textbf{API} & \textbf{A}pplication \textbf{P}rogramming \textbf{I}nterface\\
	\textbf{CLI} & \textbf{C}ommand \textbf{L}ine \textbf{I}nterface\\
	\textbf{SNMP} & \textbf{S}imple \textbf{N}etwork \textbf{M}anagement \textbf{P}rotocol\\
	\textbf{FIB} & \textbf{F}orwarding \textbf{I}nformation \textbf{B}ase\\
	\textbf{GUI} & \textbf{G}raphical \textbf{U}ser \textbf{I}nterface\\
	%\textbf{IGMP} & \textbf{I}nternet \textbf{G}roup \textbf{M}anagement \textbf{P}rotocol\\
	%\textbf{NFV} & \textbf{N}etwork \textbf{F}unction \textbf{V}irtualization\\
	%\textbf{ODL} & \textbf{O}pen\textbf{D}ay\textbf{l}ight\\
	\textbf{ONF} & \textbf{O}pen \textbf{N}etworking \textbf{F}oundation\\
	\textbf{IETF} & \textbf{I}nternet \textbf{E}ngineering \textbf{T}ask \textbf{F}orce\\
	%\textbf{ONOS} & \textbf{O}pen \textbf{N}etworking \textbf{O}perating \textbf{S}ystem\\
	%\textbf{OVS} & \textbf{O}pen \textbf{V} \textbf{S}witch\\
	%\textbf{PIM} & \textbf{P}rotocol \textbf{I}ndependent \textbf{M}ulticast\\
	\textbf{RIB} & \textbf{R}outing \textbf{I}nformation \textbf{B}ase\\
	%\textbf{REST} & \textbf{RE}presentational \textbf{S}tate \textbf{T}ransfer\\
	\textbf{SDN} & \textbf{S}oftware \textbf{D}efined \textbf{N}etwork\\
	%\textbf{SSM} & \textbf{S}ource \textbf{S}pecific \textbf{M}ulticast\\
	\textbf{UDP} & \textbf{U}ser \textbf{D}atagram \textbf{P}rotocol\\
	\textbf{TCP} & \textbf{T}ransmission \textbf{C}ontrol \textbf{P}rotocol\\
	\textbf{MIB} & \textbf{M}anagement \textbf{I}nformation \textbf{B}ase\\
	\textbf{TLS} & \textbf{T}ransport \textbf{L}ayer \textbf{S}ecurity\\
	\textbf{SSH} & \textbf{S}ecure \textbf{SH}ell\\
	\textbf{OTN} & \textbf{O}ptical \textbf{T}ransport \textbf{N}etwork\\
	\textbf{OTU} & \textbf{O}ptical \textbf{T}ransport \textbf{U}nit\\
	\textbf{ITU} & \textbf{I}nternational \textbf{T}elecommunication \textbf{U}nion\\
	\textbf{RFC} & \textbf{R}equest \textbf{F}or \textbf{C}omments\\
	\textbf{RPC} & \textbf{R}emote \textbf{P}rocedure \textbf{C}all\\
	\textbf{XML} & E\textbf{X}tensible \textbf{M}arkup \textbf{L}anguage\\
	\textbf{YANG} & \textbf{Y}et \textbf{A}nother \textbf{N}ext \textbf{G}eneration\\
	\textbf{NETCONF} & \textbf{NET}work \textbf{CONF}iguration Protocol\\
	\textbf{IANA} & \textbf{I}nternet \textbf{A}ssigned \textbf{N}umbers \textbf{A}uthority\\
	\textbf{FEC} & \textbf{F}orward \textbf{E}rror \textbf{C}orrection\\
	\textbf{FPGA} & \textbf{F}ield \textbf{P}rogrammable \textbf{G}ate \textbf{A}rray\\
	\textbf{ONOS} & \textbf{O}pen \textbf{N}etwork \textbf{O}perating \textbf{S}ystem\\
	\textbf{REST} & \textbf{RE}presentational \textbf{S}tate \textbf{T}ransfer\\
	\textbf{BSD} & \textbf{B}erkeley \textbf{S}oftware \textbf{D}istribution\\
	\textbf{MIT} & \textbf{M}assachusetts \textbf{I}nstitute \textbf{T}echnology\\
	\textbf{MB} & \textbf{M}ega \textbf{B}ytes\\
	\textbf{TB} & \textbf{T}era \textbf{B}ytes\\
	\textbf{NFV} & \textbf{N}etwork \textbf{F}unction \textbf{V}irtualization\\
	
	%\textbf{VNF} & \textbf{V}irtual \textbf{N}etwork \textbf{F}unction\\
				 
\end{abbreviations}
%----------------------------------------------------------------------------------------
%	THESIS CONTENT - CHAPTERS
%----------------------------------------------------------------------------------------

\mainmatter % Begin numeric (1,2,3...) page numbering

\pagestyle{thesis} % Return the page headers back to the "thesis" style

% Include the chapters of the thesis as separate files from the Chapters folder
% Uncomment the lines as you write the chapters

 % Chapter Template

\chapter{Introducción} % Main chapter title

\label{Chapter1} % Change X to a consecutive number; for referencing this chapter elsewhere, use \ref{ChapterX}

\section{Motivación e importancia del proyecto}

Las motivaciones para el desarrollo de este trabajo pueden dividirse en dos grandes pilares. Por un lado aquellas relacionadas al proyecto en sí, entre las cuales puede destacarse la necesidad de realizar una investigación y desarrollo de un algoritmo capaz de solucionar los problemas de vivacidad de las redes de Petri(RdP), puntualmente las que modelan Sistemas de procesos secuenciales simples con recursos($S^3PR$). Sumado a que el mismo podría formar parte de un proyecto más robusto que se está desarrollando en el Laboratorio de Arquitectura de Computadoras, y esto es algo emocionante.
\bigskip 

Por otra parte, existen sin duda ciertas motivaciones de naturaleza académica. Entre ellas podemos mencionar la integración de los conocimientos adquiridos a lo largo de nuestros estudios, la contribución a la comunidad de investigadores e incluso la puesta en práctica de procedimientos estrictos de investigación, de desarrollo de software y de documentación que nos serán sin duda de gran valor durante nuestro futuro ejercicio como profesionales.

%----------------------------------------------------------------------
% Estado del Arte
%----------------------------------------------------------------------
\section{Estado del arte}

En 1962, Petri inventó un enfoque teórico de la red para modelar y analizar los sistemas de comunicación en su tesis \cite{petri1962kommunikation}. Este modelo se basó en los conceptos de funcionamiento asíncrono y concurrente de las partes de un sistema y en la comprensión de que las relaciones entre las partes podían representarse mediante una red. Se ha realizado una gran cantidad de investigaciones tanto sobre la naturaleza como sobre la aplicación de las redes de Petri, la cuál parece estar en expansión.\\
Se ha demostrado que las redes de Petri son muy útiles en el modelado, análisis, simulación y control de sistemas concurrentes.

\par El flujo simultáneo de varios procesos en un sistema de asignación de recursos (RAS), que compiten por un conjunto finito de recursos, puede conducir a un punto muerto. Un interbloqueo (deadlock) ocurre cuando un conjunto de procesos se encuentra en un estado de ``espera circular''\footnote{Definido en el Marco Teórico (Sección \ref{sec:Interbloqueo})}, donde cada proceso del conjunto está esperando que un recurso sea liberado por otro proceso del conjunto mientras ocupa un recurso que, a su vez, es necesario por uno de los otros procesos. La noción de deadlock parcial o total es frecuente y es preferible la validación antes de la implementación para reducir los riesgos \cite{LIU2016198}.

Los estado de deadlock son una situación bastante indeseable en un sistema de fabricación automatizado. Su ocurrencia a menudo deteriora la utilización de recursos y pueden conducir a resultados catastróficos en sistemas críticos para la seguridad. Las redes de Petri son una herramienta matemática importante para manejar problemas de interbloqueo en sistemas de asignación de recursos.

Un sistema de fabricación flexible (FMS) o un sistema de fabricación automatizado (AMS) son un conglomerado de máquinas herramienta controladas numéricamente por computadora, amortiguadores, accesorios, robots, vehículos guiados automatizados (AGV) y otros dispositivos de manejo de materiales. Por lo general, exhibe un alto grado de uso compartido de recursos para aumentar la flexibilidad. La existencia de recursos compartidos puede conducir a condiciones de espera circular. En tal sistema, una vez que ocurren los puntos muertos, persisten y no se resolverían sin la intervención de seres humanos u otro agente externo.

En diversos estudios realizados por distintos autores, en busca de reducir estos estados de deadlock, se parte de la premisa de la existencia de cuatro estrategias para manejarlos.

\subsection{Estrategias de manejo de deadlock}

\subsubsection{Ignorar (Deadlock ignoring)}
Ignorar los estados de deadlock, que se conoce como el algoritmo de Ostrich \footnote{Concepto informático para denominar el procedimiento de algunos sistemas operativos. Esta teoría, acuñada por A. S. Tanenbaum, señala que dichos sistemas, en lugar de enfrentar el problema de los bloqueos mutuos asumen que estos nunca ocurrirán.}, se emplea en un sistema de asignación de recursos si la probabilidad de que se produzcan puntos muertos es mínima y la aplicación de otras estrategias de control de deadlock es técnicamente difícil. En un FMS o AMS, ignorar el estado deadlock es factible y razonable desde el punto de vista técnico y económico si el grado de intercambio de recursos es bajo.

\subsubsection{Prevenir (Deadlock prevention)}
Se logra controlando la solicitud de recursos y garantizando que nunca se produzcan estados de deadlock. Los recursos se otorgan a los procesos de tal manera que una solicitud de un recurso nunca conduce a situaciones de deadlock. El objetivo es imponer limitaciones a la evolución de un sistema. En este caso, el calculo se realiza offline de forma estática y una vez establecida una política de control, el sistema ya no puede alcanzar esos estados indeseables.
Una ventaja importante de los algoritmos de prevención de interbloqueo es que no requieren ningún costo de tiempo de ejecución, ya que los problemas se resuelven en las etapas de diseño y planificación del sistema.La principal crítica es que tienden a ser demasiado conservadores, lo que reduce la utilización de recursos y la productividad del sistema.

\subsubsection{Evitar (Deadlock avoidance)}
Para evitar los estados de deadlock se concede un recurso a un proceso solo si el estado resultante es seguro. Un estado se denomina seguro si existe al menos una secuencia de ejecución que permite que todos los procesos se ejecuten hasta su finalización. Para decidir si el próximo estado es seguro si se asigna un recurso a un proceso se debe realizar un seguimiento del estado del sistema global. Esto significa que son necesarios un gran almacenamiento y una amplia capacidad de comunicación.

\subsubsection{Detectar y recuperar (Deadlock detection and recovery)}
Se otorgan recursos a un proceso sin ningún control. El estado de la asignación de recursos y las solicitudes se examinan periódicamente para determinar si un conjunto de procesos está bloqueado. Este examen se realiza mediante un algoritmo de detección de interbloqueo. Si se encuentra un interbloqueo, el sistema se recupera abortando uno o más procesos interbloqueados y entregandole los recursos liberados a otros procesos. En la práctica de fabricación, a menudo se necesitan operadores humanos para esta estrategia y, por lo tanto, puede resultar muy costoso.

\subsection{Revisión de la literatura}
Las políticas de prevención de deadlock se han logrado ampliamente y han dado lugar a una gran cantidad de resultados. En esta sección, se revisan las estrategias de \textbf{prevención} de deadlock mediante el uso de redes de Petri y se desarrollan en base a diferentes técnicas como el análisis estructural y el análisis del grafo de alcanzabilidad.

\subsubsection{Métodos de análisis estructural}
Ezpeleta et al. \cite{paperezpeleta} utilizó una clase de redes de Petri, las del tipo S³PR y propuso un algoritmo para la asignación de recursos en FMS. El algoritmo propuesto agregó nuevos lugares a la red para imponer ciertas restricciones que prohíben la presencia de sifones vacíos.

Los trabajos de Huang et al. \cite{YiShengBook} y Huang y et al. \cite{YHuangandJeng} presentan una nueva política de prevención de deadlock para la clase de redes de Petri, donde los estados de deadlock están relacionados con sifones sin marcar. Se agregan dos tipos de lugares de control al modelo original para un sistema de fabricación flexible llamado lugar de control ordinario y lugar de control ponderado para evitar que los sifones se desmarquen.

En Li y Wei \cite{LiandWei} , se introduce el concepto de sifones elementales para diseñar un supervisor de red de Petri que haga cumplir la vivacidad para el mismo modelo de red de Petri. Basado en sifones elementales y conceptos de P-invariantes en redes de Petri, Li y Wei introducen un algoritmo de prevención de interbloqueo para una clase específica de redes de Petri que pueden modelar adecuadamente varios FMS \cite{LiandWei}, los sifones en un modelo de red de Petri se clasifican en dependientes y sifones elementales, y se agregan lugares de control para todos los sifones elementales, de modo que los sifones están controlados de forma invariante.

Huang \cite{Huang2007} propone una nueva metodología para sintetizar supervisores para la asignación de recursos en los FMS; se considera la clase de red Petri, a saber, S³PR, donde los puntos muertos están relacionados con sifones mínimos no marcados; todos los sifones mínimos deben controlarse agregando plazas de control. En este estudio, el número de plazas de control se reduce utilizando el concepto de sifón elemental.

Chen et al. \cite{paperchen}, presenta un algoritmo de prevención de deadlock y el concepto de extracción por sifón se utiliza para calcular sifones no marcados para el modelo de red de Petri. Primero, un algoritmo de extracción de sifón obtiene un sifón no marcado máximo, divide los lugares en él y determina un sifón necesario de los lugares divididos; esto se lleva a cabo para todos los sifones sin marcar. Luego, el algoritmo diseña un monitor conveniente para marcar cada sifón necesario hasta que el modelo de red de Petri controlado esté activo.

Liu et al. \cite{LiuanLi2013} propuso una variedad de algoritmos de control de interbloqueo para AMS con recursos no confiables, los monitores y las subredes de recuperación están diseñadas para sifones mínimos estrictos (sifón vacío) y recursos no confiables, respectivamente, y se utilizan dos tipos de arcos que son normales e inhibidores para conectar monitores con subredes de recuperación. Para obtener un supervisor con complejidad estructural, los sifones elementales extraídos de todos los sifones mínimos estrictos están obviamente controlados. 


\subsubsection{Enfoques basados en el grafo de alcanzabilidad}
Viswanadham et al. \cite{Viswanadham} presentó métodos de asignación de recursos estáticos para eliminar los puntos muertos; En este estudio, se utiliza un grafo de alcanzabilidad del modelo de red de Petri para llegar al método de asignación de recursos estático. Se implementa un algoritmo de prevención de interbloqueo para un sistema de fabricación de tamaño pequeño que consta de una máquina y un vehículo guiado automatizado (AGV). Los autores observaron que el algoritmo propuesto se puede aplicar de manera efectiva sólo para sistemas de fabricación de tamaño pequeño.

\par El trabajo de Uzam \cite{uzam2004, uzam2002} propone y mejora el método basado en una teoría de regiones para diseñar un supervisor de red de Petri óptimo. El tamaño del grafo de alcanzabilidad del modelo de red de Petri es un problema importante para aplicar la política de prevención de interbloqueos a un modelo de red de Petri muy grande. Por lo tanto, propusieron un algoritmo de reducción para simplificar modelos de redes de Petri muy grandes con el fin de facilitar los cálculos necesarios. Basado en la teoría de las regiones, Uzam y Zhou \cite{uzamandzhou2004} desarrollaron una política iterativa de prevención de interbloqueos para FMS. En su estudio, el grafo de alcanzabilidad de un modelo de red de Petri se divide en dos partes: zona libre de bloqueo (zona activa, LZ) y zona de bloqueo (DZ). La zona viva se desarrolla ya que el componente máximo fuertemente conectado contiene la marca inicial. La zona de interbloqueo contiene marcas desde las que no se puede alcanzar la marca inicial. FBM está definido como una marca en la zona de interbloqueo. Los interbloqueos se pueden eliminar prohibiendo el disparo de las transiciones habilitadas en FBM. En su trabajo, el enfoque presentado tiene dos problemas. Primero, no se puede garantizar un supervisor óptimo en general, incluso si existe un supervisor óptimo. En segundo lugar, en cada iteración se requiere el cálculo del grafo de alcanzabilidad total para verificar si las marcas en la zona de interbloqueo son accesibles. Este enfoque es fácil de usar y simple si el grafo de alcanzabilidad de un sistema es pequeño pero no puede garantizar la optimización del comportamiento del supervisor. Los monitores redundantes en el supervisor de red de Petri pueden existir cuando el supervisor está diseñado mediante un enfoque de control de sifón iterativo. Por tanto, Uzam et al. \cite{uzammuratli2007} introdujo un enfoque para identificar y eliminar los monitores redundantes mediante el cálculo del gráfico de accesibilidad de un modelo de red de Petri controlado. Si la red de Petri controlada no pierde la vivacidad cuando se eliminan los monitores redundantes, entonces los monitores redundantes se pueden eliminar del supervisor.\\

El objetivo de esta sección fue presentar una revisión de la literatura sobre los distintos enfoques planteados hasta el momento respecto a la prevención de los estados de deadlock.
Un enfoque híbrido de control de interbloqueos se refiere a aquel que combina distintas de estas planificaciones para tratarlos. La motivación esencial de plantear una estrategia de este tipo es aprovechar las ventajas o formalismos de múltiples de estas, evitando sus desventajas. Esto fue puntualmente lo que nos motivó a no regirnos por una única estrategia o estudio planteado.


%----------------------------------------------------------------------
% Objetivos
%----------------------------------------------------------------------
\section{Objetivos}
Como se anticipó en la motivación, el objetivo final es el desarrollo de un algoritmo capaz de determinar las plazas y arcos necesarios a incorporar a la red original para lograr la ejecución de manera controlada, alcanzando la vivacidad de la misma. 

Al tratarse de un trabajo de investigación en el cual se fue evolucionando sobre distintas situaciones que se fueron encontrando a lo largo de su desarrollo, fueron apareciendo diferentes objetivos intermedios (mencionados en las diferentes iteraciones del mismo) mediante los que se logró alcanzar el objetivo principal antes mencionado que es obtener una red viva o libre de deadlock(interbloqueo).

\bigskip

Para esto se propone usar como soporte el software Petrinator, el mismo se utilizará para simular la redes y extraer la información necesaria de las mismas. Para que esta última pueda ser utilizada por nuestro algoritmo se proponen los siguientes objetivos secundarios:

\begin{itemize}
    \item Implementar una interfaz que procese la información extraída del software antes mencionado.
    
    \item Realizar una segunda interfaz que conecta el algoritmo nuevamente con el software Petrinator, indicando mediante una retroalimentación los nuevos arcos y plazas a colocar.
\end{itemize}

%----------------------------------------------------------------------
% Requerimientos
%----------------------------------------------------------------------
\section{Requerimientos}
En la ingeniería, los requerimientos se utilizan como datos de entrada en la etapa de diseño del producto. Establecen qué debe hacer el sistema, aunque no especifican la manera en que debe hacerlo. La fase de captura y registro de requisitos puede estar precedida por una fase de análisis conceptual del proyecto. 
\bigskip

\subsection{Listado de requerimientos}
\begin{table}[H]
    \centering
    \begin{tabular}{|c|p{12.8cm}|}
    \hline
    \textbf{ID} & \textbf{Descripción}  \\  \hline
    $R_{1}$ & Se debe usar el software Petrinator para la construcción, análisis y extracción de los archivos necesarios de la red de Petri.\\
    \hline
    
    $R_{2}$ & El algoritmo debe ser capaz de leer los archivos de extensión '.html' de los diferentes análisis exportados del Petrinator y llevar a cabo su posterior procesamiento para su utilización. \\ 
    \hline
    
    $R_{3}$ & El algoritmo debe modificar el archivo de extensión '.pflow' de la red en cuestión, para agregar plazas, arcos y quitar estos últimos en caso de ser necesario.\\
    \hline
    
    $R_{4}$ & Se debe agregar una funcionalidad al Petrinator para poder recargar la red una vez modificada por el algoritmo. \\ 
    \hline
    
    $R_{5}$ & El algoritmo debe converger en redes de Petri del tipo S³PR.  \\ 
    \hline
    
    $R_{6}$ & El algoritmo debe ser desarrollado para poder ser integrado al Petrinator como una nueva funcionalidad del mismo.  \\ 
    \hline
    
    \end{tabular}
    \label{tab:listadoRiesgos}
    \caption{Listado de requerimientos.}
\end{table}

%--------
% Riesgos
%--------
\section{Análisis de riesgos}
Un riesgo es un evento o condición incierta que, en caso de ocurrir, tendrá consecuencias negativas sobre al menos uno de los requerimientos del proyecto. Por esta razón es importante identificarlos con anticipación para mitigarlos.

\subsection{Listado de riesgos}
\begin{table}[H]
    \centering
    \begin{tabular}{|c|p{12.8cm}|}
    \hline
    \textbf{ID} & \textbf{Descripción}  \\  \hline
    $R_{1}$ & Recursos de hardware insuficientes.\\
    \hline
    
    $R_{2}$ & Ausencia de algoritmo. \\ 
    \hline
    
    $R_{3}$ & Algoritmo inadecuado. \\
    \hline
    
    $R_{4}$ & Herramienta inadecuada. \\ 
    \hline
    
    $R_{5}$ & Ausencia/escasez de datos. \\ 
    \hline
    
    \end{tabular}
    \label{tab:listadoRiesgos}
    \caption{Listado de riesgos.}
\end{table}

%--------
% R1
%--------
\begin{table}[H]
    \centering
    \begin{tabular}{|p{13.7cm}|}
    \hline
    \textbf{R1 - Recurso de hardware insuficiente} \\
    \hline \hline 
    \textbf{Condición}: Los recursos de hardware disponibles (computador básico) son inferiores a los necesarios.\\
    \hline
    \textbf{Consecuencia}: No permite analizar redes de Petri demasiado complejas, es decir, que presentan un gran numero de estados. \\
    \hline
    \textbf{Efecto}: Limitar la implementación de nuestro algoritmo a redes con una cantidad reducida de estados. \\
    \hline
    \end{tabular}
    \label{tab:R1}
    \caption{R1 - Recurso de hardware insuficiente.}
\end{table}

%--------
% R2
%--------
\begin{table}[H]
    \centering
    \begin{tabular}{|p{13.7cm}|}
    \hline
    \textbf{R2 - Ausencia de algoritmo} \\
    \hline \hline
    \textbf{Condición}: Inexistencia del algoritmo.\\
    \hline
    \textbf{Consecuencia}: No lograr alcanzar la vivacidad de una red de Petri. \\
    \hline
    \textbf{Efecto}: Irrealizabilidad del proyecto. \\
    \hline
    \end{tabular}
    \label{tab:R2}
    \caption{R2 - Ausencia de algoritmo.}
\end{table}

%--------
% R3
%--------
\begin{table}[H]
    \centering
    \begin{tabular}{|p{13.7cm}|}
    \hline
    \textbf{R3 - Algoritmo inadecuado} \\
    \hline \hline 
    \textbf{Condición}: Encontrar un algoritmo pero que no converge en todas las redes del mismo tipo.\\
    \hline
    \textbf{Consecuencia}: Iteraciones infinitas del algoritmo. \\
    \hline
    \textbf{Efecto}: Incremento del tiempo dedicado en la investigación para la readaptación del algoritmo. \\
    \hline
    \end{tabular}
    \label{tab:R3}
    \caption{R3 - Algoritmo inadecuado.}
\end{table}

%--------
% R4
%--------
\begin{table}[H]
    \centering
    \begin{tabular}{|p{13.7cm}|}
    \hline
    \textbf{R4 - Herramienta inadecuada} \\
    \hline \hline 
    \textbf{Condición}: Elección incorrecta de la herramienta para implementar la red de Petri y/o utilización errónea.\\
    \hline
    \textbf{Consecuencia}: La herramienta no se adapta a nuestros datos ni a los requerimientos. \\
    \hline
    \textbf{Efecto}: Ineficiencia del modelo. \\
    \hline
    \end{tabular}
    \label{tab:R4}
    \caption{R4 - Herramienta inadecuada.}
\end{table}

%--------
% R5
%--------
\begin{table}[H]
    \centering
    \begin{tabular}{|p{13.7cm}|}
    \hline
    \textbf{R5 - Ausencia/escasez de datos} \\
    \hline \hline 
    \textbf{Condición}: Escasez de la obtención de datos.\\
    \hline
    \textbf{Consecuencia}: Imposibilidad de testear/desarrollar el algoritmo. \\
    \hline
    \textbf{Efecto}: Irrealizabilidad del proyecto. \\
    \hline
    \end{tabular}
    \label{tab:R5}
    \caption{R5 - Ausencia/escasez de datos.}
\end{table}

%----------------------------------------------------------------------
% Estructura del texto
%----------------------------------------------------------------------
\section{Estructura del texto}

Aquí se listan los distintos capítulos que conforman el proyecto, presentando una breve descripción de su contenido. El escrito está compuesto por 4 capítulos, los apéndices y la bibliografía.


\begin{itemize}   
    \item \textbf{Capítulo 1 - Introducción:} Se exponen en este capítulo los aspectos más significativos del proyecto, donde se incluye las motivaciones que llevaron a realizar el mismo junto con una revisión del estado del arte relacionado, el análisis de riesgos y requerimientos junto con los objetivos propuestos para el trabajo de fin de grado.
    
    \item \textbf{Capítulo 2 - Marco teórico:} Aquí se abordan los conceptos necesarios para comprender el enfoque del proyecto, además que los mismos dan fundamento a las posteriores implementaciones prácticas.
    
    \item \textbf{Capítulo 3 - Desarrollo:} En este capítulo se analizan todas las herramientas que permitieron la implementación del algoritmo desarrollado en este proyecto. Incluye el desarrollo en base a la investigación realizada a partir del estado del arte y los conceptos teóricos mencionados en el capítulo 2. Como también el algoritmo en sus 4 versiones, cada una con sus respectivos objetivos, desarrollo y conclusiones.
    
    \item \textbf{Capítulo 4 - Conclusión:} Se presenta en este capítulo las conclusiones obtenidas tras la realización del trabajo y posibles vías de trabajos futuros.
	
	\item \textbf{Apéndices:} En los apéndices se proporciona al lector dos tutoriales, uno que ejemplifica como desplegar el entorno de trabajo y el otro es la ejecución de la versión final del algoritmo desarrollado en este proyecto.
    
    \item \textbf{Bibliografía:} En esta parte final del documento, se muestran todas las referencias que se han consultado para el desarrollo del proyecto.
    
\end{itemize}
 % Chapter Template
% cSpell:words parencite 
\chapter{Marco teórico} % Main chapter title 

\label{Chapter2} % Change X to a consecutive number; for referencing this chapter elsewhere, use \ref{ChapterX}

%----------------------------------------------------------------------------------------
%	SECTION 1
%----------------------------------------------------------------------------------------
\section{Redes de Petri} \label{sec:rdp}

Una red de Petri es un modelo gráfico, formal y abstracto para la representación de sistemas distribuidos y el análisis del flujo de información. Este modelo facilita la comprensión sobre la estructura y el comportamiento dinámico y estático del sistema modelado. Las redes de Petri son de utilidad principalmente en el diseño de sistemas de \textit{hardware} y \textit{software} para especificación, simulación y diseño de diversos problemas de ingeniería, especialmente útiles para representar procesos concurrentes, así como procesos donde puedan existir restricciones en cuanto a la simultaneidad, la precedencia o la frecuencia de eventos concurrentes. \cite{brams}

Las redes de Petri están fuertemente asociadas a la teoría de grafos, ya que las mismas pueden representarse como un grafo dirigido bipartito compuesto por cuatro elementos \cite{libropopn}:
\begin{itemize}
    \item \textit{Plazas}: Representan los estados del sistema. Las plazas son variables de estado que pueden tomar valores enteros.
    \item \textit{Tokens}: Los tokens figuran como puntos negros dentro de las plazas. Éstos representan el valor específico de una condición o estado y generalmente se traducen a la presencia o ausencia de algún recurso del sistema.
    \item \textit{Transiciones}: Las transiciones representan el conjunto de sucesos cuya ocurrencia produce la modificación de los estados (y en consecuencia del estado global) del sistema.
    \item \textit{Arcos}: Los arcos indican las interconexiones entre las plazas y las transiciones, estableciendo el flujo de tokens que sigue el sentido de la flecha.
\end{itemize}

Una vez definidos sus componentes, se puede decir que una red de Petri es un grafo dirigido con dos tipos de nodos: plazas y transiciones. Estos nodos están vinculados por arcos, los cuales sólo pueden conectar una plaza con una transición o viceversa. Por otro lado, una red de Petri puede ser descrita mediante dos componentes:

\begin{enumerate}
    \item Una estructura de red
    \item Un marcado inicial
\end{enumerate}

La estructura de red hace referencia a la red en sí, mientras que el marcado inicial sólo representa el estado inicial del sistema (denominado estado \textbf{idle}), es decir, sin que ninguna transición haya sido disparada. Un ejemplo simple de una red de Petri marcada se muestra en la figura \ref{fig:rdp2.1_tradicional}. Éste será utilizado en las secciones siguientes para ilustrar operaciones y/o propiedades de las redes de Petri. 

\begin{figure}[H]
	\centering
	\includegraphics[scale=1.0]{Figures/marco teorico/imag1.png}
	\caption{Red de Petri marcada.}
	\label{fig:rdp2.1_tradicional}
  \end{figure}


\subsection{Estructura de una red de Petri ordinaria} %Referencia aqui 
La estructura de una red de Petri puede definirse como una tupla de 5 elementos (5-tupla)\cite{falko} de la siguiente manera:

\begin{equation}
    N = (P, T, I^- , I^+ , M_0) 
\end{equation}

\noindent Donde:
\begin{itemize}
    \item $ P = \{P_1 , P_2,..., P_n \}$ es un conjunto finito y no vacío que contiene todas las plazas de la red.
    \item $T = \{T_1 , T_2, ... ,T_m \}$ es un conjunto finito y no vacío que contiene todas las transiciones.
    \item $I^{-}$ y $I^+$ son las matrices \textit{pre} y \textit{post} respectivamente, cuya composición se abordará en la próxima sección.
    \item $M_0$ es el marcado inicial de la red. Definido como un vector con un elemento para cada plaza, donde $M_0[i]$ contendrá la cantidad de \textit{tokens} en la plaza \textit{i} para el estado inicial.
\end{itemize}

Siguiendo con el ejemplo propuesto en la sección anterior (figura \ref{fig:rdp2.1_tradicional}), se puede representar la red como $N = \{P, T, I^- , I^+ , M_0 \}$ donde:

\begin{itemize}
    \item $P = \{P_1 , P_2, P_3 , P_4 \}$ 
    \item $T =  \{T_1, T_2, T_3, T_4\}$
    \item $M_0 = [1,1,2,0]$
\end{itemize}

\noindent A continuación se explicará la manera de obtener las matrices $I^-$ e $I^+$.


\subsection{Matriz de incidencia}
Las matrices  $I^-$ e $I^+$ son las funciones de incidencia de entrada y salida de las plazas. Para el caso de la matriz $I^+$ , denominada post, se tiene que cada elemento $post(P_i , T_j)$ contiene el peso asociado al arco que va desde $T_j$ hasta $P_i$ . Este peso indica la cantidad de \textit{tokens} que se generan en la plaza $P_i$ cuando la transición $T_j$ es disparada. \\
Por otro lado, en la matriz $I^-$ , denominada $pre$, cada elemento $pre(P_i , T_j)$ contiene el peso asociado al arco que va desde $P_i$ hasta $T_j$ e indica la cantidad de $tokens$ que se retiran de la plaza $P_i$ cuando se dispara la transición $T_j$.\\

\hfill \break 
Siguiendo con el ejemplo de la figura \ref{fig:rdp2.1_tradicional} , las matrices $I^-$ e $I^+$ asociadas son:

\begin{equation}
    I^+ = 
    \begin{pmatrix}
        0 & 0 & 0 & 1 \\
        1 & 0 & 0 & 0 \\
        1 & 0 & 0 & 0 \\
        0 & 1 & 1 & 0 \\
    \end{pmatrix}
\end{equation}
\begin{equation}
    I^- = 
    \begin{pmatrix}
        1 & 0 & 0 & 0 \\
        0 & 1 & 0 & 0 \\
        0 & 0 & 1 & 0 \\
        0 & 0 & 0 & 1 \\
    \end{pmatrix}
\end{equation}

Las filas de las matrices representan las plazas mientras que las columnas representan las transiciones, lo cual quiere decir que las matrices tendrán tantas filas como plazas tenga la red de Petri, y tantas columnas como transiciones. \\ \par

De esta forma se puede observar como el elemento $I^-[0][0]$ indica la relación de salida entre $P_1$ y $T_1$ . Más precisamente indica que cuando $T_1$ se dispara, solo un $token$ es retirado de $P_1$ (ya que el peso del arco entre $P_1$ y $T_1$ es 1). De igual manera, el elemento $I^+[0][0] = 0$ indica que cuando la misma transición se dispara, no se genera ningún token en $P_1$ (ya que no existe ningún arco que parta de $T_1$ hacia $P_1$). \\ \par

A partir de estas definiciones, se puede obtener la matriz de incidencia de la red. La misma está definida a continuación:

\begin{equation}
    I = I^+ - I^-
\end{equation}

Cabe aclarar que una red de Petri puede reconstruirse completamente a partir de sus matrices $I^+$ e $I^-$ , pero no así si se tiene sólo la matriz de incidencia $I$. Esto quiere decir que puede haber varias redes de Petri distintas con la misma matriz de incidencia, pero solamente una para las matrices $I^+$ e $I^-$ . Sin embargo, cuando una red de Petri no tiene autobucles (presente cuando se tienen dos arcos (con sentidos contrarios) entre una misma plaza y transición), su matriz de incidencia determina completamente su estructura. \\ \par

\textsc{Ejemplo} La matriz de incidencia asociada a la red de Petri de la figura \ref{fig:rdp2.1_tradicional} será entonces:

\begin{equation}
    I = 
    \begin{pmatrix}
        0 & 0 & 0 & 1 \\
        1 & 0 & 0 & 0 \\
        1 & 0 & 0 & 0 \\
        0 & 1 & 1 & 0 \\
    \end{pmatrix}
    -
    \begin{pmatrix}
        1 & 0 & 0 & 0 \\
        0 & 1 & 0 & 0 \\
        0 & 0 & 1 & 0 \\
        0 & 0 & 0 & 1 \\
    \end{pmatrix}
    = 
    \begin{pmatrix}
        -1 & 0 & 0 & 1 \\
        1 & -1 & 0 & 0 \\
        1 & 0 & -1 & 0 \\
        0 & 1 & 1 & -1 \\
    \end{pmatrix}
\end{equation}


%----------------------------------------------------------------------------------------
%	SECTION 2
%----------------------------------------------------------------------------------------
\section[Dinámica de una red de Petri]{Dinámica de una red de Petri} \label{sec:drdp}

Se dice que una transición está sensibilizada cuando el marcado de todas las plazas entrantes a la transición es mayor o igual al peso de los arcos que las unen con dicha transición.\cite{papermico}  \\
Antes de expresar la condición de sensibilizado de manera general, es necesario definir los siguientes conjuntos y funciones:

\begin{itemize}
    \item $\bullet T_j$ es el conjunto compuesto por las plazas entrantes a $T_j$.
    \item $T_j \bullet$  es el conjunto compuesto por las plazas salientes de $T_j$.
    \item $\bullet \bullet P_i$ es el conjunto compuesto por las transiciones entrantes a las plazas que sensibilizan a las transiciones que le \textit{agregan} tokens a $P_i$.
    \item $P_i \bullet \bullet$  es el conjunto compuesto por las transiciones entrantes a las plazas que sensibilizan a las transiciones que le \textit{quitan} tokens a $P_i$.
    \item $m_k(P_i)$ es el marcado de la plaza Pi antes de disparar la transición $T_j$ .
    \item $m_{k+1}(P_i)$ es el marcado de la plaza $P_i$ después de disparar la transición $T_j$.
    \item $w_{ij}$ es el peso del arco $P_i \rightarrow T_j$.
    \item $w_{ji}$ es el peso del arco $T_j \rightarrow P_i$.
\end{itemize}

\noindent Entonces, el sensibilizado de una transición $T_j$ está dado por:

\begin{equation}
    T_j \ est\acute{a} \ sensibilizada \ sii \ \forall \ P_i \ \in \ \bullet T_j \Rightarrow m_k (P_i) > w_{ij}
\end{equation}

Esta definición de sensibilizado es solo válida cuando los arcos que conectan las plazas con $T_j$ son arcos comunes. \\
En la figura \ref{fig:rdp2.2_marcada} se resaltan las transiciones sensibilizadas del ejemplo anterior.

\begin{itemize}
    \item $T_1$, $T_2$, $T_3$ están sensibilizadas, mientras $T_4$ no está sensibilizada (ya que el marcado de $P_4$ es menor al peso del arco $P_4$ \rightarrow $T_4$ ).
\end{itemize}

\begin{figure}[H]
	\centering
	\includegraphics[scale=1.00]{Figures/marco teorico/imag2.png}
	\caption{Transiciones sensibilizadas de la red de Petri.}
	\label{fig:rdp2.2_marcada}
  \end{figure}


\subsection{Disparo de una transición}
Si una transición está sensibilizada, la misma puede dispararse. El disparo de una transición resulta en un nuevo marcado de la red. Más precisamente, al ejecutarse una transición $T_j$ con un marcado $m_k$, los marcados de las plazas pertenecientes a la red se alteran cumpliendo con las siguientes declaraciones:

\begin{equation}
    \sigma (m_k, T_j) = 
    \begin{array}{cc}
         \forall \ P_i \ \in \ \bullet T_j \ \Rightarrow \ m_{k+1}(P_i) = m_k(P_i) - w_{ij}  \\
         \forall \ P_i \ \in \ T_j \bullet \ \Rightarrow \ m_{k+1}(P_i) = m_k(P_i) + w_{ji}  \\
         \forall \ P_i \ \notin \ \bullet T_j \cup T_j \bullet \ \Rightarrow m_{k+1}(P_i) = \ m_k(P_i)
    \end{array}
\end{equation}

\noindent Es decir:
\begin{itemize}
    \item Para todas las plazas entrantes a $T_j$, el nuevo marcado de cada plaza se habrá \textbf{decrementado} tantos $tokens$ como peso tenga el arco $P_i \rightarrow T_j$. 
    \item Para todas las plazas salientes de $T_j$, el nuevo marcado de cada plaza se habrá \textbf{incrementado} tantos $tokens$ como peso tenga el arco $T_j \rightarrow P_i$.
    \item Para el resto de las plazas, el nuevo marcado será exactamente igual al que tenían antes del disparo de $T_j$ .
\end{itemize}

Continuando con el ejemplo anterior, se puede observar en la figura \ref{fig:rdp2.3_disparada} el nuevo marcado de la red luego del disparo de la transición $T_3$:

\begin{itemize}
    \item La única plaza entrante a $T_3(P_3 )$, ha \textbf{decrementado} su marcado en 1 $token$ (ya que el peso del arco $P_3 \rightarrow T_3$ es 1).
    \item La plaza saliente de $T_j(P_4)$ ha \textbf{incrementado} su marcado de acuerdo a los pesos de los arcos correspondientes, en este caso en 1.
    \item Las plazas que no es entrante ni saliente de $T_3(P_3)$ ha mantenido su marcado original.
\end{itemize}

Cabe destacar que como consecuencia del disparo de $T_3$, se ha producido la sensibilización de $T_4$.

\begin{figure}[H]
	\centering
	\includegraphics[scale=1.00]{Figures/marco teorico/imag3.png}
	\caption{Nuevo marcado de la red de Petri.}
	\label{fig:rdp2.3_disparada}
  \end{figure}


\subsection{Función de transferencia y ecuación de estado}
Una vez explicada la dinámica del disparo de una transición y la forma de obtener la matriz de incidencia, se detalla una expresión matemática necesaria para obtener el nuevo marcado luego del disparo de una transición. 
La misma se denomina \textbf{función de transferencia} y está definida como el producto de la matriz de incidencia $I$ con un vector $\vec{\delta}$ cuyos componentes son todos ceros, exceptuando el componente asociado a la transición que se quiere disparar, cuyo valor será uno. Entonces, se tendrá:

\begin{equation}
    I \ . \ \vec{\delta }    
\end{equation}

\noindent donde, para el disparo de una transición $T_j$ , se tiene:
\begin{itemize}
    \item $\delta[j] = 1$
    \item $\delta[i] = 0 \ \forall \ i/i \neq j$
\end{itemize}

Por otro lado, es necesario introducir la \textbf{ecuación de estado} de las redes de Petri. Con esta ecuación es posible obtener el siguiente estado del sistema luego del disparo de una transición. Esta es una manera más simple que la metodología gráfica para analizar la evolución de los sistemas. La ecuación de estado en un tiempo $i$, para calcular el nuevo marcado de la red en un tiempo $i+1$ se define como:

\begin{equation}
    M_{i+1} = M_i + I \ . \ \vec{\delta}
    \label{ec:estado}
\end{equation}

\noindent donde $M_{i+1}$ es el marcado luego del disparo de la transición, $M_i$ es el marcado antes del disparo y el segundo término de la ecuación es la función de transferencia.

Siguiendo con el ejemplo hasta ahora analizado se verá cómo calcular el marcado de la red luego del disparo de la transición $T_3$ haciendo uso de la ecuación de estado.

En la figura \ref{fig:rdp2.2_marcada} se observan las transiciones sensibilizadas de la red para el marcado inicial $M_0$.
\\
La ecuación de estado requiere tres elementos:

\begin{enumerate}
    \item El marcado antes del disparo. Éste es:
        \begin{equation}
            M_i = M_0 = 
            \begin{pmatrix}
                1 \\
                1 \\
                2 \\
                0
            \end{pmatrix}
        \end{equation}
        
    \item La matriz de incidencia de la red. La misma fue calculada en la sección 2.1.2 y es la siguiente:
        \begin{equation}
           I = 
            \begin{pmatrix}
                -1 & 0 & 0 & 1 \\
                1 & -1 & 0 & 0 \\
                1 & 0 & -1 & 0 \\
                0 & 1 & 1 & -1 
            \end{pmatrix}
        \end{equation}
        
    \item El vector de disparo $\vec{\delta}$, que tendrá tantos elementos como transiciones haya en la red, cuyos valores serán cero para todas las transiciones excepto para aquella que se desee dispara:
        \begin{equation}
            \vec{\delta} = 
            \begin{pmatrix}
                0 \\
                0 \\
                1 \\
                0
            \end{pmatrix}
        \end{equation}
\end{enumerate}

\noindent con lo cual el nuevo marcado está definido por:
\begin{equation}
    M_{i+1} = 
    \begin{pmatrix}
        1 \\
        1 \\
        2 \\
        0
    \end{pmatrix}
    +
    \begin{pmatrix}
        -1 & 0 & 0 & 1 \\
        1 & -1 & 0 & 0 \\
        1 & 0 & -1 & 0 \\
        0 & 1 & 1 & -1 
    \end{pmatrix}
    .
    \begin{pmatrix}
        0 \\
        0 \\
        1 \\
        0
    \end{pmatrix}
    =
    \begin{pmatrix}
        1 \\
        1 \\
        1 \\
        1
    \end{pmatrix}    
\end{equation}

\noindent resultado que coincide con lo obtenido en la figura \ref{fig:rdp2.3_disparada}. La ecuación de estado representa matemáticamente el comportamiento dinámico del sistema, permitiendo calcular el nuevo estado del mismo luego de la ocurrencia de un evento a través de una simple ecuación.

\subsection{Extensión de la ecuación de estado}
Como se mencionó en la sección anterior, la ecuación \ref{ec:estado} permite calcular el siguiente estado luego del disparo de una transición. Sin embargo, puede que se desee obtener el marcado final luego de una secuencia de disparos. Suponiendo que se parte del estado inicial $M_0$, esto puede representarse como:

\begin{equation}
    M_i = M_0 + I \ . \sum_{j=1}^i u_j
\end{equation}

\noindent donde la sumatoria representa un vector asociado a la secuencia de transiciones que se desea disparar y se denomina vector S. Para ejemplificar, el cálculo de un marcado $M_i$ a partir del marcado inicial y luego del disparo de las transiciones {$T_3$, $T_4$, $T_1$, $T_2$,} está dado por:

\begin{equation}
    M_i = M_0 + I \ . \ \vec{S}
\end{equation}

\noindent donde $\vec{S}$ = \{ 1, 1, 1, 1 \} e $I$ es la matriz de incidencia asociada a la red.

%----------------------------------------------------------------------------------------
%	SECTION 3
%----------------------------------------------------------------------------------------
\section{Propiedades de las redes de Petri} \label{sec:gestionred}
\subsection{Propiedades de limitación}
Dada una red de Petri definida por $PN = \{ P, T, I^-, I^+, M_0 \}$, se dice que una plaza P está k-limitada si existe un número entero k que, para todo marcado posible de la red, se verifica que la cantidad de tokens de la plaza siempre es igual o menor a k. Es decir:


\begin{equation}
    \exists \ k \ \in \ N \ / \ \forall \ M \ \in \ marcados (PN) \Rightarrow M(P) \leq k
\end{equation}

\noindent Por otro lado, se dice que la red está \textbf{k-limitada} si todas las plazas que contiene son \textbf{k-limitadas}. \\ \par

A partir de la definición de limitación surgen varios conceptos, entre los cuales se encuentran los siguientes:
\begin{itemize}
    \item Una red de Petri es \textbf{segura} si todas sus plazas son \textbf{1-limitadas}. Esto significa que nunca puede darse un disparo si la plaza de llegada ya contiene un $token$.
    
    \item Una red de Petri es \textbf{cíclica} si siempre existe la posibilidad de alcanzar el marcado inicial desde cualquier otro marcado alcanzable. Es decir, \break $\forall \ M  \in \ marcados(PN),\ M_0$ es dinámicamente alcanzable desde $M$.
    
    \item Una red de Petri es \textbf{repetitiva} si existe una secuencia de disparos $\sigma$ que contiene todas las transiciones de la red y existe un marcado $M$ que para el cual $M \xrightarrow{\sigma} M$. Es decir, existe una secuencia de disparos que contiene todas las transiciones y que lleva la red del marcado actual al mismo marcado.
    
    \item  Una red de Petri es \textbf{conservativa} si se cumple que $\forall \ M\ \in marcados (PN)$, el número total de $tokens$ en el marcado M es igual al número de tokens en el marcado $M_0$. En otras palabras, la red siempre contiene la misma cantidad de marcas.
    
\end{itemize}

\subsection{Propiedades de vivacidad}
La \textbf{vivacidad} de una transición indica que, en todo instante de la  evolución de la red, su disparo es posible. Este concepto es particularmente relevante ya que determina si la ejecución de la red puede  o no detenerse en un estado determinado. A partir de esto se puede definir la vivacidad de una red de Petri. Esta propiedad indica que una red $N = \{P, T, I^- , I^+ , M_0 \}$ es viva para un marcado si todas sus transiciones lo son.

Por otro lado, la \textbf{cuasi-vivacidad} de una transición expresa la posibilidad de dispararla al menos una vez a partir de un marcado inicial $M_0$. De la misma manera que para el caso de la vivacidad, una red de Petri es cuasi-viva si todas sus transiciones lo son.

Gracias a esta última definición, se puede definir la vivacidad en función de la cuasi-vivacidad de la siguiente manera: una transición es viva si la misma es cuasi-viva en la red para todo marcado alcanzable desde $M_0$. \\ \par

\noindent La vivacidad está directamente asociada con la ausencia de \textbf{deadlock} o interbloqueo. En términos generales, el deadlock es el bloqueo permanente de un conjunto de procesos o hilos de ejecución en un sistema concurrente que compiten por recursos del sistema o bien se comunican entre ellos. En el caso de una red de Petri, esto suele ocurrir cuando dos o más transiciones esperan mutuamente por el disparo de la otra, produciendo el bloqueo permanente de esa porción de la red. 
Una red de Petri viva garantiza la ausencia de interbloqueo sin importar la secuencia de disparos.


\subsection{Alcanzabilidad de una red de Petri}
La \textbf{alcanzabilidad} de una red de Petri es fundamental para el análisis de las propiedades dinámicas de un sistema. A grandes rasgos, permite determinar si el sistema modelado puede alcanzar un determinado estado.


Un marcado $M_i$ es alcanzable desde $M_0$ si existe una secuencia finita de disparos $\sigma$ tal que $M_0 \xrightarrow{\sigma} M_i$.

Los marcados alcanzables por la red pueden ser representados como nodos de un grafo o árbol, donde los arcos indican los disparos necesarios para alcanzar dicho marcado. El algoritmo para determinar el árbol de alcanzabilidad de una red de Petri será explicado con detalle en el desarrollo del proyecto.

Entonces, el grafo de alcanzabilidad $A$ se define como el menor conjunto que cumpla con las expresiones 2.17 y 2.18.

\begin{equation}
    M_0 \ \in \ A
\end{equation}

\noindent Esta condición simplemente aclara que el marcado inicial de la red siempre forma parte del grafo de alcanzabilidad, ya que el mismo no requiere ninguna secuencia de disparos para ser alcanzable.

\begin{equation}
    \forall \ M \ sii \ M \xrightarrow{\sigma} M_i \Rightarrow M_i \ \in \ A
\end{equation}


\noindent Esto significa que para cualquier marcado $M$, si a partir del mismo puede alcanzarse otro marcado $M_i$, entonces $M_i$ forma parte del grafo. \\ \par

\noindent \textsc{Ejemplo} \ Suponiendo una red simple como la de la figura \ref{fig:rdp2.4_estadosposibles}, compuesta por tres plazas y una única transición $T_1$ , se puede afirmar que sólo hay tres estados posibles en la red:

\begin{figure}[H]
	\centering
	\includegraphics[scale=1.00]{Figures/marco teorico/imag4.png}
	\caption{Red de Petri 3 estados posibles.}
	\label{fig:rdp2.4_estadosposibles}
  \end{figure}
  
\begin{itemize}
    \item El marcado inicial [2, 3, 0]
\end{itemize}

Luego del segundo disparo no existen disparos posibles. Por lo tanto, la red de la figura \ref{fig:rdp2.4_estadosposibles} produce el grafo de alcanzabilidad  mostrado en la figura \ref{fig:rdp2.5_grafo}. \\

\begin{figure}[H]
	\centering
	\includegraphics[scale=0.5]{Figures/marco teorico/grafoalcanzabilidad.png}
	\caption{Grafo de alcanzabilidad.}
	\label{fig:rdp2.5_grafo}
  \end{figure}

\subsection{Sifones y Trampas} \label{sec:tysalg2}
Los conceptos de sifón y trampa están directamente relacionados con las propiedades de \textbf{interbloqueo} y \textbf{vivacidad} de una red de Petri.

Un \textbf{sifón} se define como un subconjunto no vacío de plazas S para el cual se cumple que el subconjunto de transiciones entrantes a S está contenido dentro del subconjunto de transiciones salientes de S.
En otras palabras, un grupo de plazas es un sifón si, una vez que un token sale del grupo de dichas plazas, el mismo nunca puede volver a entrar. 
Decimos que un sifón S es \textbf{mínimo} si no contiene otro sifón como un subconjunto. En un sifón mínimo debe existir al menos dos lugares; de lo contrario, la estructura restante no puede considerarse un sifón.
En la figura \ref{fig:rdp2.6-sifontrampa} se puede observar una red que contiene una trampa y un sifón.

\begin{figure}[H]
	\centering
	\includegraphics[scale=1.00]{Figures/marco teorico/imag6.png}
	\caption{Sifón y trampa.}
	\label{fig:rdp2.6-sifontrampa}
  \end{figure}
  
Tomando el subconjunto de plazas $S = \{P_1 , P_2 \}$ se deben obtener los siguientes subconjuntos de transiciones:
\begin{itemize}
    \item El subconjunto $\bullet S = \{T_2, T_3 \}$ será aquel compuesto por las transiciones entrantes a las plazas que componen S.
    \item El subconjunto $S \bullet = \{T_2, T_3\} $ será aquel compuesto por las transiciones entrantes a las plazas que componen S.
\end{itemize}

\noindent Por propiedad de los sifones, para que S pueda considerarse como tal debe cumplirse que:
\begin{equation}
    \bullet S \ \subseteq \ S \bullet 
\end{equation}

En este caso, se comprueba que ${T_2 , T_3} \subseteq \ \{T_1 , T_2 , T_3 \}$, quedando demostrado que $\{P_1,P_2\}$ es en efecto un sifón y que, si la transición $T_1$ se dispara, el token removido de $P_1$ nunca volverá a ingresar al subconjunto. \\ \par

Por otro lado, una \textbf{trampa} se define como un subconjunto de plazas G para el cual se cumple que el subconjunto de transiciones salientes de G está contenido dentro del subconjunto de transiciones entrantes a G. Esto quiere decir que un conjunto de plazas constituyen una trampa si una vez que un token entra dicho grupo éste nunca vuelve
a salir.

Siguiendo con el ejemplo de la figura \ref{fig:rdp2.6-sifontrampa}, se analizará el subconjunto de plazas $G = \{P_3 , P_4\}$.  Los subconjuntos de transiciones serán:
\begin{itemize}
    \item El subconjunto $\bullet G = \{T_1 , T_4 , T_5 \}$ será aquél compuesto por las transiciones entrantes a las plazas que componen G.
    \item El subconjunto $G \bullet = \{T_4, T_5 \}$ será aquél compuesto por las transiciones salientes de las plazas que componen G.
\end{itemize}

\noindent Por propiedad de las trampas, para que G pueda considerarse como tal, debe cumplirse que:
\begin{equation}
    G \bullet \ \subseteq \ \bullet G
\end{equation}

Propiedad que es simplemente comprobable ya que ${T_4 , T_5} \subseteq \{T_1 , T_4 , T_5 \}$, demostrando que $\{P_3 , P_4\}$ es una trampa.


\subsection{Invariantes de plazas y transiciones}
Las invariantes de una red son propiedades independientes tanto del marcado inicial como de la secuencia de disparos, y pueden asociarse a ciertos subconjuntos de plazas o de transiciones; con lo cual surgen dos conceptos:

\subsubsection{P-invariantes}
Una \textbf{invariante de plazas} o \textbf{P–invariante} es un conjunto de plazas cuya suma de tokens no se modifica con una secuencia de disparos arbitraria. 
Esto se puede observar en el ejemplo de la figura \ref{fig:rdp2.7_p-invariante}:

\begin{figure}[H]
	\centering
	\includegraphics[scale=0.8]{Figures/marco teorico/imag7.png}
	\caption{P-invariantes de la red de Petri.}
	\label{fig:rdp2.7_p-invariante}
  \end{figure}
 
 \noindent Tras el análisis de la red , se obtienen las siguientes invariantes de plazas:
 
 \begin{equation}
    \begin{array}{cc}
        m(P_1) + m(P_2) = 1  \\
        m(P_3) + m(P_4) = 1   \\
        m(P_2) + m(P_4) + m(P_5) = 1 
    \end{array}
 \end{equation}

El primer ítem expresa que la sumatoria de tokens en las plazas $P_1$ y $P_2$ siempre será igual a uno, afirmación completamente observable al mirar la red de la figura \ref{fig:rdp2.7_p-invariante}. Estas declaraciones implican la siguiente consecuencia:

\begin{equation}
    I . x = 0
\end{equation}

\noindent donde $I$ es la matriz de incidencia y $x$ es un vector característico de un subconjunto $Q$ de las plazas que forman parte de la invariante (un uno en una posición indica que esa plaza es parte de la invariante y un cero indica lo contrario). A partir de esto surge la siguiente fórmula:

\begin{equation}
    \sum_{P \in \bullet t \cap Q} W(p,t) = \sum_{P \in t \bullet \cap Q} W(t,p)
\end{equation}

\noindent la cual puede ser expresada en función de un vector t de la siguiente manera:

\begin{equation}
    \sum_{P \in \bullet t \cap Q} t(p) = \sum_{P \in t \bullet \cap Q} t(p)
\end{equation}

\noindent Esto quiere decir que:

\begin{equation}
    \sum_{P \in (t \bullet \cup \bullet t) \cap Q} t(p) = 0 \ \ y \  \sum_{P \in Q} t(p) = 0
\end{equation}

Si reemplazamos Q por los vectores característicos $I_q$ , estas dos igualdades pueden escribirse como:

\begin{equation}
    \sum_{P \in Q} t(p) I_q(p) = 0 \ \ y \  \sum_{p \in P} t(p)I(p) = 0
\end{equation}

\noindent Lo cual es simplemente la definición del producto escalar entre dos vectores:

\begin{equation}
    t.I_q = 0
\end{equation}

\noindent Como los disparos son arbitrarios, podemos establecer la siguiente relación:
\begin{equation}
    t_j I_q; \forall \ t_j \ \in \ T \Longleftrightarrow \ I^T I_q = 0
\end{equation}

\noindent donde $I^T$ es la matriz de incidencia transpuesta.


\subsubsection{T-invariantes}
Un \textbf{invariante de transición} ó \textbf{T-invariante} es el conjunto de transiciones que deben dispararse para que la red de Petri retorne a su estado inicial.

Como se mencionó en el apartado anterior, para el cálculo de las P-invariantes se hace uso de la ecuación $I . x = 0$, siendo $I$ la matriz de incidencia y $x$ un vector característico constituido por las plazas que forman parte de la invariante. En este caso, para el cálculo de los vectores que constituyen las T-invariantes, la ecuación asociada será similar a las P-invariantes, a diferencia que se hace uso de  $I^T$ en vez de $I$:

\begin{equation}
    I^T . x = 0
\end{equation}

Aquí, a diferencia de las P-invariantes, el vector x está constituido por el conjunto de transiciones que deben dispararse para que la red retorne al estado inicial. Un "1" en una posición indica que esa transición es parte de la invariante y un "0" indica lo contrario.

Tomando la figura \ref{fig:rdp2.7_p-invariante} como ejemplo y se obtienen las siguientes invariantes de transiciones:

\begin{equation}
    T-invariantes = 
    \begin{pmatrix}
         T0 & T1 & T2 & T3  \\
         0 & 1 & 0 & 1  \\
         1 & 0 & 1 & 0  
    \end{pmatrix}
\end{equation}

Ambos vectores cumplen la condición planteada ($I^T .\ x = 0$) y si se observa la imagen en cuestión, se puede apreciar que la red retorna a su estado inicial si las transiciones especificadas en los vectores se disparan.


%-----------------------------------------------------
% Monitores (SECTION 4)
%-----------------------------------------------------
\section{Concurrencia y sincronización} \label{sec:monitor}
En los sistemas de computación actuales conviven múltiples procesos que cooperan para lograr determinados objetivos y compiten por recursos del sistema, entre ellos el procesador, la memoria RAM, los puertos de entrada/salida, etc.\\
Dado que generalmente el numero de procesos de un sistema supera ampliamente el numero de recursos, se deben establecer formas de comunicación y sincronización entre ellos que hagan que el sistema funcione correctamente. 

\par En ésta sección se definirá cuando dos programa son concurrentes y/o paralelos y las condiciones que deben cumplirse para que dos secciones de código fuente puedan ser ejecutadas de manera concurrente. Luego, se verá que la ejecución concurrente de procesos trae aparejados ciertos problemas como el interbloqueo y la inanición.\\
Por esta razón deben ejecutarse ciertos mecanismos de control para garantizar la correcta ejecución de los programas, entre ellos, los semáforos y monitores.\\
El objetivo de esta sección es que el lector adquiera una idea general sobre la programación concurrente y sobre los problemas inherentes a la misma.


\subsection{Concurrencia y paralelismo}
Dos procesos serán concurrentes cuando la primera instrucción de uno de ellos se ejecuta después de la primera instrucción de otro proceso y antes de la última.
No es necesario que estos se ejecuten al mismo tiempo, basta con el hecho de que se intercalen sus instrucciones. En caso de ejecutarse al mismo tiempo se dice que hay programación paralela.
La programación concurrente es un paralelismo potencial, dependiente del hardware que lo soporte  \cite{mendez}.

\subsubsection{Problemas inherentes a la programación concurrente}
La intercalación de instrucciones de diferentes procesos, debe ser bien manejada y controlada dado que puede producir mal funcionamiento del sistema. Los problemas inherentes a la concurrencia son:
\begin{itemize}
    \item \textbf{Exclusión mutua}: Se debe garantizar que si un proceso adquiere el recurso los demás deberán esperar hasta que sea liberado.

    \item \textbf{Condición de sincronización}: Hay situaciones en las que un proceso debe esperar que ocurra algún determinado evento para poder continuar. Por ello se debe garantizar que si el evento \textbf{no} ocurrió, el proceso \textbf{no} continúe. 
    
    \item \textbf{Interbloqueo}: Esta situación se produce cuando todos los procesos están esperando un evento que nunca ocurrirá. Se debe garantizar que estas situaciones no ocurran.
    
    \item \textbf{Inanición}: En este caso, el sistema en su conjunto hace progresos, pero existe un grupo de procesos que nunca progresaran pues no se les otorga tiempo de procesador para hacerlo.
\end{itemize}

\subsubsection{Exclusión mutua}
La exclusión mutua implica que dos o más procesos intentan acceder a un único recurso compartido entre ellos pero solo uno puede acceder a cada instante.
Cuando se da un caso de estas características, se desea que todo lo que necesite hacer unos de los procesos sobre el recurso se realice de manera indivisible y luego lo deje disponible para que otro proceso ejecute sus instrucciones sobre el recurso.
\par A la porción de código que se desea que se ejecute de manera indivisible o atómica se le llama \textbf{sección crítica}. Se debe lograr que todas las instrucciones dentro de la sección crítica se ejecuten en exclusión mutua lo que implica que el hecho de que cuando uno de los procesos este ejecutando esa porción de código el resto no podrá hacerlo. \\
\\
\textbf{Solo uno de los procesos podrá estar en la sección crítica en un instante dado.}

\begin{figure}[H]
	\centering
	\includegraphics[scale=0.5]{Figures/marco teorico/seccion.jpg}
	\caption{Sección crítica.}
	\label{fig:seccioncritica}
  \end{figure}
  
En la figura \ref{fig:seccioncritica} se observa como dos procesos $P1$ y $P2$ intentan ejecutar una porción de código de una sección crítica. La imagen de la izquierda (a) muestra que el proceso $P1$ consigue ingresar a ejecutar la sección crítica. La imagen de la derecha (b) muestra que el proceso $P2$ puede ingresar solo cuando el proceso $P1$ ya no esta en la misma.

\noindent La exclusión mutua se puede representar de la siguiente manera.
\begin{figure}[H]
	\centering
	\includegraphics[scale=0.7]{Figures/marco teorico/mutex.png}
	\caption{Sección crítica.}
	\label{fig:exmutuardp}
  \end{figure}

En la figura \ref{fig:exmutuardp} la plaza \textit{MUTEX} esta limitada a un único token y el análisis de invariantes de plazas demuestra formalmente la propiedad de exclusión mutua entre los procesos $P1$ y $P2$. 

\begin{equation}
    m(EjecutandoSCP1) + m(MUTEX) + m(EjecutandoSCP2) = 1
\end{equation}

\subsection{Interbloqueo} \label{sec:Interbloqueo}
En un sistema donde los procesos compiten por limitados recursos, pueden producirse demandas contradictorias de los mismos. Por ejemplo, si existen dos procesos, \textit{A} y \textit{B}, y dos recursos \textit{R1} y \textit{R2}, y ambos procesos necesitan los dos recursos para proseguir, si el proceso \textit{A} toma el recurso \textit{R1} y el \textit{B} el recurso \textit{R2}, ambos procesos se bloquearan a la espera del otro recurso, pero ninguno liberará el recurso que posee hasta no conseguir los dos. A esta situación se la conoce como interbloqueo \cite{stallings}. 

\subsubsection{Condiciones para producir interbloqueo: }
\noindent Deben presentarse tres condiciones de gestión para que sea posible un interbloqueo:
\begin{enumerate}
    \item \textit{Exclusión mutua}: sólo un proceso puede utilizar un recurso en cada momento. Ningún proceso puede acceder a un recurso que se ha asignado a otro proceso.
    
    \item \textit{Retención y espera}: un proceso puede mantener los recursos asignados mientras espera la asignación de otros recursos.
    
    \item \textit{No apropiación}: ningún proceso podrá ser forzado a abandonar un recurso que retiene.
    
    \item \textit{Espera circular}: existe una cadena cerrada de procesos donde cada proceso retiene un recurso que necesita un proceso que le sigue en la cadena.
\end{enumerate}

Las tres primeras condiciones son necesarias pero no suficientes para que exista interbloqueo.
La cuarta es una consecuencia potencial de las tres primeras y, en caso de darse, generará una \textbf{espera circular irresoluble}. Esta es de hecho la definición de interbloqueo.

\subsection{Sincronización}
Para solucionar los problemas inherentes a la programación concurrente, se utiliza lo que se llama \textit{sincronización entre los procesos} . \\
Se habla de sincronización , en general, cuando determinados fenómenos ocurren o deben ocurrir en un determinado orden o a la vez.
\par Para la computación, la sincronización es representada por las señales que se envían los procesos para colaborar entre ellos o para indicar el estado de recursos compartidos, para indicar que un evento o acción ocurrió o no y determinar la continuidad o no de un proceso, etcétera.
\par La condición de sincronización puede definirse como la propiedad requerida para que un proceso no realice ninguna acción o evento hasta que otro proceso realice una determinada acción o evento.


\subsubsection{Semáforos}
Los semáforos son un sistema de señales simples utilizadas por los procesos para comunicarse entre ellos y lograr la sincronización requerida.
Estos tienen una variable de sincronización, del tipo entero no negativo, que indica la cantidad de recursos disponibles. Sobre esta se realizan dos tipos de operaciones:

\begin{itemize}
    \item \textit{wait}: decrementa el valor del semáforo solo si este es mayor que cero. Este proceso indica que se utiliza uno de los recursos que controla el semáforo. Si el valor del semáforo al momento de ejecutar la operación \textit{wait} es cero, indica que no hay recursos disponibles y el proceso deben bloquearse hasta que se libere alguno.
    \item \textit{signal}: es la acción de liberar un recurso que estaba siendo utilizado. En caso de haber algún proceso bloqueado en el semáforo se lo despierta para que utilice el recurso. De no existir algún proceso, se incrementa el valor del semáforo.
\end{itemize}

Los semáforos son primitivas con las cuales es difícil expresar una solución a grandes problemas de concurrencia, ya que tienen algunas debilidades:
\begin{itemize}
    \item La omisión de una de las primitivas puede corromper el funcionamiento de un sistema concurrente.
    \item El control de concurrencia es responsabilidad del programador.
    \item Las primitivas de control se encuentran esparcidas por todo el código, lo que hace muy difícil la corrección de errores y el mantenimiento del mismo.
\end{itemize}

Debido a estas razones existe otro mecanismo de software para el control de concurrencia denominado \textbf{monitor}.

\subsubsection{Monitores}

Como se dijo, los semáforos, generalmente se encuentran dispersos en el código, lo que lo hace más confuso y muchas veces es difícil notar cual es el recurso compartido y determinar si está correctamente sincronizado. Por ello, se necesita un sistema que sea igual de versátil que los semáforos pero que permita efectuar un control más estructurado de la exclusión mutua. Una herramienta con estas características fue propuesta por \textit{C.A.R Hoare} en 1975 y es conocida como \textbf{\textit{monitor}}.\\
Un \textbf{monitor} es un mecanismo de abstracción de datos, lo que permite representar de forma abstracta un recurso compartido mediante variables que indican su estado. El acceso a esas variables solo es posible a través de un conjunto de funciones/métodos que el monitor exporta al exterior.\\

Un monitor se compone de los siguientes elementos:
\begin{itemize}
    \item Un \textit{conjunto de variables} locales que pueden denominarse permanentes. Se utilizan para almacenar el estado interno del recurso que representa el monitor. Se denominan permanentes ya que permanecen sin modificarse entre dos llamadas consecutivas al monitor y solo pueden ser accedidas dentro del mismo.
    
    \item Un \textit{código de inicialización} que se ejecuta antes que la primera instrucción ejecutable del programa y su fin es inicializar las variables permanentes. 

    \item Un \textit{conjunto de procedimientos internos} que manipulan las variables permanentes.
    
    \item Una \textit{declaración de los procedimientos} que son \textit{exportados} y pueden ser accedidos por los procesos activos externos.
\end{itemize}

\paragraph{Exclusión mutua en monitores}
\hfill
\par El control de la exclusión mutua en un monitor se basa en la existencia de una cola asociada al mismo que se denominara \textit{cola del monitor}. La gestión de esta cola se realiza de la siguiente manera:

\begin{enumerate}
    \item Cuando un proceso activo está dentro del monitor (ejecutando alguno de los procedimientos del mismo) y aparece otro proceso activo que intenta ejecutar otro (o el mismo) procedimiento, el código de acceso al monitor bloquea el proceso que realiza la llamada y lo inserta en la cola del monitor (con política FIFO). Así, se impide que dos procesos estén al mismo tiempo dentro del monitor.

    \item Cuando un proceso activo abandona el monitor, este ultimo selecciona el proceso que esta al frente de la cola del monitor y lo desbloquea para que comience a ejecutar las operaciones que le solicitó al monitor. Si la cola estaba vacía, el monitor queda libre y cualquier proceso activo que llame alguno de sus procedimientos entrará al monitor.
\end{enumerate}

Esto asegura que las variables compartidas nunca son accedidas concurrentemente. Una cuestión importante es que la responsabilidad de bloquear un proceso es del monitor y no del proceso.
Al comparar este sistema con un semáforo se ve que en el caso de los semáforos son los propios procesos activos los que manejan las políticas de acceso a variables compartidas.

\paragraph{Condición de sincronización en monitores}
\hfill
\par El procedimiento anterior sólo controla la exclusión mutua, es decir, pueden haber casos donde un proceso activo tenga acceso al monitor (ha obtenido la exclusión mutua al mismo) pero no puede seguir su ejecución debido a alguna razón, tal como un \textit{buffer} lleno que no puede ser escrito. En estos casos, es necesario bloquear ese proceso y permitir que otro ingrese al monitor. Para realizar esto surgen nuevos componentes que deben formar parte del monitor:

\begin{itemize}
    \item Variables de condición
    \begin{itemize}
        \item Las mismas son declaradas en el monitor. 
        \item Deben ser privadas. 
        \item Tienen una cola \textit{FIFO} asociada.
    \end{itemize}
    \item Operaciones sobre las variables de condición
    \begin{itemize}
        \item \textit{Delay}
        \item \textit{Resume}
        \item \textit{Empty}
    \end{itemize}
\end{itemize}

La operación delay se realiza sobre una variable de condición. Si se supone la existencia de una variable C, al realizar \textit{delay(C)}, el proceso que la ejecutó libera el mutex del monitor, se bloquea y se envía al final de la cola asociada a la condición C. A diferencia de la operación wait que se utiliza en los semáforos, delay bloquea al proceso incondicionalmente.\\

\par La operación resume, cuando se realiza sobre una variable C, libera al primer proceso que ejecutó \textit{delay(C)}. Si la cola está vacía, resume es una operación nula. Por otra parte, la función empty simplemente devuelve un valor boolean true si una cola se encuentra vacía o false en caso contrario.\\

Con lo dicho hasta este punto, se podría decir que si un proceso que se está ejecutando dentro del monitor ejecuta una operación \textit{resume(C)}, se desbloqueará un proceso de esa cola que continuará con su ejecución dentro del monitor también. Esto lleva a una situación con dos procesos dentro del monitor, lo que violaría la exclusión mutua. Para evitar esto, el proceso que ejecuta el \textit{resume} cederá la exclusión mutua al recien desbloqueado. Y espera en una cola diferente llamada \textbf{cola de cortesía} hasta que el proceso recién desbloqueado por el \textit{resume(C)} termine su ejecución teniendo preferencia por sobre los procesos esperando en otras colas.


\begin{figure}[H]
	\centering
	\includegraphics[scale=0.5]{Figures/marco teorico/monitor.png}
	\caption[Monitor.]{Monitor \footnotemark.}
	\label{fig:monitor}
\end{figure} \footnotetext{Figura adaptada del Proyecto Integrador \textit{Desarrollo de IP cores con procesamiento de Redes de Petri Temporales para sistemas multicore en FPGA} \cite{papermico, tesisnonino}.}  
  

\subsubsection{Implementación de monitores con redes de Petri}
Es posible ver a un monitor formado por dos secciones: primero, la referida a la política de colas que se debe ejecutar para lograr que sólo un proceso esté en el monitor, que se bloqueen los procesos que no tienen los recursos y que se desbloqueen los que obtuvieron los recursos, y segundo, la lógica con que se administran los recursos.\\
En la figura \ref{fig:monitor} se puede observar que existe una cola de entrada, para los procesos que aún no ingresaron al monitor y desean hacerlo, una serie de colas, una por cada recurso (cada condición de sincronización) y una cola de cortesía para que  proceso dentro del monitor pueda, de manera segura, ceder la exclusión mutua al cambiar el estado de un recurso.\\

\par \textbf{\textit{Una red de Petri puede realizar el trabajo de la lógica del monitor}}, es decir, la administración y sincronización de recursos disponibles; esto es cuando el vector de estado que resultó del disparo no tiene componentes negativas es porque los recursos están disponibles, el disparo de la transición solicitada conduce a un nuevo estado valido. De no ser así, en caso de existir algún valor negativo en el nuevo vector de estado, se llegó a un estado no válido que indica que el recurso no está disponible. Además el vector de estado indica si el disparo ha devuelto o tomado recursos. Si la cantidad de tokens para un recurso dado disminuye, significa que se han tomado recursos, en caso contrario, que se han devuelto recursos \cite{papermico, tesisnonino}.

\begin{figure}[H]
	\centering
	\includegraphics[scale=0.5]{Figures/marco teorico/monitormico.png}
	\caption[Monitor.]{Monitor \footnotemark.}
	\label{fig:monitormico}
\end{figure} \footnotetext{Figura adaptada del paper publicado por \textit{Micolini, Orlando \& Ventre, Luis \& Cebollada, Marcelo \& Eschoyez, Maximiliano} \cite{papermico}}

Por lo tanto, el monitor integra la red de Petri (lógica), la política y las acciones, conformando un sistema heterogéneo. La importancia de la metodología aquí planteada radica en desacoplar la lógica de la política y las acciones, con el fin de obtener un sistema resultante modular, simple, mantenible y verificable. 

En la figura \ref{fig:monitormico} se expone la arquitectura modular de un sistema reactivo y guiado por eventos. Para el interés de este proyecto sólo se investigó sobre la lógica del sistema y específicamente sobre como desbloquear las redes de Petri (del tipo S³PR) que la representa. 


%----------------------------------------------------------------------------------------
%	SECTION 5
%----------------------------------------------------------------------------------------
\section[S³PR]{S³PR}
Definimos la clase de los procesos secuenciales simples (S²P); luego, lo extendemos para modelar el uso de recursos (la clase de S²PR) y, finalmente, definimos la clase de sistemas de procesos secuenciales simples con recursos (S³PR) por la composición neta de S²PR a través de un conjunto de lugares comunes \cite{libro2}.

\subsection{Definición de S²P}
Un sistema de proceso secuencial simple (S²P) es una red de petri $N = (P \cup \{P^0\}, T, F)$ donde:
\begin{enumerate}
    \item $P \neq \emptyset,\ p^0 \notin P$ ($p^0$ llamada plaza idle);
    \item N es una máquina de estado fuertemente conectada
    \item Cada circuito de N contiene la plaza $p^0$.
\end{enumerate}

\noindent La tercera condición impone una propiedad de "terminación" \ a los procesos de trabajo que estamos considerando: si un proceso evoluciona, terminará.

\begin{figure} [H]
    \centering
    \includegraphics[scale=0.4]{Figures/marco teorico/s2p_chica.png}
    \caption[Red de Petri S²P.]{Red de Petri S²P \footnotemark.}
    \label{fig:rdp_s2p}
\end{figure}  \footnotetext{Figura adaptada del libro \textit{Deadlock Resolution in Automated Manufacturing Systems} \cite{libro2} .}

\noindent En la figura \ref{fig:rdp_s2p} se puede observar la plaza idle ($P_1$) destacada en lila. \\

\par Definimos ahora un proceso secuencial simple con recursos (S²PR), como un S²P que necesita el uso de un recurso único en cada estado que no sea el estado idle. Debido a que las interacciones con el resto de los procesos en el sistema se realizarán compartiendo el conjunto de recursos, es natural suponer que en el estado idle no hay interacción con el resto del sistema y, por lo tanto, no se utiliza ningún recurso en este estado.

\subsection{Definición de S²PR}
Un sistema de proceso secuencial simple con recursos (S²PR) es una red de Petri 
$N = \langle P \cup \{p^0\} \cup P_R, T, F \rangle $ tal que:
\begin{enumerate}
    \item La subred generada por $X = P \cup \{p^0\} \cup T$ es una S²P.
    \item $P_R \neq \emptyset$ y $(P \cup \{p^0\}) \cap P_R = \emptyset$
    \item $\forall p \in P.\ \forall \ t \in \bullet p.\ \forall \ t' \in p \bullet. \ \bullet t \cap P_R = \ t' \bullet \cap P_R = \{r_p\}$
    \item Las dos siguientes declaraciones son verificadas:
    \begin{enumerate}[a)]
        \item $\forall \ r \in \ P_R. \ \bullet \bullet r \cap P = r \bullet \bullet \cap P \neq \emptyset$
        \item $\forall \ r \in \ P_R. \ \bullet r \cap r \bullet = \emptyset$
    \end{enumerate}
    \item $\bullet \bullet (p^0) \cap P_R = (p^0) \bullet \bullet \cap P_R = \emptyset$
\end{enumerate}

\noindent $P_R$: conjunto de plazas de recursos. \\
$P$: conjunto de plazas de estado.

\begin{figure} [H]
	\centering
	\subfloat [] {
	    \includegraphics[scale=0.65]{Figures/marco teorico/imag8.jpg}
    	\label{fig:rdp_a}
    }
    \subfloat [] {
        \includegraphics[scale=0.65]{Figures/marco teorico/imag9.jpg}
    	\label{fig:rdp_b}
    } 
   \caption[Redes de Petri S²PR.]{Redes de Petri S²PR. (A) RdP (N1,M1) - (B) RdP (N2,M2) \footnotemark. } 
\end{figure} \footnotetext{Figura adaptada del libro \textit{Deadlock Resolution in Automated Manufacturing Systems} \cite{libro2} .}

\noindent Esto se puede ver representado en las figuras \ref{fig:rdp_a} y \ref{fig:rdp_b}. Las plazas idles ($P_1, P_5$) destacadas en lila, mientras que las plazas recursos ($P_9, P_{10}, P_{11}$) destacadas en naranja.\\


\par Para una plaza de estado dado $p \in P$, la plaza $r \in P_R$ dado por la condición 3 en la definición modela el recurso utilizado en este estado. 
Para un $r \in P_R$, denotaremos $H(r) = (\bullet \bullet r) \cap P$ conjunto de plazas complemento de r (estados que usan r). La condición 4 en la definición anterior impone que dos estados adyacentes de un proceso de trabajo (WP) (ambos diferentes del estado inactivo) no pueden usar el mismo recurso. Esto no es una restricción, ya que desde la perspectiva de la vivacidad, dos estados adyacentes que usan el mismo recurso puede colapsar en un estado único, preservando las propiedades de comportamiento de la red. 

Nótese que $\bullet r$ representa las transiciones entrantes a las plazas r. \\
$\bullet \bullet r = \sideset{}{_{t \in \bullet r}} \bigcup \bullet t$ es el conjunto de plazas de entrada de todas las transiciones de entrada de la plaza r. De manera similar,  $r \bullet \bullet = \sideset{}{_{t \in r \bullet}} \bigcup \bullet t$ representa el conjunto de todas las plazas de salida de todas las transiciones de salida de la plaza r. Por ejemplo, en la figura \ref{fig:s3pr}: $\bullet p9 = \{t_2,t_8\}$ y $\bullet \bullet p_9 =  \bullet t_2 \cup  \bullet t_8 = \{p_2, p_{10}, p_8 \}$.  $p_9 \bullet = \{ t_1, t_7 \}$ y $p_9 \bullet \bullet = t_1 \bullet \cup \ t_7 \bullet = \{ p_2, p_{10}, p_8 \}$. Claramente  $\bullet \bullet p_9 = p_9 \bullet \bullet$

Sea $N = \langle P \cup \{p^0\} \cup P_R, T, F \rangle $ una S²PR. Un marcado inicial $m_0$ es llamado un marcado inicial aceptable para N sii:
\bigskip

\begin{enumerate}
    \centering
    \item $m_0 (p^0) \geq 1$
    \item $m_0 (p) = 0, \ \forall \ p \in \ P$
    \item $m_0 (r) \geq 1, \ \forall \ r \in \ P_R$
\end{enumerate}
\bigskip

Observe que una marca aceptable asigna al menos un token en la plaza idle (entonces, suponemos que, inicialmente, cada marca -token- de cada proceso está inactiva) y al menos un token en cada recurso, es decir, hay al menos una marca de cada recurso en el sistema. Está claro que si existe un recurso para el que no hay marca, el sistema no está bien definido, porque puede tener alguna secuencia de producción que no se puede llevar a cabo. 
\bigskip

\subsection{Definición de S³PR}
Entonces se puede concluir que un sistema de proceso secuencial simple con recursos 
$S^3PR = \{N = \sideset{}{_{n=1}^{k}} \bigcirc = N_i = (P \cup P^0 \cup P_R, T, F)\}$
se define como la unión de un conjunto de redes, de tipo $S^2PR= \{ N_i = (P_i \cup {pi^0} \cup P_{Ri}, T_i, F_i) \}$. Como se puede observar en la figura \ref{fig:s3pr}. \\

\begin{figure} [H]
    \centering
    \includegraphics[scale=0.8]{Figures/marco teorico/imag10.jpg}
    \caption[Composición de una red de Petri S³PR.]{Composición de una red de Petri S³PR.\footnotemark}
    \label{fig:s3pr}
\end{figure}  \footnotetext{Figura adaptada del libro \textit{Deadlock Resolution in Automated Manufacturing Systems} \cite{libro2} .}

\newline
Sean $(N_1,M_1)$ y $(N_2,M_2)$ dos redes de Petri con $N_1=(P_1, T_1, F_1, W_1)$ y $N_2 =  (P_2, T_2$, $F_2, W_2)$, donde $P_1 \cap P_2 = P_c \neq \emptyset$  y $T_1 \cap T_2 = \emptyset$. $(N,M)$ con $N = (P, T, F, W)$ es la red resultante de la unión entre $(N_1,M_1)$ y $(N_2,M_2)$ a través de compartir el conjunto de plazas $P_c \ sii (1) P = P_1 \cup P_2$ ,  $T = T_1 \cup T_2$, $F = F_1 \cup F_2$, y $W(x,y) = W_i(x,y)$ si  $(x,y) \in F_i$, $i=1,2$; y (2) $ \forall p \in P_1 \setminus P_c$ , $M(p)=M_1(p)$ , $\forall p \in P_2 \setminus P_c$ , $M(p)=M_2(p)$ y $\forall p \in P_c$ , $M(p)=max\{M_1(p), M_2(p)\}$. \\


Por ejemplo: dos redes $(N_1, M_1)$ y $(N_2, M_2)$ se muestran en la figura \ref{fig:rdp_a} y figura \ref{fig:rdp_b}, respectivamente, donde $P_1= \{p_1-p_4, p_9-p_11\}$, $T_1={t_1-t_4}$, $P_2={p_5-p_{11}}$ y $T_2={t_2-t_8}$. Donde $P_1 \cap P_2 = \{p_9,p_{10},p_{11}\}$ y $T_1 \cap T_2 = \emptyset$. En la figura \ref{fig:s3pr} se observa la red resultante de la composición de $(N_1, M_1)$ y $(N_2, M_2)$ es denotado como $(N,M)$ donde $P = P_1 \cup P_2 = \{p_1-p_{11}\}$, $T = T_1 \cup T_2 = \{t_1-t_8\}$, $M(p_1) = M_1(p_1) = 10$, $M(p_2) = M_1(p_2) = 0$, $M(p_3) = M_1(p_3) = 0$, $M(p_4) = M_1(p_4) = 0$, $M(p_5) = M_2(p_5) = 10$, $M(p_6) = M_2(p_6) = 0$, $M(p_7) = M_2(p_7) = 0$, $M(p_8) = M_2(p_8) = 0$, $M(p_9) = max\{M_1(p_9),M_2(p_9)\} = 2$, $M(p_{10}) = max\{M_1(p_{10})$, $M_2(p_{10})\} = 2$, $M(p_11) = max\{M_1(p_{11}), M_2(p_{11})\} = 3$.



 % Chapter Template 
% cSpell: words Mininet prototipado parencite enrutamiento includegraphics veth  mininetonf cellcolor Nicira vswitchd interconectar multicapa datapath ovsflow resizebox flowtable netlink OVSDB dpctl ofctl vsctl rowcolor mininetovs

\chapter{Desarrollo} % Main chapter title

\label{Chapter3} % Change X to a consecutive number; for referencing this chapter elsewhere, use \ref{ChapterX}

%----------------------------------------------------------------------------------------
%	SECTION 1
%----------------------------------------------------------------------------------------

\section{Desarrollo de la investigación}

En la presente sección se busca explicar brevemente cómo se fue estructurando y desarrollando el algoritmo en cuestión. 

Las redes implementadas en esta investigación fueron redes del tipo $S^3PR$ dado que modelan la ejecución concurrente de procesos de trabajo. La finalización de alguno de estos puede iniciar más de una nueva operación. Como resultado de estas características dinámicas, pueden ocurrir dos situaciones: conflicto y deadlock. 

El conflicto puede ocurrir cuando dos o más procesos requieren un recurso común al mismo tiempo. Por ejemplo, dos estaciones de trabajo pueden compartir un sistema de transporte común o necesitar acceso al mismo almacenamiento. Una forma sencilla de resolver el conflicto es asignar un nivel de prioridad a cada uno de los procesos. 

El deadlock puede suceder al compartir dos recursos entre los dos procesos. En este caso, se puede alcanzar un estado en el que ninguno de los procesos puede continuar. Tenga en cuenta que uno de los procesos puede continuar si el conflicto se puede resolver, mientras que en el caso de deadlock no se puede hacer nada para que el sistema vuelva a funcionar.


%----------------------------------------------------------------------------------------
%	SECTION Modelo de Desarrollo
%----------------------------------------------------------------------------------------
\section{Modelo de desarrollo}
Para la elaboración del presente proyecto se optó por utilizar un modelo iterativo e incremental. A grandes rasgos, este tipo de modelo de desarrollo no es más que un conjunto de tareas agrupadas en pequeñas etapas repetitivas, las cuales inician con un análisis y finalizan con una versión nueva del algoritmo y sus conclusiones. \\

\par Se planifica un proyecto en distintos bloques temporales denominados iteraciones. Dentro de una iteración se repite un determinado proceso de trabajo sobre uno o varios objetivos, obteniéndose al final de la misma un resultado con más funcionalidades implementadas que el de la iteración anterior. La ventaja principal de este modelo es que no se debe esperar a que el sistema esté completo para que el mismo sea utilizable y operacional. \\

\par Para lograr esto, al realizar el análisis de una iteración, se especifican los objetivos que se esperan conseguir al finalizar la misma. Estos se establecen en función de los requerimientos, de los riesgos y de la evaluación de los resultados de las iteraciones precedentes. Se busca que en cada iteración los componentes logren evolucionar el producto dependiendo de aquellos completados en las iteraciones antecesoras. \\

\par \noindent La implementación del algoritmo se realizó en cuatro iteraciones:
\begin{itemize}
    \item \textbf{Iteración 1}: Procesamiento de los archivos '.html' y detección de estados de deadlock.
    \begin{enumerate}
        \item Realizar la conversión de los archivos de formato '.html' exportados del Petrinator y su posterior procesamiento para que sean utilizados como entradas en el algoritmo.
        \item Detección de estados de deadlock basándose en los fundamentos teóricos (Capítulo \ref{Chapter2}).
        \item Una vez detectado, se evita alcanzar el mismo inhibiendo el disparo de la transición que lo desencadena.
    \end{enumerate}
    
    \item \textbf{Iteración 2}: Sifones en estado de deadlock y plaza de control.
    \begin{enumerate}
        \item Detectar los sifones vacíos en estado de deadlock.  
        \item Obtener el marcado inicial de cada sifón.
        \item Detectar la transición cuyo disparo lleva al camino del deadlock.
        \item Incorporar plaza y arco de control (salida) que evite el vaciado del sifón en cuestión.
    \end{enumerate}
    
    \item \textbf{Iteración 3}: Sifones mínimos en estado de deadlock y supervisor.
    \begin{enumerate}
        \item Obtener los sifones mínimos vacíos (bad siphons) en estado de deadlock.
        \item Determinar el marcado y arcos (entrada y salida) del supervisor.
        \item Incorporar los supervisores que eviten el vaciado de los sifones en cuestión.
    \end{enumerate}
    
    \item \textbf{Iteración 4}: Mantener los T-invariantes e interfaz de salida hacia el Petrinator
    \begin{enumerate}
        \item Realizar el análisis sobre la totalidad de la red permitiendo preservar los T-invariantes de la red original.
        \item Modificar el archivo de extensión '.pflow' de la red en cuestión, para agregar plazas, arcos y quitar estos últimos en caso de ser necesario.
    \end{enumerate}
\end{itemize}

\par \noindent Las mismas serán tratadas con mayor profundidad a lo largo de este capitulo. 
%----------------------------------------------------------------------------------------
%	SECTION 3
%----------------------------------------------------------------------------------------

\section{Iteración 1: Algoritmo v1.0}
\subsection{Introducción}
En esta iteración se hizo foco en analizar el grafo de alcanzabilidad de las redes, detectando los estados de deadlock alcanzables y trazando las secuencias de disparo que desencadenan a los mismos. 
\subsection{Objetivos}
\begin{itemize}
	\item Conversión de archivos de formato '.html' (salida del software Petrinator) a archivos tipo '.txt' para facilitar la lectura de los datos en el algoritmo.
	\item Detectar el/los estado/s con deadlock y evitar el camino que desencadenan a ellos.
\end{itemize}

\subsection{Desarrollo}
\subsubsection{Herramienta a utilizar}
El software seleccionado para obtener la información necesaria para el desarrollo del algoritmo fue Petrinator \cite{petrinator} debido a que el mismo fue desarrollado dentro del Laboratorio de Arquitecturas de Computadoras (LAC) lo que nos permitió solicitar modificaciones según nuestras necesidades. 

\noindent Sus principales características son:
\begin{itemize}
	\item Una interfaz intuitiva que permite la creación, edición, almacenamiento y exportación de redes de Petri.
	\item Código abierto y bajo licencia GPL. 
\end{itemize}

\paragraph{Datos necesarios}
Se utilizaron las funcionalidades del Petrinator para obtener y exportar la siguiente información de la red:
\begin{itemize}
    \item Análisis de Invariantes (Invariant analysis)
    \begin{itemize}
        \item T-invariantes
        \item P-invariantes
    \end{itemize}
    \item Matrices
    \begin{itemize}
        \item Pre(+)
        \item Post(-)
    \end{itemize}
    \item Sifones y Trampas (Siphons and Traps)
    \begin{itemize}
        \item Sifones (plazas que lo componen)
        \item Trampas (plazas que lo componen)
    \end{itemize}
    \item Grafo de Alcanzabilidad (Reachability graph)
    \begin{itemize}
        \item Estados en deadlock
        \item Marcado inicial de los sifones
    \end{itemize}
\end{itemize}

\subsubsection{Conversión de datos}

Como puente entre el software Petrinator y el algoritmo principal, se desarrolló un algoritmo de conversión en Python “html\_to\_txt”, el cual dados los archivos '.html' exportados por el software trabaja sobre estos, con el fin de obtener los datos característicos de la red (mencionados anteriormente) en el formato adecuado de entrada que alimentará al algoritmo en desarrollo, en este caso matrices y vectores.

\subsubsection{Construcción del algoritmo}

Una vez realizada la carga y el filtrado de los datos exportados del software Petrinator, se llevó a cabo la detección de todos los estados con deadlock presentes en la red. Luego, se selecciona uno de estos y se trazan las secuencias de disparo que desencadenaron al mismo, se verifican todos los arcos de los estados que desencadenaron al mismo; si alguno de estos estados también estaban en deadlock se inhiben los mismos en el vector de sensibilización, esto se realiza recursivamente hasta encontrar algún estado que no presente deadlock dejando \textbf{solamente} este brazo habilitado en el vector de sensibilización, de esta manera nos aseguramos de no entrar en un camino que solo conduce al estado de deadlock.
\bigskip

\begin{figure}[H]
	\centering
	\includegraphics[scale=0.50]{Figures/algoritmo1/1.png}
	\caption{RdP con un estado de deadlock.}
	\label{fig:rdp3.1}
  \end{figure}

La red que se observa en la figura \ref{fig:rdp3.1} presenta 4 estados alcanzables y uno de ellos es un estado con deadlock.
\begin{itemize}
	\item $S_0[1,0,0,0]$
	\item $S_1[0,0,1,0]$
	\item \textcolor{red}{$S_2[0,0,0,1]$}
	\item $S_3[0,1,0,0]$
\end{itemize}
\bigskip

Como se mencionó con anterioridad, se parte desde el estado $S_2$ verificando que estado nos lleva al mismo. El único estado que nos lleva a este es el $S_1$ (al ejecutarse la transición $T_2$) por este motivo se inhibe esta transición del vector de sensibilizadas. Al realizar esto ahora el estado $S_1$ se convierte en un estado con deadlock, y se realiza el mismo procedimiento mencionado con anterioridad inhibiendo la transición $T_1$ que es la única que nos lleva a este estado a partir del estado $S_0$. 
De esta manera se obtiene una red sin deadlock dado que nunca se va tomar el camino de la izquierda,ya que se produjo la inanición de esta parte de la red.

\bigskip
Posteriormente se decidió aplicar la misma hipótesis sobre una red más compleja, con un mayor número de estados en su grafo de alcanzabilidad. Para esto se eligió la red representada en la figura \ref{fig:panama}

\begin{figure}[H]
	\centering
    \includegraphics[width=\textwidth]{Figures/algoritmo1/panama.png}
    \caption[RdP Canal de Panamá.]{RdP Canal de Panamá. \footnotemark}
	\label{fig:panama}
 \end{figure} \footnotetext{Figura adaptada del libro \textit{An  Algorithm  for  Deadlock Prevention  Based  on  Iterative  Siphon  Control  of  Petri  Net} \cite{paperpanama} .}

\begin{figure} [H]
	\centering
	\subfloat [Estado número 82.] {
	    \includegraphics[scale=0.45]{Figures/algoritmo1/version1_deadlock2.png}
    	\label{fig:state_deadlock1}
    }
    \subfloat [Estado número 484.] {
        \includegraphics[scale=0.45]{Figures/algoritmo1/version1_deadlock.png}
    	\label{fig:state_deadlock2}
    } 
   \caption{Estados de deadlock.} 
\end{figure}

Como se puede observar en las figuras \ref{fig:state_deadlock1} y \ref{fig:state_deadlock2} donde las transiciones $T_2$ y $T_{12}$ al ser ejecutadas llevan a los estados de deadlock; al deshabilitar la ejecución de las mismas la red sólo permite dos disparos (de las transiciones $T_1$ y $T_{13}$), luego de estos la red no se vuelve a ejecutar.
Esta solución no es viable dado que la solución alcanzada al inhibir estas transiciones no representa la ejecución del modelo original.

\subsection{Conclusiones}
Si bien se alcanzó el objetivo planteado, este primer algoritmo no era escalable dado que si bien lograba resolver el deadlock para redes simples, en el caso de redes más complejas no llegaba a converger a una solución. 
Sin embargo, gran parte de las funcionalidades implementadas fueron utilizadas en las posteriores versiones del mismo.
\bigskip

%----------------------------------------------------------------------------------------
%	SECTION 4
%----------------------------------------------------------------------------------------

\section{Iteración 2: Algoritmo v2.0}
\subsection{Introducción}
Al observar que la red presentaba un camino que llevaba al deadlock, se comenzó a analizar el porqué de este deadlock y observamos que el mismo se producía por el vaciado de al menos un sifón de la red. En esta nueva versión se hizo hincapié en evitar esta problemática.

\subsection{Objetivos}
\begin{itemize}
	\item Encontrar los sifones vacíos en estado de deadlock.
	\item Evitar estados con deadlock.
\end{itemize}

\subsection{Desarollo}
En cuanto a los datos necesarios para el funcionamiento del algoritmo se reutiliza el código que permite la conversión de los mismos implementado en la versión 1.0 del algoritmo. 
A diferencia de la versión anterior, en vez de inhibir el arco imposibilitando la ejecución del mismo lo que se hace es lo siguiente:
\begin{enumerate}
    \item Detectar el sifón vacío en el estado con deadlock.
    \item Obtener el marcado del mismo en el estado inicial.
    \item Detectar la transición cuyo disparo lleva a ese camino de deadlock.
    \item Colocar una plaza “de control” cuyo marcado es la marca del sifón menos uno, evitando de esta manera que el sifón se vacíe (dejando su marcado al menos con un token).
\end{enumerate}

\begin{figure}[H]
	\centering
	\includegraphics[scale=0.5]{Figures/algoritmo2/1.png}
	\caption{RdP con un estado de deadlock.}
	\label{fig:rdp3.2}
  \end{figure}

\begin{figure}[H]
	\centering
	\subfloat[Sifón]{
	\includegraphics[scale=0.5]{Figures/algoritmo2/sifon.png}
	\label{fig:rdp3.2sifon}
	}
	\subfloat[Grafo de alcanzabilidad.]{
	\includegraphics[scale=0.5]{Figures/algoritmo2/grafo.png}
	\label{fig:rdp3.2grafo}
	}
	\caption{Análisis de la figura \ref{fig:rdp3.2}.}
    \end{figure}


En la figura \ref{fig:rdp3.2} se puede observar que si la red ejecuta a la transición $T_1$ y luego $T_2$ dos veces, la red se bloquea. Como se puede observar en la figura \ref{fig:rdp3.2grafo} al ejecutar $T_2$ desde el estado $S_1$ conduce al deadlock. Al analizarla mediante el algoritmo y conociendo que el sifón que se presenta en la red está compuesto por $\{P_1,P_2\}$ como se ve en la figura \ref{fig:rdp3.2sifon} con un marcado inicial de 2, se obtiene un controlador que permite la ejecución de $T_1$ una única vez, es decir, una vez ejecutada ésta la parte izquierda de la red no se podrá volver a ejecutar.

\begin{figure}[H]
	\centering
	\includegraphics[scale=0.55]{Figures/algoritmo2/2.png}
	\caption{Control sobre la RdP.}
	\label{fig:rdp3.3}
  \end{figure}

\bigskip
Posteriormente se decidió aplicar la misma hipótesis sobre una red más compleja, con un mayor número de estados en su grafo de alcanzabilidad. Para esto se eligió la red representada en la figura \ref{fig:panama}

\begin{figure}[H]
	\centering
	\includegraphics[scale=0.55]{Figures/algoritmo2/sifones_panama.png}
	\caption{Sifones RdP Panamá.}
	\label{fig:rdp_panama_sifones}
  \end{figure}

Como se puede observar en la figura \ref{fig:rdp_panama_sifones} estos son los sifones presentes en la red. Al obtener los estados con deadlock \ref{fig:state_deadlock1} y \ref{fig:state_deadlock2} se indagó para verificar cuáles eran los sifones vacíos en los mismos.
Para el estado número 82 el sifón que se vacía es el compuesto por las plazas \{$P_4, P_{12}, P_{15}, P_{17}, P_{19}, P_{20}$\} (\ref{fig:rdp_panama_sifon1}), mientras que para el estado número 484 el sifón vacío es el compuesto por las plazas \{$P_6, P_{10}, P_{16}, P_{18}, P_{20}, P_{21}$\} (\ref{fig:rdp_panama_sifon2}).

\begin{figure}[H]
	\centering
	\subfloat[Sifón izquierdo]{
	\includegraphics[width=\textwidth]{Figures/algoritmo2/panamaizq.png}
	\label{fig:rdp_panama_sifon1}
	}\\
    \subfloat[Sifón derecho]{
	\includegraphics[width=\textwidth]{Figures/algoritmo2/panamader.png}
	\label{fig:rdp_panama_sifon2}
	}
	\caption{Sifones vacíos RdP Panamá}
\end{figure}

En las figuras \ref{fig:rdp_panama_sifon1} y \ref{fig:rdp_panama_sifon2} se pueden observar, destacados en color rojo, los sifones que se vacían en el estado de deadlock. De los mismos se obtiene el marcado inicial para luego colocar una plaza de control correspondiente a cada uno.

\begin{figure}[H]
	\centering
	\includegraphics[width=\textwidth]{Figures/algoritmo2/panama_control_deadlock_true.png}
	\caption{Control RdP Panamá, deadlock true.}
	\label{fig:rdp_panama_deadlock_true}
  \end{figure}

\subsection{Conclusiones}
Nuevamente se alcanzó el objetivo propuesto, pero a diferencia de la versión anterior, cuando la red presenta un camino de deadlock este se ejecuta un número limitado de veces. \\ 
Y en coincidencia con la versión previa, sucede que la red presenta inanición y problemas de convergencia de la solución para redes grandes, como puede observarse en la figura \ref{fig:rdp_panama_deadlock_true}. \\
%----------------------------------------------------------------------------------------
%	SECTION 5
%----------------------------------------------------------------------------------------
\section{Iteración 3: Algoritmo v3.0}
\subsection{Introducción}

Luego de una profunda investigación y adentrándonos en el trabajo realizado por Ezpeleta et al. \cite{paperezpeleta}, el cuál expone una metodología en donde su principal idea es caracterizar situaciones de deadlock en términos de una marca igual a cero para ciertos sifones presentes en la red, pero no cualquiera de los sifones, sino los \textbf{sifones mínimos} mencionados en la sección \ref{sec:tysalg2}.\\
Para evitar que el sistema presente deadlock se propuso una política para la asignación de recursos basada en la adición de nuevas plazas de control a la red que impiden la presencia de sifones mínimos sin marcar. Estas nuevas plazas denominadas \textbf{supervisores}, están definidas por un marcado inicial y tres tipos de arcos; para obtener los primeros dos se tuvo en cuenta las fuentes investigadas, basándonos en la idea de un estado \textit{idle} y de un conjunto complemento del sifón en cuestión. Mientras que para lograr el tercer arco se descubrió una relación entre el sifón a atacar y los T-invariantes encontrados en la red. \\
\par
La adición de las plazas de control aseguran que el marcado de los sifones mínimos de la red sea al menos mayor o igual a uno para cada estado alcanzable, que es la condición necesaria para la prevención del deadlock. 
Esto se expresa matemáticamente de la siguiente manera:

\begin{equation}
    \sum m(p_i) \geq 1,\ donde \ p_i \in \ S  
\end{equation}

\noindent donde, la sumatoria de las marcas de todas las plazas que componen al sifón debe ser mayor o igual a uno; siendo $p_i$ las plazas que pertenecen al sifón S.

\subsection{Objetivos}
Al igual que en las iteraciones anteriores el objetivo es evitar alcanzar estados de deadlock.

\subsection{Desarrollo}
Se parte de la hipótesis de obtener supervisores que impidan el vaciado de los \textbf{bad siphons mínimos}\footnote{Por simplicidad a partir de esta sección y a lo largo de todo el trabajo, la mención de \textit{bad siphon} hará referencia a los \textit{bad siphon mínimos} definidos.}, estos son aquellos sifones mínimos que desencadenan el estado deadlock al vaciarse, limitando el comportamiento de la red de Petri evitando que se alcancen dichos estados; y de esta manera preservar la vivacidad de una red determinada. \\ 
Ezpeleta propone una política para la asignación de recursos basada en la adición de supervisor(es). Esta política de control limita el comportamiento del sistema a un conjunto de estados de manera tal que, independientemente del estado que alcance el mismo, estamos seguros de que este no será una situación indeseable de deadlock.
Tomando como base esto, se detallan en los siguientes puntos el progreso del algoritmo.


\subsubsection{Definición del supervisor} \label{sec:definicionsup}
Para evitar alcanzar estados de deadlock, el sistema debe ser controlado por una nueva plaza denominada supervisor (\textbf{Vs}), que está conectado con el proceso (\textbf{G}) en un bucle cerrado (figura \ref{fig:fig3.4}). El mismo genera una secuencia de eventos discretos (\textbf{s}) y lo envía al supervisor, el cual desencadena un conjunto de eventos permitidos ($\gamma$) que suceden en el proceso \textbf{G} en el siguiente disparo. El conjunto  $\gamma$ depende de la secuencia \textbf{s} y no consta de los eventos que pueden llevar al proceso a un estado de deadlock en el siguiente paso. 

\begin{figure}[H]
	\centering 
	\includegraphics[scale=0.45]{Figures/algoritmo3/desarrollo/retroalimentacion.png} 
	\caption[Retroalimentación entre el  proceso y el supervisor]{Retroalimentación entre el  proceso y el supervisor \footnotemark.}
	\label{fig:fig3.4}
  \end{figure} \footnotetext{Figura adaptada del paper publicado por \textit{D. Kezić} et al. \cite{paperkezic}.}
  
Para alcanzar el control antes mencionado se inicia analizando el grafo de alcanzabilidad en busca de estados con deadlock. En caso de encontrarlos, se continúa con la búsqueda de bad siphons en cada uno de estos.  \\
Ya con todos los estados en deadlock y sus sifones vacíos, se realiza un filtrado de los mismos descartando aquellos que se encuentren vacíos desde el marcado inicial de la red; estos no son de interés para nuestro análisis dado que una vez vacíos (por definición) estos permanecerán en ese estado sin modificar su comportamiento. \\
Una vez realizado el filtrado, se seleccionó uno de los sifones perteneciente a uno de los estados deadlock, y es a este, al que se buscó controlar.

El supervisor que va a controlarlo está compuesto por un conjunto de arcos y por una plaza con un marcado proporcionado por el marcado en el estado inicial (\textit{idle}) del sifón menos uno.
Esto se expresa matemáticamente de la siguiente manera: 

\begin{equation}
    m_0(V_s) = m_0(S_i)-1 
\end{equation}

\noindent El conjunto de arcos vincula la plaza mencionada con tres tipos de transiciones:
\begin{enumerate}
    \item \textbf{Transiciones sensibilizadas en estado idle}: la ejecución de estas transiciones son las que extraen tokens del supervisor dado que el disparo de las mismas inicia los diferentes procesos que componen a la red, pudiendo tomar un camino que conduce al sifón (por lo que el token consumido será devuelto luego de una secuencia de disparos) ó tomar otro camino que no contemple al sifón y en tal caso las transiciones del ítem 3 serán las encargadas de devolver el token consumido al supervisor. 
    
    \item Para el segundo conjunto de transiciones es necesario definir un nuevo conjunto de plazas denominadas \textbf{complemento del sifón}, estas son aquellas plazas que no forman parte del sifón pero para evolucionar en su marcado requieren del disparo de transiciones que se habilitan mediante el marcado de las plazas recurso que componen al sifón, es decir , hacen uso de estas. Es por esto que las transiciones de salidas al conjunto complemento del sifón son aquellas que le agregan tokens al supervisor.
    
    \item Para definir este último conjunto de transiciones se realizó una investigación paralela dado que Ezpeleta no contemplaba este caso. En donde el último arco a tener en cuenta se obtuvo a partir de la relación entre el bad siphon a controlar y los T-invariantes presentes en la red, dependiendo del camino que tome la secuencia de disparos es necesario que este arco devuelva el token al supervisor. \\
    Para esto es necesario verificar la presencia de un conflicto o bifurcación en la red entre T-invariantes, y en caso de existir, enfocarse en las transiciones involucradas en el mismo tal que al dispararse cualquiera de estás no alcancen al sifón, posterior a una secuencia de disparos, ya que el token que se le quitó al supervisor con el disparo de las transiciones idle, mencionadas en ítem 1, nunca volverá a este. \\
    Para corregir esto, con el disparo de la transición conflictiva en cuestión, es necesario implementar el arco que devuelva el token consumido al supervisor. \\
	En caso de dispararse cualquiera de las transiciones del conflicto que posteriormente toman el camino hacía el sifón, no es necesario este arco ya que implica que la extracción por parte de las transiciones idle al supervisor va ser utilizada por la subred formada por el sifón a controlar y los arcos mencionados en el punto 2, los cuales devolverán el token al supervisor.
\end{enumerate}

Tomando la red definida por Ezpeleta se pueden observar los tres tipos de transiciones que conforman el supervisor. El sifón a controlar, en este caso, es el compuesto por las plazas \{$P_7,P_8,P_9,P_{10}$\}.

 \begin{figure}[H]
	\centering
	\includegraphics[width=\textwidth]{Figures/algoritmo3/desarrollo/ezpeleta1.png}
	\caption[RdP Ezpeleta.]{RdP Ezpeleta \footnotemark .}
	\label{fig:fig3.5}
  \end{figure} \footnotetext{Figura adaptada del paper publicado por \textit{Ezpeleta} et al. \cite{paperezpeleta} .}

En la figura \ref{fig:fig3.5} se observan las transiciones sensibilizadas en el estado inicial $\{T_1,T_7\}$ , definiendo de esta manera los primeros arcos que componen al supervisor.  
  
 \begin{figure}[H]
	\centering
	\includegraphics[width=\textwidth]{Figures/algoritmo3/desarrollo/ezpeleta2.png}
	\caption{RdP Ezpeleta con sifón a controlar y sus plazas complemento.}
	\label{fig:fig3.6}
  \end{figure}

Observando la figura \ref{fig:fig3.6} se distingue en verde las plazas que conforman el sifón a controlar y en rojo las plazas complemento del mismo; mientras que en azul las transiciones $\{T_4,T_9\}$ que conectarán con el supervisor mediante el segundo conjunto de arcos. \\


Para definir el tercer tipo de transiciones es necesario tener en cuenta los \break T-invariantes de la red dado que estos nos permiten observar los diferentes caminos que puede tomar la red en la evolución de los disparos de las transiciones de la misma.

 \begin{figure}[H]
	\centering
	\includegraphics[scale=0.5]{Figures/algoritmo3/desarrollo/ezpeleta3.png}
	\caption{RdP Ezpeleta y sus T-invariantes.}
	\label{fig:fig3.7}
  \end{figure}

Como ilustra la figura \ref{fig:fig3.7}, existen tres invariantes que componen la red (en distintos colores) y la plaza $P_2$ involucrada en un conflicto, dado que su marcado habilita dos posibles caminos de T-invariantes, pudiendo tomar uno u otro. 
En caso de que se proceda con el disparo de la transición $T_2$, produciendo de esta manera la ejecución de las transiciones que componen al T-invariante rojo, el cual no afecta el marcado del sifón en cuestión y por lo que su ejecución devolverá el token al supervisor (se conectara con el supervisor mediante el tercer conjunto de arcos). \\


\noindent Según la definición de la sección \ref{sec:definicionsup}, el supervisor queda conformado de la siguiente manera:
\begin{itemize}
    \item El marcado = $ M(S_i) - 1 = \{M(P_7) + M(P_8) + M(P_9) + M(P_{10}) \} -1 = 2$ 
    \item Transiciones output = $\{T_1, T_7\}$
    \item Transiciones input = $\{T_2, T_4, T_9\}$
\end{itemize}

\begin{figure}[H]
    \centering
    \includegraphics[scale=0.7]{Figures/algoritmo3/desarrollo/ezpeleta4.png}
    \caption{Colocación de un supervisor en RdP Ezpeleta.}
    \label{fig:fig3.8}
 \end{figure}

Por último, en la figura \ref{fig:fig3.8}, se observa la red al controlar uno de los bad siphon, este control se logra al colocar el supervisor ($P_{16}$) con su marcado y arcos respectivos. \\


\noindent El análisis completo de la red se llevará a cabo en las iteraciones posteriores dado que la red aún presenta deadlock.


\subsection{Implementación del algoritmo}
En esta sección se desarrollan los 5 pasos que modelan el algoritmo en la determinación del supervisor. \\
Una de las consideraciones a tener en cuenta, la cuál emerge del estudio de las distintas redes simétricas es:
\begin{itemize}
    \item En caso de que la red a controlar fuese simétrica, aislar en subredes y ejecutar el algoritmo en una sola de ellas obteniendo el control de la misma; el cual se podrá extender a las restantes partes. Teniendo en cuenta, al momento de integrarlas en la red completa, que los arcos pertenecientes al supervisor de cada una de las partes deben aplicarse de igual manera a los supervisores de las otras subredes (como puede observarse en el caso de la red Portugal (sección \ref{sub:portugal})). %Vincular cuando este terminado con la sección.
\end{itemize}

\newpage
\subsubsection{Desarrollo} \label{sec:desarrollov3}
\begin{enumerate}
    \item Obtener los sifones vacíos en el estado inicial; estos sifones deben ignorarse dado que una vez vacíos permanecerán así por el resto de los estados alcanzables.
    \item Obtener los estados en deadlock con sus respectivos sifones vacíos. 
    \item Seleccionando uno de los sifones mencionados en el ítem anterior:
        \begin{enumerate}
            \item Se obtiene su marcado inicial para posteriormente definir el marcado de su correspondiente supervisor.
            \item Se localizan las transiciones sensibilizadas en el estado idle (para el marcado inicial). 
            \item Se obtienen las plazas complemento del mismo.
                \begin{enumerate}[i. ]
                    \item Se buscan las transiciones que quitan y agregan tokens a estas plazas. 
                    \item Las transiciones que agregan más tokens de los que quitan al sifón son las que nos interesan.
                \end{enumerate}
            \item Se verifica si hay transiciones en conflicto, de ser así se utilizan los \break T-invariantes para verificar si la ejecución de la misma se encuentra en el camino de las plazas del sifón. De no ser así, estas transiciones serán también de interés.
        \end{enumerate}
    Las transiciones destacadas en los ítems anteriores van a ser las que van a incorporar y extraer tokens del supervisor, como se mencionaron en la iteración 2.

    \item Agregar una nueva plaza de control (perteneciente al supervisor) a la red puede producir un nuevo sifón mínimo no controlado y un nuevo estado de bloqueo. Por lo tanto, debemos volver al punto 1 calculando nuevamente el árbol de alcanzabilidad y repetir todo el algoritmo, atacando la totalidad de los bad siphon hasta alcanzar una red viva. \\
    El algoritmo finaliza cuando no es posible encontrar un nuevo punto muerto en la red de Petri, es decir se resuelve el deadlock de la misma. \\ Sin embargo, puede darse la situación en donde el algoritmo no converge a una solución dado que sugiere supervisores con marcado igual a 0 o supervisores ya colocados.

    \item En caso, de que los pasos anteriores no convergen a una red libre de deadlock se realiza un análisis de división de la misma y se ataca cada subred resultante por separado, es decir, se debe ejecutar el algoritmo desde el paso 1 al 4 para cada subred, para luego reunir las soluciones en la red de petri original. \\
    La división se realiza teniendo en cuenta los T-invariantes y su relación con los bad siphon.

    Para esto se deben tener en cuenta algunos factores:        
    \begin{enumerate}
            \item En caso de que la división de la red resulte en una de las subredes que contempla el conflicto en su totalidad (caso red POPN (sección \ref{sub:POPN})), las soluciones de ambas subredes se pueden unir conservando la vivacidad de la red sin problema. 
            
            \item En caso de ser necesaria la división en subredes y la misma no pueda contemplar el conflicto en su totalidad (caso red Hospital (sección \ref{sub:Hospital})), notar que las transiciones pertenecientes al conflicto deben devolver el token, al momento de unir las subredes, al supervisor que no es propio de su subred; debiendo colocar de esta manera un nuevo arco en la red original. 
        \end{enumerate} %Agregar la sección luego
    En caso de tener supervisores en común entre las subredes, es decir, tienen el mismo conjunto de arcos entrantes y salientes, al momento de definir el marcado del mismo en la red original, el supervisor debe tomar el valor del marcado del menor de ellos dado que no tiene que permitir el vaciado del sifón con menos marcas.
\end{enumerate}


\noindent Las fórmulas para calcular el supervisor para un sifón son las siguientes:
\begin{enumerate}
    \item $m(V_s) = m(BS_i) - 1$
    \item $Arco_1 = \{(V_s, t) \ / \ t \in P_0 \bullet \}$
    \item $Arco_2 = \{(t, V_s) \ / \ t \in C_s \bullet \}$
    \item $Arco_3 = \{(t, V_s) \ / \ t \in conflicto \wedge t \notin T_{inv_{BS_i}} \}$
\end{enumerate}

\noindent siendo:
\begin{enumerate}
    \item $V_s$ = plaza supervisor
    \item $P_0$ = plazas marcadas idle
    \item $C_s$ = complemento sifón
    \item $BS_i$ = bad siphon
    \item $t$ = conjunto de transiciones
\end{enumerate}

En la figura \ref{fig:fig3.9} se puede observar, a modo de ejemplo de aplicación, la salida del algoritmo por consola para la ejecución de la red de Petri Panamá, analizada en profundidad en sección 3.4.3.3.

\begin{figure}[H]
	\centering
	\includegraphics[scale=0.5]{Figures/algoritmo3/desarrollo/consola.png}
	\caption{Ejecución del algoritmo v3.0.}
	\label{fig:fig3.9}
 \end{figure}

\begin{enumerate}[i.]
    \item Las primeras 4 líneas son para la conversión de los datos exportados mediante el software Petrinator:
        \begin{itemize}
            \item \textit{pa\_panama.html} contiene el grafo de alcanzabilidad.
            \item \textit{ma\_panama.html} contiene la matriz pre y post.
            \item \textit{si\_panama.html} contiene los sifones y las trampas.
            \item \textit{in\_panama.html} contiene los invariantes de plaza y de transición.
        \end{itemize}
    
    \item En este caso no fue necesario descartar ningún sifón en el análisis dado que en el estado inicial todos los sifones se encuentran marcados. 
    
    \item Se obtienen los estados en deadlock con sus respectivos sifones vacíos y el marcado de los mismos.
    
    \item Como se puede observar el algoritmo brinda todos los posibles supervisores para el control de los sifones vacíos en deadlock. 
\end{enumerate}

Se toma uno de los supervisores, se incorpora a la red en el software Petrinator realizando el análisis correspondiente en búsqueda de verificar que el deadlock de la red haya desaparecido; de no ser así se exportan nuevamente los archivos y se realiza la ejecución del algoritmo nuevamente. Y así iterativamente hasta lograr que el deadlock de la red desaparezca.

\subsubsection{Pseudocódigo}
\noindent Definiendo:
\begin{itemize}
    \item $C_S$: complemento del sifón
    \item t: conjunto de transiciones
    \item $V_S$: plaza supervisor
    \item Cantidad de sifones = cantidades de sifones total del sistema
    \item SD: state deadlock
    \item S: conjunto de sifones
    \item BS: conjunto de bad siphons
\end{itemize}

\begin{algorithm} [H]
  \floatname{algorithm}{Pseudocódigo}
  \caption{búsqueda de bad siphon a controlar (v3)}
  \label{alg:algoritmo1} 
  \begin{algorithmic}[1]
 
    % ENTRADA / SALIDA
    \Require{RdP ($N,M_0$) de tipo S³PR.}
    \Ensure{bad siphon.}
 
    \State Generar el grafo de alcanzabilidad G($N,M_0$) de la RdP.
    \State Obtener matrices I+, I-, invariantes, trampas y sifones.
 
    \For{i: 0 \textbf{to} cantidad de estados}
        \If{estado = estado en deadlock}
            \State $SD_j$ $\leftarrow$ \ $estado_i$
        \EndIf
    \EndFor
    
    \If{estado =  idle}
        \For{i: 0 \textbf{to} cantidad de sifones}
            \If{M($S_i$) = 0}
                \State $BS_{idle}$ $\leftarrow$ \  $S_i$
            \EndIf
        \EndFor
    \EndIf
    
    \State en SD[0]
    \For{i: 0  \textbf{to}  cantidad de sifones}
        \If{$M(S_i)$ = 0}
            \State $BS_{SD}$ $\leftarrow$ \ $S_i$  
        \EndIf
    \EndFor
    \State Se eliminan de $BS_{SD}$ aquellos que estén en $BS_{idle}$ 
    \State $BS_{SD}$[0] \  $\rightarrow$\ control de bad siphon usando \textbf{Algoritmo \ref{alg:algoritmo2}}
  \end{algorithmic}
\end{algorithm}
\bigskip

\begin{algorithm}[H] 
  \floatname{algorithm}{Pseudocódigo}
  \caption{Búsqueda de supervisor que controle bad siphon (v3)}
  \label{alg:algoritmo2} 
  \begin{algorithmic}[1]
 
    % ENTRADA / SALIDA
    \Require{RdP ($N,M_0$) de tipo S³PR.}
    \Ensure{supervisor.}
 
    \State \textbf{Agregar} plaza de control $VS_i$ con $M(VS_i)= M(BS_{SD}) - 1$
    
    \If{estado = idle}
            \State \textbf{Agregar} arco $(VS_i , t) \ \forall \  t \in \ p\bullet$
    \EndIf
 
    \State \textbf{Agregar} arco $(t, VS_i) \forall \ t \in \ C_S\bullet$
    \State \textbf{Agregar} arco $(t, VS_i)  \forall \ t \in \ conflicto \ \wedge \ t \notin T-invariante_{BS}$
  \end{algorithmic}
\end{algorithm}
\bigskip

El algoritmo iterativo para lograr una red de Petri sin deadlock se muestra en la figura \ref{fig:fig3.10}

\begin{figure}[H]
	\centering
	\includegraphics[scale=0.5]{Figures/algoritmo3/desarrollo/diagrama_flujo.png}
	\caption{Ejecución del algoritmo v3.0.}
	\label{fig:fig3.10}
 \end{figure}

\subsection{Criterio de elección del supervisor} \label{sec:criterio}
Al ejecutar el algoritmo, se obtiene una lista de supervisores a colocar dependientes al bad siphon que se va a controlar. El criterio de elección de qué supervisor agregar se realiza teniendo en cuenta: 
\begin{itemize}
    \item Afecte a un sifón mínimo.
    \item Cantidad de veces que aparece el supervisor en la lista.
    \item En caso de haber más de un supervisor sugerido para el control de un bad siphon, se elige aquel que presente la menor cantidad de \textit{inputs}.
\end{itemize}


\subsection{Ejecución en diferentes escenarios}
En esta sección se busca explicar cómo fue progresando el algoritmo a medida que se implementó en diferentes casos de redes de Petri, dado que en cada nueva red se encontraban situaciones diferentes que debían tenerse en cuenta y cada una implicó una extensión más para el algoritmo final. \\
Estas extensiones se deben a que en cada una de las redes analizadas la relación que presentaban sus T-invariantes con el bad siphon a controlar era diferente, esto fue lo que permitió generalizar el algoritmo de manera de contrarrestar el deadlock en cada una de las variantes.

\subsubsection{Caso Panamá}
Esta red modela un sistema de tráfico marino. El problema es la posibilidad de interbloqueo entre los barcos que circulan a través de un sistema de canales y dársenas (tres canales y cuatro dársenas). Los barcos circulan hacia la derecha o izquierda, esperando a cada lado del sistema de canales su derecho a paso. 
La capacidad de cada canal y dársena es de un barco a la vez, por lo que un barco no puede ingresar a un canal o dársena ocupado, estando obligado a esperar a que se desocupe.

\begin{figure}[H]
    \centering
    \includegraphics[width=\textwidth]{Figures/algoritmo3/panamacanal.png}
    \caption[Modelo del Canal de Panamá.]{Modelo del Canal de Panamá \footnotemark.}
    \label{fig:fig3.11}
 \end{figure} \footnotetext{Figura adaptada del libro \textit{An  Algorithm  for  Deadlock Prevention  Based  on  Iterative  Siphon  Control  of  Petri  Net} \cite{paperpanama} .}

\paragraph{Características generales}
\begin{itemize}
    \item Las cuatro dársenas están representadas por las plazas \{$P_{15}, P_{16}, P_{17}, P_{18}$\}.
    \item Los tres canales se denotan por las plazas \{$P_{19}, P_{20}, P_{21}$\}.
    \item Mientras que los barcos por las marcas de las plazas \{$P_1, P_7$\}.
\end{itemize}

\newpage
\paragraph{Análisis estructural}
\hfill
\begin{figure}[H]
	\centering
    \includegraphics[scale = 0.41]{Figures/algoritmo3/Panama1.png}
    \caption[RdP Canal de Panamá y sus T-invariantes.]{RdP Canal de Panamá \footnotemark \ y sus T-invariantes.}
	\label{fig:panamaytinvariantes}
 \end{figure} \footnotetext{Figura adaptada del libro \textit{An  Algorithm  for  Deadlock Prevention  Based  on  Iterative  Siphon  Control  of  Petri  Net} \cite{paperpanama} .}
 
\hfill \par En la figura \ref{fig:panamaytinvariantes}, la red no presenta un conflicto entre los T-invariantes, representados en rojo y verde. Se ejecuta el algoritmo desde el punto 1 al 4 permitiendo encontrar los supervisores sin necesidad de subdividir la red; incluso realizar esta acción en esta red no tendría mucho sentido dado que los T-invariantes por separado no representan el comportamiento de la red en su totalidad.

\paragraph{Red controlada}
\hfill  
\par Una vez ejecutado el algoritmo, se obtienen los supervisores a agregar para controlar la red solucionando el deadlock, estos están representado por las plazas $P_{22}$ y $P_{23}$ en la figura \ref{fig:panamaytinvariantes}.

\begin{table}[H]
    \centering
    \begin{tabular}{|c|c|P{2.2cm}|P{2.2cm}|c|}
    \hline
    \textbf{Supervisor} & \textbf{Marcado} & \textbf{Transiciones input} & \textbf{Transiciones output} & \textbf{Bad Siphon Controlado}  \\  \hline
    $P_{22}$ & 3 & \{$T_{5}, T_{11}$\} & \{$T_{1}, T_{13}$\} & \{$P_6,P_{10},P_{16},P_{18},P_{20},P_{21}$\} \\ 
    \hline
    $P_{23}$ & 3 & \{$T_{3}, T_{9}$\} & \{$T_{1}, T_{13}$\} & \{$P_4,P_{12},P_{15},P_{17},P_{19},P_{20}$\} \\ 
    \hline
    \end{tabular}
    \caption{Supervisores: RdP Panamá.}
    \label{tab:panama}
    
    \end{table}

\begin{figure}[H]
	\centering
	\includegraphics[scale=0.4]{Figures/algoritmo3/Panama2.png}
	\caption{RdP Canal de Panamá controlada.}
	\label{fig:panamacontrolada}
 \end{figure}
 
 \subsubsection{Caso Ezpeleta}
Esta red modela la ejecución concurrente de procesos de trabajo en FMS, representando un sistema donde se ejecutan dos tipos de procesos de trabajo. \\
En la red existen plazas que simulan la disponibilidad de recursos (5) y un control incorrecto de estos en la ejecución de los procesos de trabajo, puede conducir a situaciones de deadlock.

\paragraph{Características generales}
\begin{itemize}
    \item Presenta conflicto entre T-invariantes.
    \item Los recursos de la red están representados por las plazas \{$P_3,P_4,P_7,P_{10},P_{14}$\}.
\end{itemize}

\paragraph{Análisis estructural}
\hfill
\begin{figure}[H]
	\centering
	\includegraphics[scale=0.5]{Figures/algoritmo3/ezpeleta1.png}
	\caption[RdP Ezpeleta y sus T-invariantes.]{RdP Ezpeleta \footnotemark \ y sus T-invariantes.}
	\label{fig:ezpeletatinvariante}
 \end{figure} \footnotetext{Figura adaptada del paper publicado por \textit{Ezpeleta} et al. \cite{paperezpeleta}.}

En la figura \ref{fig:ezpeletatinvariante}, la plaza $P_2$ forma parte de un conflicto permitiendo la ejecución de un subcircuito de la red u otro, pudiendo seguir dos T-invariantes diferentes (rojo o azul en este caso). \\
Es por esto que fue necesario ejecutar el algoritmo de forma completa, es decir, desde el punto 1 al 5, contemplando la división de la red dado que los primeros 4 pasos no lograron alcanzar una red libre de deadlock.\\
En la división resultaron dos subredes, ambas preservando el conflicto (plaza y transiciones que lo conforman).

\subparagraph{Subred izquierda}
\hfill \break
Una de las subredes (figura \ref{fig:ezpeletasubizquierda}) no fue necesario controlarla dado que no presentaba deadlock.

\begin{figure}[H]
	\centering
	\includegraphics[scale=0.55]{Figures/algoritmo3/ezpeleta2.png}
	\caption{Subred izquierda Ezpeleta.}
	\label{fig:ezpeletasubizquierda}
 \end{figure}

\subparagraph{Subred derecha}
\hfill \break
Mientras que la segunda subred (figura \ref{fig:ezpeletasubderecha}) si fue necesario encontrar los supervisores que resuelvan el deadlock. \\
\bigskip

\begin{figure}[H]
	\centering
	\includegraphics[scale=0.5]{Figures/algoritmo3/ezpeleta3.png}
	\caption{Subred derecha Ezpeleta.}
	\label{fig:ezpeletasubderecha}
 \end{figure}
\bigskip

\subparagraph{Control subred derecha}
\hfill \break
Una vez ejecutados los 4 primeros pasos del algoritmo, sobre la subred en cuestión, se obtuvieron los supervisores a agregar para controlar la misma, resolviendo el problema de punto muerto. \\
\bigskip

\begin{table}[H]
    \centering
    \begin{tabular}{|c|c|P{2.2cm}|P{2.2cm}|c|}
   \hline
    \textbf{Supervisor} & \textbf{Marcado} & \textbf{Transiciones input} & \textbf{Transiciones output} & \textbf{Bad Siphon Controlado}  \\  \hline
    $P_{15}$ & 4 & \{$T_{2}, T_{5}, T_{9}$\} & \{$T_{1}, T_{7}$\} & \{$P_5,P_{6},P_{8},P_{11},P_{12}$\} \\ 
    \hline
    $P_{14}$ & 2 & \{$T_{2}, T_{5}, T_{8}$\} & \{$T_{1}, T_{7}$\} & \{$P_8,P_{9},P_{11},P_{12}$\} \\ 
    \hline
    $P_{13}$ & 2 & \{$T_{2}, T_{4}, T_{9}$\} & \{$T_{1}, T_{7}$\} & 
    \{$P_5,P_{6},P_{7},P_{8}$\} \\ 
    \hline
    \end{tabular}
    \label{tab:ezpeletaderecha}
    \caption{Supervisores: RdP Ezpeleta (R).}
\end{table}

\begin{figure}[H]
	\centering
	\includegraphics[scale=0.55]{Figures/algoritmo3/ezpeleta4.png}
	\caption{Subred derecha Ezpeleta controlada.}
	\label{fig:ezpeletasubderechacontrolada}
 \end{figure}

\subparagraph{Control red original}
Al contemplar el conflicto en el análisis de cada subred por separado, la unión de estas soluciones permiten resolver el deadlock de la red completa, sin necesidad de realizar otro análisis.\\

\begin{figure}[H]
	\centering
	\includegraphics[scale=0.55]{Figures/algoritmo3/ezpeleta5.png}
	\caption{RdP Ezpeleta controlada.}
	\label{fig:ezpeletacontrolada}
 \end{figure}

\subsubsection{Caso POPN} \label{sub:POPN}
Esta red modela un sistema de manufacturación robotizado que consiste principalmente de: tres robots {R1, R2, R3} y cuatro máquinas {M1, M2, M3, M4}.\\
En este sistema se procesan tres tipos diferentes de piezas A, B y C; estas provienen de tres contenedores de entrada distintos I1, I2 e I3, de los cuales los robots las retiran, las colocan en las máquinas para su procesamiento y posteriormente depositan en tres contenedores de salidas distintos, que son: O1, O2 y O3. \\
Todo esto se puede visualizar en figura \ref{fig:popnrobot}.

\begin{figure}[H]
	\centering
	\includegraphics[scale=0.55]{Figures/algoritmo3/POPNrobot.png}
	\caption[Modelado de partes del sistema POPN.]{Modelado de partes del sistema POPN\footnotemark .}
	\label{fig:popnrobot}
 \end{figure} \footnotetext{Figura adaptada del libro \textit{System Modeling and Control with Resource-OrientedPetri Nets} \cite{libropopn}.}

Los robots tienen tareas definidas, cada operación de traslado de las piezas le corresponde a un único robot. \\
En la figura \ref{fig:popnrobotlinea}, se observan las tres trayectorias diferentes para cada tipo de pieza.

\begin{figure}[H]
	\centering
	\includegraphics[scale=0.55]{Figures/algoritmo3/POPNrobotlinea.png}
	\caption{Trayectorias de la producción de las piezas.}
	\label{fig:popnrobotlinea}
 \end{figure} 

Cada una de las máquinas, M1, M2, M3 y M4, tiene un color diferente. Y R1, R2 y R3 representan a los tres robots del sistema, encargados de trasladar las piezas de un lado a otro, hasta llegar a su objetivo. Si los recursos no se distribuyen de la forma correcta entre las máquinas la red se bloquea.

\paragraph{Características generales}
    \begin{itemize}
        \item Presenta conflicto entre T-invariantes.
        \item Las máquinas M1-M4 estan representadas por $\{P_8, P_9, P_{21}, P_{22}\}$
        \item Mientras que los recursos R1-R3 están denotados por $\{P_7, P_{14}, P_{10}\}$
    \end{itemize}
   
\paragraph{Análisis estructural}
\hfill \break

\begin{figure}[H]
	\centering
	\includegraphics[width=\textwidth]{Figures/algoritmo3/POP1.png}
	\caption[RdP POPN y sus T-invariantes.]{RdP POPN \footnotemark \ y sus T-invariantes.}
	\label{fig:popntinvariantes}
 \end{figure} \footnotetext{Figura adaptada del libro \textit{System Modeling and Control with Resource-OrientedPetri Nets} \cite{libropopn} .}

En la figura \ref{fig:popntinvariantes}, la plaza $P_{16}$ forma parte de un conflicto, permitiendo la ejecución de un subcircuito de la red u otro, pudiendo seguir dos T-invariantes diferentes (rojo o verde en este caso).
Es por esto que fue necesario ejecutar el algoritmo de forma completa, es decir desde el punto 1 al 5, contemplando la división de la red dado que los primeros 4 pasos no lograron alcanzar una red libre de deadlock.\\
De la división resultaron dos subredes, preservando el conflicto (plaza y transiciones que lo conforman) en cada una de ellas.

\subparagraph{Subred derecha}
\hfill
\begin{figure}[H]
	\centering
	\includegraphics[scale=0.48]{Figures/algoritmo3/POP2.png}
	\caption{Subred derecha POPN.}
	\label{fig:popnredder}
 \end{figure}

\subparagraph{Subred izquierda}
\hfill
\begin{figure}[H]
	\centering
	\includegraphics[scale=0.5]{Figures/algoritmo3/POP3.png}
	\caption{Subred izquierda POPN.}
	\label{fig:popnredizq}
 \end{figure}
\bigskip

Dado que ambas subredes presentan deadlock es necesario ejecutar el algoritmo en cada una de ellas (desde el paso 1 al 4). 
\bigskip

\newpage
\subparagraph{Control de subredes}
\hfill \break
En las siguientes figuras \ref{fig:popnreddercontrolada} y \ref{fig:popnredizqcontrolada} se puede observar el control logrado en cada una de las subredes:\\ 
\bigskip

\begin{table}[H]
    \centering
    \begin{tabular}{|c|c|P{2.2cm}|P{2.2cm}|c|}
    \hline
    \textbf{Supervisor} & \textbf{Marcado} & \textbf{Transiciones input} & \textbf{Transiciones output} & \textbf{Bad Siphon Controlado}  \\  \hline
    $P_{28}$ & 1 & \{$T_{3}, T_{8}, T_{14}$\} & \{$T_{6}, T_{12}$\} & \{$P_3,P_{8},P_{12},P_{14}$\} \\
    \hline
    $P_{29}$ & 1 & \{$T_{4}, T_{9}, T_{14}$\} & \{$T_{6}, T_{12}$\} & \{$P_4,P_{9},P_{13},P_{14}$\} \\
    \hline
    $P_{30}$ & 1 & \{$T_{5}, T_{10}, T_{14}$\} & \{$T_{6}, T_{12}$\} & 
    \{$P_5,P_{9},P_{10},P_{15}$\} \\ 
    \hline
    \end{tabular}
    \label{tab:popnderecha}
    \caption{Supervisores: RdP POPN (R).}
    \end{table}
\bigskip

\begin{figure}[H]
	\centering
	\includegraphics[scale=0.5]{Figures/algoritmo3/POP4.png}
	\caption{ Subred derecha POPN controlada.}
	\label{fig:popnreddercontrolada}
 \end{figure}
\bigskip

\begin{table}[H]
    \centering
    \begin{tabular}{|c|c|P{2.2cm}|P{2.2cm}|c|}
    \hline
    \textbf{Supervisor} & \textbf{Marcado} & \textbf{Transiciones input} & \textbf{Transiciones output} & \textbf{Bad Siphon Controlado}  \\  \hline
    $P_{27}$ & 1 & \{$T_{7}, T_{15}, T_{19}$\} & \{$T_{12}, T_{17}$\} & \{$P_1,P_{7},P_{9},P_{12}$\} \\ 
    \hline
    \end{tabular}
    \label{tab:popnizq}
    \caption{Supervisores: RdP POPN (L).}
    \end{table}

\begin{figure}[H]
	\centering
	\includegraphics[scale=0.5]{Figures/algoritmo3/POP5.png}
	\caption{Subred izquierda POPN controlada.}
	\label{fig:popnredizqcontrolada}
 \end{figure}
 
 \paragraph{Control red original}
 \hfill \break
Al contemplar el conflicto en el análisis de cada subred, la unión de las soluciones individuales permiten resolver el deadlock de la red completa, sin necesidad de realizar otro análisis.\\
\bigskip

\begin{figure}[H]
	\centering
	\includegraphics[width=\textwidth]{Figures/algoritmo3/POP6.png}
	\caption{RdP POPN controlada.}
	\label{fig:popncontrolada}
 \end{figure}
\bigskip

\subsubsection{Caso Guanjun} \label{sec:guanjun}
Esta red modela la ejecución concurrente de procesos de trabajo en FMS que describe el comportamiento de 2 subredes relacionadas por dos recursos, en caso de llevar una mala gestión de estos la red terminará en deadlock.
\bigskip

\paragraph{Características generales}
\begin{itemize}
    \item Presenta conflicto entre T-invariantes.
    \item Los recursos R1-R5 están representados por las plazas $\{P_{11}, P_{12}, P_{14}, P_{15}, P_{18}\}$, con $P_{14}$ y $P_{15}$ compartidas por las tres subredes.
\end{itemize}

\subparagraph{Análisis estructural}
\hfill
\begin{figure}[H]
	\centering
	\includegraphics[scale=0.5]{Figures/algoritmo3/Guanjun1.png}
	\caption[RdP Guanjun \ y sus T-invariantes.]{RdP Guanjun \footnotemark \ y sus T-invariantes.}
	\label{fig:guanjuntinvariantes}
 \end{figure} \footnotetext{Figura adaptada del paper publicado por \textit{G. Liu, C. Jiang and M. Zhou} \cite{paperguanjun}. }
 
En la figura \ref{fig:guanjuntinvariantes}, la plaza $P_1$ forma parte de un conflicto permitiendo la ejecución de un subcircuito de la red u otro, pudiendo seguir dos T-invariantes diferentes (rojo o verde en este caso). Pero en este caso, a diferencia de los anteriores, al ejecutar el algoritmo mencionado desde el punto 1-4 por primera vez sobre la red completa, permite encontrar los supervisores que resuelven el deadlock sin necesidad de subdividir la red (es decir, ejecutar el paso 5). Dado que el token independientemente del T-invariante que siga siempre vuelve a los supervisores, como se puede observar en la figura \ref{fig:guanjuncontrolada}.\\

\subparagraph{Red controlada}
\hfill \break
Una vez ejecutado el algoritmo, se obtienen la plaza y los arcos a agregar para controlar la red, solucionando el problema de deadlock.\\

\begin{table}[H]
    \centering
    \begin{tabular}{|c|c|P{2.5cm}|P{2.2cm}|c|}
    \hline
    \textbf{Supervisor} & \textbf{Marcado} & \textbf{Transiciones input} & \textbf{Transiciones output} & \textbf{Bad Siphon Controlado}  \\  \hline
    $P_{19}$ & 6 & \{$T_{2}, T_{7}, T_{10}, T_{13}$\} & \{$T_{1}, T_{9}, T_{12}$\} & \{$P_6,P_{8},P_{10},P_{14},P_{15},P_{18}$\} \\ 
    \hline
    $P_{20}$ & 5 & \{$T_{2}, T_{5}, T_{10}, T_{13}$\} & \{$T_{1}, T_{9}, T_{12}$\} & \{$P_4,P_{6},P_{8},P_{10},P_{14},P_{15}$\} \\ 
    \hline
    \end{tabular}
    \label{tab:guanjun}
    \caption{Supervisores: RdP Guanjun.}
    \end{table}
\bigskip

\begin{figure}[H]
	\centering
	\includegraphics[scale=0.5]{Figures/algoritmo3/Guanjun2.png}
	\caption{RdP Guanjun controlada.}
	\label{fig:guanjuncontrolada}
 \end{figure}
\bigskip

\subsubsection{Caso Hospital} \label{sub:Hospital}
En esta red se modela el caso de un hospital. Los elementos del mismo considerados por esta serán la recepción (PW) donde se reciben los pacientes y se realiza el trabajo administrativo, la sala de consulta (CR) donde un médico da el diagnóstico a los pacientes, la sala de cirugía (S) donde se realizan las cirugías y el médico (D) que realizará todos estos procedimientos.\\
Además se tendrá las siguientes consideraciones:

\begin{itemize}
    \item La persona que trabaja en la recepción y el médico podrían atender a una persona por vez.
    \item Tanto en la sala de consulta como en la sala de cirugía, solo un paciente a la vez podría ser tratado, es decir, podemos decir que su capacidad máxima es igual a uno.
    \item Si queremos que el hospital funcione en buenas condiciones, los pacientes y el personal del hospital deben respetar algunos protocolos:
        \begin{itemize}
            \item El hospital tiene dos entradas: una normal y otra de emergencia.
                \begin{itemize}
                    \item Los pacientes que llegan a la entrada normal (IN1) tienen que ir primero a la recepción (PW) para realizar el papeleo. Dependiendo de los problemas que tengan los pacientes, pueden ser enviados a la sala de consulta (CR) o a la sala de cirugía (S). Después de que los pacientes terminan con cualquiera de estos, van a ver al médico (Dr) para obtener el certificado de liberación. Con todas estas cosas hechas, los pacientes pueden salir del hospital (OUT1).
                    \item El hospital tiene otra entrada (IN2): para los casos de emergencia. Los pacientes que llegaron a IN2, son revisados por el médico (D), y luego enviados a la sala de cirugía. Desde el quirófano tienen que llenar los papeles, por lo que primero deben ir a PW y luego pueden irse a casa(OUT2). 
                    \bigskip
                    
                    \begin{figure}[H]
                	\centering
                	\includegraphics[scale=0.6]{Figures/algoritmo3/hospitalmodelado.png}
                	\caption[Modelado de partes del sistema Hospital]{Modelado de partes del sistema Hospital. \footnotemark .}
                	\label{fig:onosdistribuido}
                 \end{figure}
                \end{itemize}
        \end{itemize}
\end{itemize}

\footnotetext{Figura adaptada del paper publicado por \textit{A. Timotei y J. Colom}\cite{paperhospital} .}

\bigskip

\newpage
\paragraph{Características generales}
\hfill
\begin{itemize}
    \item Presenta conflictos entre T-invariantes.
    \item La sala de consulta está representada por la plaza $P_{11}$, la recepción por la $P_8$, la sala de cirugía por la $P_9$ y doctores por la $P_{10}$.
\end{itemize}

\subparagraph{Análisis estructural}
\hfill
\begin{figure}[H]
	\centering
	\includegraphics[scale=0.4]{Figures/algoritmo3/Hospital1.png}
	\caption[RdP Hospital y sus T-invariantes.]{RdP Hospital \footnotemark \ y sus T-invariantes.}
	\label{fig:onosdistribuido}
 \end{figure} \footnotetext{Figura adaptada del paper publicado por \textit{A. Timotei y J. Colom} \cite{paperhospital} .}

En este caso particular de red en el que la plaza $P_1$ forma parte de un conflicto permitiendo la ejecución de un subcircuito de la red u otro, pudiendo seguir dos T-invariantes diferentes (rojo o verde en este caso).
Por esto fue necesario ejecutar el algoritmo de forma completa (5 pasos).\\

En un principio se buscó dividir la red preservando el conflicto en las subredes individualmente (como en los casos previos) pero las subredes resultantes no presentaban deadlock (figura \ref{fig:hospitalsubredderechaconflicto}).

\hfill
\begin{figure}[H]
	\centering
	\includegraphics[scale=0.4]{Figures/algoritmo3/Hospital2.png}
	\caption{Subred derecha Hospital contemplando el conflicto.}
	\label{fig:hospitalsubredderechaconflicto}
 \end{figure}

Por este motivo, se optó por dividir la red contemplando en cada una de las subredes sólo uno de los caminos del conflicto y el otro T-invariante presente en la red; obteniendo dos subredes.

\subparagraph{Subred derecha}
\hfill
\begin{figure}[H]
	\centering
	\includegraphics[scale=0.5]{Figures/algoritmo3/Hospital3.png}
	\caption{RdP Hospital preservando lado derecho del conflicto.}
	\label{fig:ladoderechoconflicto}
 \end{figure}

\subparagraph{Subred izquierda}
\hfill
\begin{figure}[H]
	\centering
		\includegraphics[scale=0.5]{Figures/algoritmo3/Hospital4.png}
	\caption{RdP Hospital preservando lado izquierdo del conflicto.}
	\label{fig:ladoizquierdoconflicto}
 \end{figure}

\newpage
\subparagraph{Control subredes}
\hfill \break
Se ejecutaron los primeros 4 ítems del algoritmo obteniendo los supervisores correspondientes, resolviendo el deadlock en cada subred.
\bigskip

\begin{table}[H]
    \centering
    \begin{tabular}{|c|c|P{2.2cm}|P{2.2cm}|c|}
    \hline
    \textbf{Supervisor} & \textbf{Marcado} & \textbf{Transiciones input} & \textbf{Transiciones output} & \textbf{Bad Siphon Controlado}  \\  \hline
    $P_{14}$ & 3 & \{$T_{3}, T_{7}$\} & \{$T_{1}, T_{5}$\} & \{$P_2, P_{7}, P_{6}, P_{7}, P_{8}$\} \\ 
    \hline
    \end{tabular}
    \caption{Supervisores: RdP Hospital (L).}
    \label{tab:Hospital-SubL}
\end{table}
\bigskip

\begin{figure}[H]
	\centering
		\includegraphics[width=\textwidth]{Figures/algoritmo3/Hospital5.png}
	\caption{Control RdP Hospital preservando lado izquierdo del conflicto.}
	\label{fig:ladoizquierdoconflictocontrol}
 \end{figure}
\bigskip

\begin{table}[H]
    \centering
    \begin{tabular}{|c|c|P{2.2cm}|P{2.2cm}|c|}
    \hline
    \textbf{Supervisor} & \textbf{Marcado} & \textbf{Transiciones input} & \textbf{Transiciones output} & \textbf{Bad Siphon Controlado}  \\  \hline
    $P_{14}$ & 1 & \{$T_{2}, T_{3}, T_9$\} & \{$T_{1}, T_{7}$\} & \{$P_2, P_{6}, P_{7}, P_{8}$\} \\ 
    \hline
    $P_{15}$ & 2 & \{$T_{4}, T_{5}, T_9$\} & \{$T_{1}, T_{7}$\} & \{$P_{3},P_{6},P_{7},P_{8}, P_{9}$\} \\ 
    \hline
    $P_{16}$ & 1 & \{$T_{2}, T_{5}, T_8$\} & \{$T_{1}, T_{7}$\} & \{$P_{3},P_{5},P_{8},P_{9}$\} \\ 
    \hline
    \end{tabular}
    \caption{Supervisores: RdP Hospital (R).}
    \label{tab:Hospital-SubR}
\end{table}

\begin{figure}[H]
	\centering
		\includegraphics[scale=0.45]{Figures/algoritmo3/Hospital6.png}
	\caption{Control RdP Hospital preservando lado derecho del conflicto.}
	\label{fig:ladoderechoconflictocontrol}
 \end{figure}

\subparagraph{Control red original}
\hfill \break
Al unir las soluciones individuales la red seguía presentando deadlock, lo que sucedió en este caso particular es que al unir ambas subredes entraron en juego nuevos supervisores que formaban parte del control de las plazas pertenecientes al otro \break T-invariante en conflicto. Por este motivo en caso de tomarse el camino ajeno al supervisor, la transición en conflicto debería devolver el token al mismo. Esto se puede visualizar en el caso de los supervisores representados por las plaza $P_{14}$ y $P_{16}$, dado que los mismos pertenecen al control de la subred derecha, por ende no están presentes en el control de la subred izquierda; de esta manera si la red al ejecutarse toma el camino del T-invariante izquierdo, la transición $T_2$ (que es la que da inicio a este camino) debe devolver el token consumido a estos supervisores no contemplados, conservando de esta manera la vivacidad de la red.\\
Además, se da el caso en el que las subredes individualmente presentan supervisores en común, por lo que el marcado del mismo (al momento de la unión) debía colocarse con el menor de ellos.\\
Para solucionar lo anterior se realizó lo mencionado en el punto 5.b (sección \ref{sec:desarrollov3}) del algoritmo.

\begin{figure}[H]
	\centering
		\includegraphics[scale=0.4]{Figures/algoritmo3/Hospital7.png}
	\caption{RdP Hospital controlada.}
	\label{fig:hospitalcontrolada}
 \end{figure}

\subsubsection{Caso Huang}
Esta red modela la ejecución concurrente de procesos de trabajo en FMS, representando un sistema donde se ejecutan tres tipos de procesos de trabajo.\\
En la red existen plazas que simulan la disponibilidad de recursos (4) y un control incorrecto de estos en la ejecución de los procesos de trabajo, puede conducir a situaciones de deadlock.\\

\paragraph{Características generales}
\begin{itemize}
    \item Presenta un conflicto entre T-invariantes. 
    \item Los recursos R1-R4 están representados por las plazas $\{P_6, P_7, P_{12}, P_{13}\}$
    \end{itemize}
\bigskip

\subparagraph{Análisis estructural}
\hfill
\begin{figure}[H]
	\centering
	\includegraphics[width=\textwidth]{Figures/algoritmo3/Huang1.png}
	\caption[RdP Huang y sus T-invariantes.]{RdP Huang \footnotemark \ y sus T-invariantes.}
	\label{fig:huang_T-invariantes}
 \end{figure} \footnotetext{Figura adaptada del paper publicado por \textit{Huang} et al. \cite{paperhuang} .}
\bigskip

En la figura \ref{fig:huang_T-invariantes}, la plaza $P_2$ forma parte de un conflicto permitiendo la ejecución de un subcircuito de la red u otro, pudiendo seguir dos T-invariantes diferentes (rojo o verde, en este caso).\\
Dado que los primeros 4 pasos del algoritmo no lograron alcanzar una red libre de deadlock, fue necesario ejecutarlo de forma completa, es decir, incluyendo el paso 5, contemplando la división de la red.
De la división resultaron dos subredes, preservando el recurso compartido ($P_{13}$) en cada una de ellas.

\subparagraph{Subred izquierda}
\hfill \break
Esta subred no fue necesario controlarla mediante un supervisor dado que no presenta deadlock.

\begin{figure}[H]
	\centering
	\includegraphics[scale=0.5]{Figures/algoritmo3/Huang2.png}
	\caption{Subred izquierda Huang.}
	\label{fig:Huang_subredizq}
 \end{figure}
 
 \subparagraph{Subred derecha}
\hfill \break
Esta subred no fue necesario controlarla mediante un supervisor dado que no presenta deadlock.

\begin{figure}[H]
	\centering
	\includegraphics[scale=0.5]{Figures/algoritmo3/Huang3.png}
	\caption{Subred derecha Huang.}
	\label{fig:Huang_subredder}
 \end{figure}

\subparagraph{Control subredes}
\hfill \break
Una vez ejecutado el algoritmo (4 primeros pasos) sobre la subred derecha, se obtuvo el supervisor a agregar para su control, resolviendo así el problema de deadlock.

 \begin{figure}[H]
	\centering
	\includegraphics[scale=0.5]{Figures/algoritmo3/Huang4.png}
	\caption{Subred derecha Huang controlada.}
	\label{fig:subredder_huang_controlada.}
 \end{figure}

\subparagraph{Red controlada}
\hfill \break
Al haber contemplado en ambas subredes el recurso compartido, cuando se las integró en la red completa no fue necesario agregar otro tipo de arcos para mantener el deadlock false.

\begin{table}[H]
    \small
    \centering
    \begin{tabular}{|c|c|P{2.2cm}|P{2.2cm}|c|}
    \hline
    \textbf{Supervisor} & \textbf{Marcado} & \textbf{Transiciones input} & \textbf{Transiciones output} & \textbf{Bad Siphon Controlado}  \\  \hline
    $P_{14}$ & 1 & \{$T_{9}$\} & \{$T_{7}$\} & \{$P_{11}, P_{12}, P_{13}$\} \\ 
    \hline
    \end{tabular}
    \caption{Supervisores: RdP Huang.}
    \label{tab:Huang-v3}
\end{table}

 \begin{figure}[H]
	\centering
	\includegraphics[scale=0.46]{Figures/algoritmo3/Huang5.png}
	\caption{RdP Huang controlada.}
	\label{fig:huang_controlada}
 \end{figure}
 \bigskip

\subsubsection{Caso MFC}
En esta red se modela un FMS. El mismo consta de un conjunto de estaciones de trabajo que comparten una serie de recursos como: 
\begin{itemize}
    \item Robots R1, R2 y R3 , donde cada uno puede contener un producto a la vez.
    \item Máquinas M1,M2,M3 y M4 , donde cada una puede procesar dos productos a la vez.
    \item Accesorios de vehículos guiados automáticamente (AGV).
    \item Buffers tanto de carga I1 e I2 , como de descarga O1 y O2.
\end{itemize}

\noindent Con los cuales lleva a cabo la producción de dos tipos de productos: $parte_1$ y $parte_2$.
\bigskip

\begin{figure}[H]
	\centering
	\includegraphics[scale=0.45]{Figures/algoritmo3/mfcmodelado.png}
	\caption[Modelado de partes del sistema MFC.]{Modelado de partes del sistema MFC\footnotemark.}
	\label{fig:sistema_mfc}
 \end{figure} \footnotetext{Figura adaptada del paper publicador por \textit{Mowafak H. Abdul-Hussin} \cite{papermfc} .}
\bigskip

\paragraph{Características generales}
\begin{itemize}
    \item Las máquinas M1,M2,M3,M4 están representadas por las plazas $\{P_{12}, P_{14}, P_{18}, P_{19}\}$
    \item Los robots R1,R2,R3 están representados por las plazas $\{P_{11}, P_{13}, P_{15}\}$
    \item La línea de producción de las ${parte_1}$ está representada por las plazas $\{P_1-P_5\}$, mientras que las ${parte_2}$ por las plazas $\{P_6-P_{10}\}$.
\end{itemize}

\paragraph{Análisis estructural}
\hfill
\begin{figure}[H]
	\centering
	\includegraphics[scale=0.45]{Figures/algoritmo3/MFC1.png}
	\caption[RdP MFC \ y sus T-invariantes.]{RdP MFC \footnotemark \ y sus T-invariantes.}
	\label{fig:mfc_T-invariantes} 
 \end{figure} \footnotetext{Figura adaptada del paper publicador por \textit{Mowafak H. Abdul-Hussin} \cite{papermfc} .}
 
En la figura \ref{fig:mfc_T-invariantes}, se representan los T-invariantes, en rojo y verde. \\
Sobre esta red se ejecutaron los primeros 4 pasos del algoritmo, permitiendo encontrar el supervisor que resuelva el deadlock sin necesidad de subdividirla (paso 5); incluso realizar esta acción carecía de sentido dado que los T-invariantes por separado no representan el comportamiento de la red en su totalidad.\\

\paragraph{Red controlada}
\hfill \break
Una vez incorporado el supervisor a la red, es decir, las plazas y arcos determinados por el algoritmo, se logra su control resolviendo así el problema de deadlock.

\begin{table}[H]
    \centering
    \begin{tabular}{|c|c|P{2.2cm}|P{2.2cm}|c|}
    \hline
    \textbf{Supervisor} & \textbf{Marcado} & \textbf{Transiciones input} & \textbf{Transiciones output} & \textbf{Bad Siphon Controlado}  \\  \hline
    $P_{21}$ & 2 & \{$T_{2}, T_{11}$\} & \{$T_{1}, T_{7}$\} & \{$P_{2},P_{10},P_{11}, P_{12}$\} \\ 
    \hline
    $P_{22}$ & 2 & \{$T_{3}, T_{10}$\} & \{$T_{1}, T_{7}$\} & \{$P_3, P_{9}, P_{12}, P_{13}$\} \\ 
    \hline
    $P_{23}$ & 2 & \{$T_{4}, T_{9}$\} & \{$T_{1}, T_{7}$\} & \{$P_{4},P_{8},P_{13},P_{14}$\} \\ 
    \hline
    \end{tabular}
    \caption{Supervisores: RdP MFC.}
    \label{tab:MFC-v3}
\end{table}

\begin{figure}[H]
	\centering
	\includegraphics[scale=0.5]{Figures/algoritmo3/MFC2.png}
	\caption{RdP MFC controlada.}
	\label{fig:mfc_controlada}
 \end{figure}
\bigskip

\subsubsection{Caso Portugal} \label{sub:portugal}
Esta red modela la ejecución concurrente de procesos de trabajo en FMS, representando un sistema donde se ejecutan cuatro tipos de procesos de trabajo con estaciones de trabajo (4) idénticas, con dos recursos compartidos por todas, en caso de llevar una mala gestión de estos la red terminará en deadlock.

\paragraph{Características generales}
\begin{itemize}
    \item En este caso particular la red es simétrica, pudiéndose dividir en 4 partes iguales, todas manteniendo el deadlock.
    \item Se trabajó sobre una de estas porciones de red, extendiendo luego la solución a las restantes partes.
\end{itemize}
\bigskip

\newpage
\paragraph{Análisis estructural}
\hfill

\begin{figure}[H]
	\centering
	\includegraphics[scale= 0.45]{Figures/algoritmo3/Portugal1.png}
	\caption[RdP Portugal y sus T-invariantes.]{RdP Portugal \footnotemark \ y sus T-invariantes.}
	\label{fig:portugal_T-invariantes}
 \end{figure} \footnotetext{Figura adaptada del paper publicado por \textit{S. Wang} et al. \cite{paperportugal} .}


De acuerdo con las características de la red, se trabajó sobre una de las cuatro subredes resultantes mediante el algoritmo (pasos 1 al 4), obteniendo el supervisor que controla el problema del deadlock presente en ese ¼ de red.\\
Pudiéndose aplicar la misma solución a todas las restantes subredes, dada su simetría.\\
Al momento de unirlas con sus respectivas soluciones, hay que tener en cuenta que el recurso compartido está afectando todos los T-invariantes, por lo que se deben aplicar los arcos que componen a cada supervisor ajeno.\\

\subparagraph{¼ de red}
\hfill

\begin{figure}[H]
	\centering
	\includegraphics[scale=0.6]{Figures/algoritmo3/Portugal2.png}
	\caption{Porción de la RdP Portugal.}
	\label{fig:cuartodered_portugal}
 \end{figure}
\subparagraph{Control ¼ de la red}
\hfill \break
Una vez ejecutado el algoritmo (4 primeros ítems) sobre ¼ de la red el problema de deadlock fue solucionado.

\begin{table}[H]
    \centering
    \begin{tabular}{|c|c|P{2.2cm}|P{2.2cm}|c|}
    \hline
    \textbf{Supervisor} & \textbf{Marcado} & \textbf{Transiciones input} & \textbf{Transiciones output} & \textbf{Bad Siphon Controlado}  \\  \hline
    $P_{9}$ & 2 & \{$T_{2}, T_{5}$\} & \{$T_{1}, T_{4}$\} & \{$P_3, P_{6}, P_{7}, P_{8}$\} \\ 
    \hline
    \end{tabular}
    \caption{Supervisores: RdP Portugal.}
    \label{tab:Portugal-v3}
\end{table}

\begin{figure}[H]
	\centering
	\includegraphics[scale=0.55]{Figures/algoritmo3/Portugal3.png}
	\caption{Porción de la RdP Portugal controlada.}
	\label{fig:cuartodered_portugalcontrolada}
 \end{figure}

\subparagraph{Control ½ de la red}
\hfill \break
Realizando lo mencionado en el análisis estructural se decidió unificar las soluciones de dos cuartos de la red y como se observa en la figura \ref{fig:mitadred_portugalcontrolada} la subred no presenta deadlock.\\
Lo mismo ocurriría si se unifican las 4 partes pero con el objetivo de lograr una mejor visualización del control,  se optó por no desarrollarla en el presente informe.

\begin{figure}[H]
	\centering
	\includegraphics[scale=0.45]{Figures/algoritmo3/Portugal4.png}
	\caption{Mitad de la RdP Portugal controlada.}
	\label{fig:mitadred_portugalcontrolada}
 \end{figure}
\bigskip

\subsubsection{Caso Zhao}
Esta red modela la ejecución concurrente de procesos de trabajo en FMS que describe el comportamiento de 3 subredes relacionadas por tres recursos, en caso de llevar una mala gestión de estos la red terminará en deadlock.

\paragraph{Características generales}
\begin{itemize}
    \item Presenta conflictos entre T-invariantes.
    \item Los recursos R1-R3 están representados por las plazas $\{P_{15}, P_{16}, P_{17}\}$. \\
\end{itemize}
\bigskip

\subparagraph{Análisis estructural}
\hfill \break
\begin{figure}[H]
	\centering
	\includegraphics[width=\textwidth]{Figures/algoritmo3/Zhao1.png}
	\caption[RdP Zhao y sus T-invariantes.]{RdP Zhao \footnotemark \ y sus T-invariantes.}
	\label{fig:zhao_T-invariantes}
 \end{figure} \footnotetext{Figura adaptada del paper publicado por \textit{Mi Zhao} et al. \cite{paperzhao} .}
 
En la figura \ref{fig:zhao_T-invariantes} \footnote{En esta RdP fue modificado el peso de los arcos reduciéndolos a uno (1) para adaptarla al tipo de red que admite el algoritmo.}, las plazas $P_1$ y $P_5$ forma parte (cada una) de un conflicto permitiendo su disparo la ejecución de un subcircuito de la red u otro, pudiendo seguir dos T-invariantes diferentes (rojo o verde en este caso). Pero en este caso, al igual que en la red Guanjun (sección \ref{sec:guanjun}), al ejecutar el algoritmo mencionado desde el punto 1-4 por primera vez sobre la red completa, permite encontrar los supervisores que resuelven el deadlock sin necesidad de subdividirla (es decir, ejecutar el paso 5). Dado que el token independientemente del T-invariante que siga siempre vuelve a los supervisores, como se puede observar en la figura \ref{fig:zhao_controlada}.\\

\subparagraph{Red controlada}
\hfill \break
Una vez ejecutado el algoritmo e incorporando el supervisor determinado (plaza y arcos correspondientes) a la red, se logra controlar el problema de deadlock.

\begin{table}[H]
    \centering
    \begin{tabular}{|c|c|P{2.2cm}|P{2.2cm}|c|}
    \hline
    \textbf{Supervisor} & \textbf{Marcado} & \textbf{Transiciones input} & \textbf{Transiciones output} & \textbf{Bad Siphon Controlado}  \\  \hline
    $P_{18}$ & 1 & \{$T_{2}, T_6, T_{11}$\} & \{$T_{1}, T_{9}$\} & \{$P_2, P_{6}, P_{12}, P_{15}, P_{16}$\} \\ 
    \hline
    \end{tabular}
    \caption{Supervisores: RdP Zhao.}
    \label{tab:Zhao-v3}
\end{table}

\begin{figure}[H]
	\centering
	\includegraphics[width=\textwidth]{Figures/algoritmo3/Zhao2.png}
	\caption{RdP Zhao controlada.}
	\label{fig:zhao_controlada}
 \end{figure}

\bigskip
\subsubsection{Caso Ezpeleta v2.0}
El modelo representado en esta red es un FMS, en donde existen recursos que son compartidos por varios procesos dentro de la misma (diferentes partes de la red), los cuales se ejecutan simultáneamente compitiendo por dichos recursos, pudiendo en dicha competencia alcanzar puntos muertos indeseables, los que deberían de controlarse.

\paragraph{Características generales}

\begin{itemize}
    \item Presenta un conflicto entre T-invariantes.
    \item Los lugares $\{P_{15}, P_{16}, P_{17}, P_{18}, P_{19}, P_{20}\}$ denotan R1, M2, M1, R2, M3 y R3, respectivamente.
    \item $M_0(P_2)$ = 3 y $M_0(P_{13})$ = 3 representa el número máximo de actividades concurrentes que pueden tener lugar en cada parte de la red.
\end{itemize}
\bigskip

\subparagraph{Análisis estructural}
\hfill 
\begin{figure}[H]
	\centering
	\includegraphics[width=\textwidth]{Figures/algoritmo3/ezpeletav21.png}
	\caption[RdP Ezpeleta v2 y sus T-invariantes.]{RdP Ezpeleta v2 \footnotemark \ y sus T-invariantes.}
	\label{fig:ezpeletav2_T-invariantes}
 \end{figure} \footnotetext{Figura adaptada del paper publicado por \textit{Zhong} et al. \cite{paperezpeletav2} .}
\bigskip

En este caso particular la plaza $P_1$ de la red figura \ref{fig:ezpeletav2_T-invariantes} forma parte de un conflicto permitiendo la ejecución de un subcircuito de la red u otro, pudiendo seguir dos T-invariantes diferentes.
Dado que al ejecutar los primeros 4 pasos no se alcanzo una red libre de deadlock, fue necesario ejecutar el algoritmo de forma completa (incluyendo el paso 5), es decir, dividiendo la red para lograr su control. \\
En primera instancia se dividió la red en dos subredes manteniendo por un lado los T-invariantes en conflicto (rojo y verde) y por otro lado el otro T-invariante (azul); y se observó que las mismas no presentaban deadlock figura \ref{fig:subredizq_ezpeletav2}.

\newpage
\subparagraph{Subred izquierda}
\hfill
\bigskip

\begin{figure}[H]
	\centering
	\includegraphics[width=\textwidth]{Figures/algoritmo3/ezpeletav22.png}
	\caption{Subred izquierda Ezpeleta v2.}
	\label{fig:subredizq_ezpeletav2}
 \end{figure}
\bigskip

Por este motivo:
\begin{itemize}
    \item Se dividió la red en 2 subredes, en las cuales cada una contiene sólo uno de los caminos del conflicto (por un lado rama verde y por otro rama roja) junto con el otro T-invariante (azul) presente en la red.
    \item Se ejecutaron los 4 ítems principales del algoritmo sobre cada una de estas, obteniendo los correspondientes supervisores y resolviendo así el problema de deadlock.
\end{itemize}

\newpage
\subparagraph{Lado izquierdo del conflicto y subred derecha}
\hfill
\begin{figure}[H]
	\centering
	\includegraphics[scale=0.625]{Figures/algoritmo3/ezpeletav23.png}
	\caption{RdP Ezpeleta v2 preservando lado izquierdo del conflicto.}
	\label{fig:conflictoizq_ezpeletav2}
 \end{figure}

\subparagraph{Lado derecho del conflicto y subred derecha}
\hfill \break
\begin{figure}[H]
	\centering
	\includegraphics[scale=0.625]{Figures/algoritmo3/ezpeletav24.png}
	\caption{RdP Ezpeleta v2 preservando lado derecho del conflicto.}
	\label{fig:conflictoder_ezpeletav2}
 \end{figure}
 
\subparagraph{Control subredes}
\hfill
\bigskip

\begin{table}[H]
    \centering
    \begin{tabular}{|c|c|P{2.2cm}|P{2.2cm}|c|}
    \hline
    \textbf{Supervisor} & \textbf{Marcado} & \textbf{Transiciones input} & \textbf{Transiciones output} & \textbf{Bad Siphon Controlado}  \\  \hline
    $P_{17}$ & 5 & \{$T_{3}, T_{6}$\} & \{$T_{1}, T_{10}$\} & \{$P_3, P_{5}, P_{6}, P_{11}, P_{13}, P_{P14}$\} \\ 
    \hline
    \end{tabular}
    \caption{Supervisores: RdP Ezpeleta v2 (L).}
    \label{tab:Ezpeletav2-SubL-v3}
\end{table}
\bigskip

\begin{figure}[H]
	\centering
	\includegraphics[scale=0.5]{Figures/algoritmo3/ezpeletav25.png}
	\caption{Control RdP Ezpeleta v2 preservando lado izquierdo del conflicto.}
	\label{fig:control_conflictoizq_ezpeletav2}
 \end{figure}
 \bigskip
 
 \begin{table}[H]
    \centering
    \begin{tabular}{|c|c|P{2.2cm}|P{2.2cm}|c|}
    \hline
    \textbf{Supervisor} & \textbf{Marcado} & \textbf{Transiciones input} & \textbf{Transiciones output} & \textbf{Bad Siphon Controlado}  \\  \hline
    $P_{18}$ & 4 & \{$T_{4}, T_9, T_{13}$\} & \{$T_{1}, T_{12}$\} & \{$P_3, P_5, P_{8}, P_{13}, P_{16}$\} \\ 
    \hline
    $P_{19}$ & 4 & \{$T_{3}, T_8, T_{13}$\} & \{$T_{1}, T_{12}$\} & \{$P_{2},P_3, P_{7},P_{14}, P_{16}$\} \\ 
    \hline
    $P_{20}$ & 2 & \{$T_4, T_{10}, T_{13}$\} & \{$T_{1}, T_{12}$\} & \{$P_5, P_{8}, P_{12}, P_{16}$\} \\ 
    \hline
    $P_{21}$ & 2 & \{$T_3, T_{9}, T_{13}$\} & \{$T_{1}, T_{12}$\} & \{$P_{3},P_{7},P_{13},P_{16}$\} \\ 
    \hline
    $P_{22}$ & 3 & \{$T_2, T_{8}, T_{13}$\} & \{$T_{1}, T_{12}$\} & \{$P_{2},P_{3},P_{6},P_{14}$\} \\ 
    \hline
    \end{tabular}
    \caption{Supervisores: RdP Ezpeleta v2 (R).}
    \label{tab:Ezpeletav2-SubR-v3}
\end{table}
 
\begin{figure}[H]
	\centering
	\includegraphics[scale=0.48]{Figures/algoritmo3/ezpeletav26.png}
	\caption{Control RdP Ezpeleta v2 preservando lado derecho del conflicto.}
	\label{fig:control_conflictoider_ezpeletav2}
 \end{figure}

\subparagraph{Red controlada}
\hfil \break
Como indica el paso 5 del algoritmo al llevar a cabo la solución de la red dividiéndola como se mencionó anteriormente; al momento de unificar las soluciones parciales en la red original, se deben agregar los arcos desde las transiciones en conflicto a las plazas de los supervisores que no son propios de su subred y así  devolver el token.\\

\begin{figure}[H]
	\centering
	\includegraphics[scale=0.48]{Figures/algoritmo3/ezpeletav27.png}
	\caption{RdP Ezpeleta v2 controlada.}
	\label{fig:control_ezpeletav2}
 \end{figure}

\subsection{Conclusión}
Para lograr el correcto funcionamiento del algoritmo fue de vital importancia incorporar un tercer arco, el cual Ezpeleta en su desarrollo omitía para las situaciones en donde la red presentaba conflicto entre sus T-invariantes. Decimos de vital importancia dado que si este no estuviera presente, el supervisor alcanzaría un marcado cero producto del disparo de la transición que en su trayectoria no contempla ningún arco que devuelva el token al supervisor y en consecuencia la red no volvería a ejecutarse. \\

\par Además, con el fin de obtener una mejor visualización de la aplicación del algoritmo en los diferentes escenarios, se realizó un cuadro comparativo con las características más relevantes de cada caso.

\begin{landscape}
 \begin{table}[H]
    \scriptsize 
    \centering
    \begin{tabular}{|p{1cm}|p{1.5cm}|p{0.7cm}|p{1cm}|p{2cm}|p{4cm}|p{2cm}|p{2cm}|p{2cm}|p{2cm}|}
    \hline
    \textbf{Redes} & \textbf{Conflicto entre T-invariantes} & \textbf{Tipo de red} & \textbf{Símetría} & \textbf{¿Para la solución fue necesario devidir la red?} & \textbf{Solución alternativa a partir de la división del conflicto} & \textbf{Detalles de la solución} & \textbf{Recursos compartidos forman parte de un bad siphon} & \textbf{Recursos compartidos forman parte de un invariante} & \textbf{Recursos compartidos forman parte de una trampa }  \\  \hline
    Panama & No & S³PR & Si & No & - & Al notar que es simetrica, se pensó en dividir la red y tratar una sola parte, pero al hacerlo se perdia el deadlock y no servia para el analisis. & Todos los recursos compartidos forman parte de algun bad siphon, solo uno se enuentra en todos los bad siphon. & Si, los tres recursos compartidos forman parte de ambos T-invariantes. & Una trampa contiene los tres recursos compartidos.  \\ 
    \hline
    
    Ezpeleta & Si (pero no todos) & S³PR & No & Sí (Se dividió preservando en las subredes resultantes los T-invariantes involucrados en el conflicto con recursos compartidos). & La solucion planteada fue romper el conflicto, es decir ejecutar el codigo en dos subredes que contenian respectivamente sólo uno de los T-invariante en conflicto y luego, al momento de unir las soluciones, añadir un brazo que devolviera un token al otro supervisor en caso de no tomar el camino del T-invariante correspondiente. & - & No todos los recursos compartidos forman parte de algun bad siphon. & Los recursos forman parte de algun T-invariante. Solo un T-invariante incluye todos los recursos compartidos. & Ninguna trampa contiene todos los recursos compartidos. Hay una trampa que contiene 3 de los 4 recursos compartidos.  \\ 
    \hline
    
    POPN & Si (pero no todos) & S³PR & No & Sí (Se dividió preservando en las subredes resultantes los T-invariantes involucrados en el conflicto con recursos compartidos). & La solucion planteada fue romper el conflicto, es decir ejecutar el codigo en dos subredes que contenian respectivamente sólo uno de los T-invariante en conflicto y luego, al momento de unir las soluciones, añadir un brazo que devolviera un token al otro supervisor en caso de no tomar el camino del T-invariante correspondiente. & - & No todos los recursos compartidos forman parte de algun bad siphon. & Un unico recurso esta compartido con todos los T-invariantes. Ningun T-invariante hace uso de todos los recursos compartidos. & Hay trampas que contienen todos los recursos compartidos.  \\ 
    \hline
    
    \end{tabular}
    \caption{Análisis de los casos - Parte 1.}
    \label{tab:Analisis-Casos}
\end{table}

 \begin{table}[H]
    \scriptsize 
    \centering
    \begin{tabular}{|p{1cm}|p{1.5cm}|p{0.7cm}|p{1cm}|p{4.8cm}|p{1.5cm}|p{2cm}|p{1.7cm}|p{2cm}|p{2cm}|}
    \hline
    \textbf{Redes} & \textbf{Conflicto entre T-invariantes} & \textbf{Tipo de red} & \textbf{Símetría} & \textbf{¿Para la solución fue necesario devidir la red?} & \textbf{Solución alternativa a partir de la división del conflicto} & \textbf{Detalles de la solución} & \textbf{Recursos compartidos forman parte de un bad siphon} & \textbf{Recursos compartidos forman parte de un invariante} & \textbf{Recursos compartidos forman parte de una trampa }  \\  \hline
    
    Guanjun & Si (pero no todos) & S³PR & No & No. La primera ejecución del algoritmo resolvió el problema de deadlock sin necesidad de dividirla. Se probó dividir la red teniendo en cuenta los parámetros de las anteriores (es decir a partir del conflicto y de los T-invariantes) para observar si se obtenia alguna mejora en cuanto a la cantidad de plazas supervisores y marcado. Se observó que ambas subredes no presentaban deadlock y como consecuencia de esto la división no servia. & - & - & No todos los recursos compartidos forman parte de algun bad siphon. & Los recursos forman parte de algun T-invariante. Ningun T-invariante hace uso de todos los recursos compartidos. & Ninguna trampa contiene todos los recursos compartidos.  \\ 
    \hline
    
    Hospital & Si (pero no todos) & S³PR & No & Sí. Se dividió la red manteniendo el conflicto entre los T-invariantes como en el caso de las redes "Ezpeleta" y "POPN" pero el deadlock desaparecia. La solucion planteada entonces fue romper el conflicto, es decir ejecutar el codigo en dos subredes que contenian respectivamente sólo uno de los T-invariante en conflicto y luego, al momento de unir las soluciones, añadir un brazo que devolviera un token al otro supervisor en caso de no tomar el camino del T-invariante correspondiente. & - & - & Todos los recursos compartidos forman parte de algun bad siphon, solo uno se enuentra en todos los bad siphon. & Los recursos forman parte de algun T-invariante. Dos T-invariantes incluyen todos los recursos compartidos. & Una trampa contiene los tres recursos compartidos.  \\  
    \hline
    
    Huang & Si (pero no todos) & S³PR & No & Sí (Se dividió preservando en las subredes resultantes los T-invariantes involucrados en el conflicto con recursos compartidos) & - & Es el primer caso en el que el supervisor de la solución no influye sobre el conflicto & El recurso compartido forma parte del bad siphon. & El recurso compartido forma parte de todos los T-invariantes. & Hay trampas que contienen el recurso compartido.  \\  
    \hline   
    \end{tabular}
    \caption{Análisis de los casos - Parte 2.}
    \label{tab:Analisis-Casos2}
\end{table}

 \begin{table}[H]
    \scriptsize 
    \centering
    \begin{tabular}{|p{1cm}|p{1.5cm}|p{0.7cm}|p{1cm}|p{4.5cm}|p{1.2cm}|p{3.2cm}|p{2cm}|p{2cm}|p{1.6cm}|}
    \hline
    \textbf{Redes} & \textbf{Conflicto entre T-invariantes} & \textbf{Tipo de red} & \textbf{Símetría} & \textbf{¿Para la solución fue necesario devidir la red?} & \textbf{Solución alternativa a partir de la división del conflicto} & \textbf{Detalles de la solución} & \textbf{Recursos compartidos forman parte de un bad siphon} & \textbf{Recursos compartidos forman parte de un invariante} & \textbf{Recursos compartidos forman parte de una trampa }  \\  \hline
    
    MFC & No & S³PR & Si & No & - & Al notar que es simetrica, se pensó en dividir la red y tratar una sola parte, pero al hacerlo se perdia el deadlock y no servia para el analisis. & Todos los recursos compartidos forman parte de algun bad siphon. & Todos los recursos compartidos forman parte de ambos T-invariantes. & Una trampa que contiene todos los recursos compartidos. \\
    \hline
    
    Portugal & No & S³PR & Si & Sí. Se dividió dada la simetría tratando sólo una de las partes con el algoritmo y extendiendo la solución a las restantes. Simplicidad para el analisis. & - & Al trabajarlas por separado se tuvieron que tener en cuenta los brazos que devolvian token a los supervisores "ajenos" a las subredes analizadas & Todos los recursos compartidos forma parte del bad siphon. & Todos los recursos compartidos forman parte de ambos T-invariantes. & Una trampa que contiene todos los recursos compartidos. \\
    \hline
    
    Zhao & Si (pero no todos) & S³PR & No & No & - & - & No todos los recursos compartidos forman parte del bad siphon. & Los recursos compartidos forman parte de todos los T-invariantes. & Hay trampas que contienen los recursos compartido. \\
    \hline
    
    Ezpeleta v2.0 & Si (pero no todos) & S³PR & No & Sí. Se dividió la red manteniendo el conflicto entre los T-invariantes como en el caso de las redes "Ezpeleta" y "POPN" pero el deadlock desaparecia. La solucion planteada entonces fue romper el conflicto, es decir ejecutar el codigo en dos subredes que contenian respectivamente sólo uno de los T-invariante en conflicto y luego, al momento de unir las soluciones, añadir un brazo que devolviera un token al otro supervisor en caso de no tomar el camino del T-invariante correspondiente. & - & - & No todos los recursos compartidos forman parte de los bad siphon. & Los recursos forman parte de algun T-invariante. Solo dos recursos compartidos afectan a todos los T-invariantes. Dos T-invariantes hacen uso de todos los recursos compartidos. & Hay trampas que contienen todos los recursos compartidos. \\
    \hline
    
    \end{tabular}
    \caption{Análisis de los casos - Parte 3.}
    \label{tab:Analisis-Casos3}
\end{table}
\end{landscape}

\newpage
El criterio con el que se eligieron las columnas del mismo se debe a que cada red analizada estaba constituida por bad siphons, recursos compartidos y T-invariantes donde no todos se presentaban de la misma manera; y a partir de las mismas se buscaba encontrar patrones de comportamiento repetitivos o similares con el objetivo de lograr una mejora del algoritmo para obtener así uno más abarcativo.\\

\noindent Del análisis del mismo se pudieron realizar las siguientes observaciones:
\begin{enumerate}
    \item Al agregar en la red las plazas y los arcos que componen al supervisor, la red modifica su comportamiento limitando su grafo de alcanzabilidad a estados deseables y evitando así que la misma evolucione a un posible estado de deadlock.
    Por cada supervisor incorporado en la red se generan:
    \begin{itemize}
        \item Nuevos sifones
        \item Nuevas trampas
        \item Un nuevo P-invariante
    \end{itemize}
    Dentro de los cuales se puede destacar la presencia de un sifón, una trampa y un P-invariante compuestos por el mismo conjunto de plazas; en los cuales siempre se incluye la plaza perteneciente al supervisor y las plazas complemento del bad siphon a controlar, entre otras plazas.\\ 
    La presencia de una trampa y un sifón conformados por las mismas plazas implica que este último nunca se va a vaciar logrando así su control.
    
    \item Al menos un recurso compartido forma parte de algún T-invariante.
    
    \item El algoritmo fue desarrollado para analizar redes del tipo S³PR.
    
    \item Si la red a analizar es simétrica: se busca trabajar con la red mínima (simplificando el análisis; como sucede en el caso Portugal) controlando la misma y luego extendiendo dicha solución a la totalidad de la red. \\
    Pueden presentarse casos en los que las redes sean simétricas y mínimas por lo que no es necesario dividirlas para llevar a cabo el análisis, como es el caso de Panamá y MFC.
    
    \item Para el caso de las redes (Ezpeleta, POPN, Huang, Ezpeleta v2.0 y Hospital) que presentan conflicto entre T-invariantes fue necesario dividir en el análisis en diferentes subredes para lograr así controlar el deadlock. Para esta división lo que se hizo fue mantener en las diferentes subredes el conflicto y luego unir todas las soluciones. Pero en ciertas redes (Hospital y Ezpeleta v2.0) al llevar a cabo la división, antes planteada, las subredes resultantes no presentaban deadlock por lo cual el análisis no podría realizarse, para esto lo que se planteó fue dividir la red en diferentes subredes pero sin mantener el conflicto y luego a las subredes resultantes, al momento de integrarlas en una única red,  se debe agregar un brazo que devuelva el token desde la subred al o a los supervisor o supervisores de la otra subred.
    
    \item Para la división de la red, en diferentes subredes, es condición necesaria que exista conflicto entre T-invariantes pero no condición suficiente (a excepción de Huang).
    
    \item La red Huang al igual que las que se incluyen en la generalización antes mencionada presenta conflicto, pero al dividirla y analizarla como las anteriores, los supervisores encontrados no convergen en una solución sin deadlock. Para el análisis de esta red lo que se hizo fue dividirla en dos subredes manteniendo el recurso compartido (común a todos los T-invariantes) en ambas; al realizar esto se observó que la subred que presentaba el conflicto era una subred sin deadlock por lo cual no hizo falta el agregado de supervisores. Por otro lado, la otra subred si presentaba deadlock y fue analizada para su control mediante el algoritmo. Finalmente, al unir la subred izquierda con la subred derecha ya controlada, la red resultante está libre de deadlock.
\end{enumerate}

\subsection{Extensión: algoritmo v3.1}
A partir de la observación 6, en la cual las redes Hospital y Ezpeleta v2.0 presentaban conflictos con el algoritmo, se decidió poner énfasis en esto y realizar una modificación del algoritmo logrando que el mismo llegue a una generalización que resuelva cada uno de los casos.\\
En esta extensión, se decide dividir las redes en el conflicto, pero sin mantenerlo como se hacía anteriormente. Y al unir las redes, se ejecuta una nueva parte del algoritmo que detecta qué transición (de las que formaban parte originalmente del conflicto) debe devolver un token al supervisor que no forma parte del control de la subred en cuestión (es decir, aquellos supervisores que no forman parte de su T-invariante). Esta última solución planteada al ejecutarse en las otras redes (Ezpeleta, POPN) también resuelve el deadlock.\\
\bigskip

%----------------------------------------------------------------------------------------
%	SECTION 5
%----------------------------------------------------------------------------------------
\section{Iteración 4: Algoritmo v4.0} \label{sec:algoritmo4}

\subsection{Introducción}
A partir del análisis llevado a cabo en las secciones anteriores se realizó una última versión del algoritmo logrando generalizar su aplicación a todas las redes antes analizadas \textbf{sin la necesidad de dividir las mismas en subredes} como se presentó en diversos casos.

\subsection{Objetivos}
\begin{itemize}
    \item Generalizar el algoritmo sin la necesidad de tener que dividir las redes para su análisis.
    \item Preservar los T-invariantes de la red original.
\end{itemize}

\subsection{Desarrollo}
Para alcanzar la versión final del algoritmo se analizaron las iteraciones anteriores, logrando encontrar dos puntos importantes a tener en cuenta. Uno de ellos es la generalización del algoritmo, para que pueda ser ejecutado en cada una de las redes sin necesidad de dividirlas. Y por otro lado, se observó que al agregar los arcos de los supervisores a las plazas idle, existían casos en los que dichos arcos “rompían” el T-invariante de la red original, provocando una situación de deadlock. \\
En busca de evitar que la red alcance este estado y lograr recuperar dichos \break T-invariantes, se comprendió que una de las especificaciones que planteaba Ezpeleta no se tiene que cumplir en todos los supervisores. Él plantea que las transiciones idle deben quitarle token a todos los supervisores, pero sucede que los supervisores aplican su control en un segmento acotado de la red y no en la red en su totalidad; se notó que la existencia de este brazo en las redes analizadas producía que los \break T-invariantes de la red original se perdieran. \\ 
Para esto, fue necesario llevar a cabo la siguiente prueba en cada uno de los t\_idle que presentaban problemas con los T-invariantes. \\
Se comienza con la verificación de que cada T-invariante esté realizando la devolución del token tomado del supervisor $V_s$, en caso contrario, probar desde cuál transición del T-invariante tendría que devolver el mismo. Esta prueba se realizó de la siguiente manera: comenzando desde la última transición que forma parte del \break T-invariante agregando un arco que le devuelva el token al supervisor. Si el arco  que devuelve el token llega al inicio sin que se provoque deadlock, esto quiere decir, que ese T-invariante no necesita utilizar el token del supervisor $V_s$ por lo que el arco hacia la transición idle que extrae tokens del mismo debe eliminarse.

\subsubsection{Ejecución del algoritmo}
\noindent El algoritmo puede ser ejecutado de diferentes maneras, dependiendo el estado del análisis de la red: 

\begin{enumerate}
    \item \textbf{Primer análisis de la red}: incluye la versión 3.0 del algoritmo y además permite extraer información relevante de la red original, como las transiciones en conflictos y los T-invariantes que servirán para el tratamiento en las próximas iteraciones. Este análisis \textbf{sólo} se realiza la primera vez.
    
    \item \textbf{Análisis de red con supervisores}: esta ejecución se realiza de manera iterativa en busca de supervisores a incorporar hasta obtener \textit{deadlock false}. En caso de que eso no suceda y ya no indique más supervisores por colocar; el siguiente paso es realizar la opción “3” que es un tratamiento de conflicto y t\_idle.
    
    \item \textbf{Red con supervisores, tratamiento de conflicto y t\_idle}: el objetivo de esta ejecución es determinar que aquellas \textit{transiciones en conflicto} le devuelvan los tokens a los supervisores en caso que el T-invariante al que pertenecen no hagan uso del mismo. Mientras que para las t\_idle en la implementación de la versión 3.0 del algoritmo todas estas le \textit{quitan token a los supervisores}, pero no siempre al T-invariante al cual pertenecen le realizan la devolución del mismo y esto desencadena en el bloqueo de la red. Por lo tanto, el objetivo de esta ejecución es encontrar estas t\_idle y eliminar el arco que produce la extracción de esos supervisores.
\end{enumerate}

\begin{figure}[H]
	\centering
	\includegraphics[width=\textwidth]{Figures/algoritmo4/terminal-ejecucion-algoritmo.png}
	\caption{Ejecución del algoritmo versión 4.0.}
	\label{fig:algoritmov4}
\end{figure}


\subsubsection{Pseudocódigo}
\noindent Definiendo:
\begin{itemize}
    \item Cantidad de sifones = cantidades de sifones total del sistema.
    \item SD: state deadlock.
    \item S: conjunto de sifones.
    \item BS: conjunto de bad siphons.
    \item VS: plaza supervisor.
    \item TC: transiciones en conflicto. 
\end{itemize}

\begin{algorithm} [H]
  \floatname{algorithm}{Pseudocódigo}
  \caption{Busqueda de bad siphon a controlar (v4)}
  \label{alg:algoritmo3} 
  \begin{algorithmic}[1]
 
    % ENTRADA / SALIDA
    \Require{RdP ($N,M_0$) de tipo S³PR.}
    \Ensure{bad siphon.}
 
    \State Generar el grafo de alcanzabilidad G($N,M_0$) de la RdP.
    \State Obtener matrices I+, I-, invariantes, trampas y sifones.
 
    \For{i: 0 \textbf{to} cantidad de estados}
        \If{estado = estado en deadlock}
            \State $SD_j$ $\leftarrow$ \ $estado_i$
        \EndIf
    \EndFor
    
    \If{estado =  idle}
        \For{i: 0 \textbf{to} cantidad de sifones}
            \If{M($S_i$) = 0}
                \State $BS_{idle}$ $\leftarrow$ \  $S_i$
            \EndIf
        \EndFor
    \EndIf
    
    \State en SD[0]
    \For{i: 0  \textbf{to}  cantidad de sifones}
        \If{$M(S_i)$ = 0}
            \State $BS_{SD}$ $\leftarrow$ \ $S_i$  
        \EndIf
    \EndFor
    \State Se eliminan de $BS_{SD}$ aquellos que estén en $BS_{idle}$ 
    \State $BS_{SD}$[0] \  $\rightarrow$\ control de bad siphon usando \textbf{Algoritmo \ref{alg:algoritmo4}}
  \end{algorithmic}
\end{algorithm}
    

\begin{algorithm}[H] 
  \floatname{algorithm}{Pseudocódigo}
  \caption{Búsqueda de supervisor controle bad siphon (v4)}
  \label{alg:algoritmo4} 
  \begin{algorithmic}[1]
 
    % ENTRADA / SALIDA
    \Require{RdP ($N,M_0$) de tipo S³PR.}
    \Ensure{supervisor.}
 
    \State \textbf{Agregar} plaza de control $VS_i$ con $M(VS_i)= M(BS_{SD}) - 1$
    
    \If{estado = idle}
            \State \textbf{Agregar} arco $(VS_i , t) \ \forall \  t \in \ p\bullet$
    \EndIf
 
    \State \textbf{Agregar} arco $(t, VS_i) \forall \ t \in \ C_S\bullet$
    \State \textbf{Agregar} arco $(t, VS_i)  \forall \ t \in \ conflicto \ \wedge \ t \notin T-invariante_{BS}$
  \end{algorithmic}
\end{algorithm}
\bigskip


\begin{algorithm}[H] 
  \floatname{algorithm}{Pseudocódigo}
  \caption{Búsqueda de supervisor controle bad siphon (v4)}
  \label{alg:algoritmo5} 
  \begin{algorithmic}[1]
 
    % ENTRADA / SALIDA
    \Require{RdP ($N,M_0$) de tipo S³PR.}
    \Ensure{supervisor.}
    
    \State \textbf{Obtener:} 
    \begin{itemize}
        \item Transiciones en conflicto.
        \item Supervisores agregados.
        \item T-invariantes de la red original (sin supervisores).
        \item Matriz post (I+)
        \item Matriz pre (I-)
    \end{itemize}
    
    \For{i: 0 \textbf{to} \textit{len}(t\_idle)}
        \For{: 0 \textbf{to} Cantidad de T-invariante}
            \If{$t\_idle_i \ \in \ T-invariante_j$}
                \For{ m: 0 \textbf{to} Cantidad de supervisores}
                    \If{$\notin$ \ transición del $T-invariante_j$ al cual pertenece la $t\_idle_i$ que devuelva token al $VS_m$}
                        \For{i: 0 \textbf{to} \textit{len}(TC)}
                            \If{$TC_k \ \in \ T-invariante_j$}
                                \State \textbf{Agregar} arco ($TC_k, \ VS_m$)
                            \EndIf
                        \EndFor
                        \If{$\notin$ transición del T-invariante que $\in$ TC} 
                            \If{$\in$ arco ($VS_m, \ t\_idle_i$)}
                                \State \textbf{Eliminar} arco ($VS_m, \ t\_idle_i$) 
                            \EndIf
                        \EndIf
                    \EndIf
                \EndFor
            \EndIf
        \EndFor
    \EndFor
    
  \end{algorithmic}
\end{algorithm}
\bigskip

\newpage
\subsubsection{Diagramas de secuencia}
El algoritmo iterativo para alcanzar una red de Petri sin deadlock se muestra a partir de los siguientes diagramas de secuencia: \\
\bigskip

\begin{figure}[H]
	\centering
	\includegraphics[width=\textwidth]{Figures/diagramasecuencia/Diagrama0.jpeg}
	\caption{Diagrama de secuencia: procesamiento de información y tipos de ejecución.}
	\label{fig:diagrama-sec}
\end{figure}
\bigskip

En la figura \ref{fig:diagrama-sec} se observa la manipulación de los datos extraídos del software utilizado para el análisis (en este caso Petrinator) desde el archivo \textit{filter\_data.py}; junto con las opciones iniciales de ejecución del algoritmo. Cada una de las opciones luego se desarrolla individualmente para entrar más en detalle de la secuencia del mismo.

\begin{figure}[H]
	\centering
	\includegraphics[width=\textwidth]{Figures/diagramasecuencia/Diagrama2.png}
	\caption{Diagrama de secuencia: ejecución del primer o análisis de red con supervisores.}
	\label{fig:diagrama-sec2}
\end{figure}
\bigskip

En la figura \ref{fig:diagrama-sec2} se muestra la ejecución en caso de seleccionar la opción “Primer análisis de la red y Análisis de red con supervisores”, en caso de que el \textit{análisis = 1} se exportan los archivos que serán necesarios (cantidad de plazas, T-invariantes y transiciones en conflicto de la red original, es decir, sin ningún supervisor) en caso de que la red presente conflicto entre T-invariantes y/o sea necesario el tratamiento de t\_idle. \\
La función \textit{supervisor} que es la encargada de indicar al usuario el supervisor a colocar con el marcado y los arcos del mismo se muestra a continuación en la figura \ref{fig:diagrama-sec3}.

\begin{figure}[H]
	\centering
	\includegraphics[width=\textwidth]{Figures/diagramasecuencia/Diagrama3.png}
	\caption{Diagrama de secuencia: función que incorpora al supervisor.}
	\label{fig:diagrama-sec3}
\end{figure}

Para verificar si la red presenta conflicto se hace uso de la función \textit{path\_conflict} , cuya secuencia de ejecución se ve reflejada en la figura \ref{fig:diagrama-sec4}.

\begin{figure}[H]
	\centering
	\includegraphics[width=\textwidth]{Figures/diagramasecuencia/Diagrama4.png}
	\caption{Diagrama de secuencia: función path conflict.}
	\label{fig:diagrama-sec4}
\end{figure}

En caso de que análisis = 3, es decir, que sea necesario llevar a cabo un tratamiento de conflicto y t\_idle, el funcionamiento del mismo se muestra a continuación.

\begin{figure}[H]
	\centering
	\includegraphics[width=\textwidth]{Figures/diagramasecuencia/Diagrama5.png}
	\caption{Diagrama de secuencia: tratamiento de conflictos y t\_idle.}
	\label{fig:diagrama-sec5}
\end{figure}

En esta sección se hace uso de los archivos exportados cuando se realizó el primer análisis de la red original; es necesario utilizar estos archivos dado que de esta manera podemos: 
\begin{itemize}
    \item Discernir cuáles son los supervisores de la red controlada, realizando una OR con las plazas de la red original.
    
    \item Mantener el comportamiento de la red original pero acotando su alcanzabilidad a estados sin deadlock, para esto es necesario conocer los T-invariantes presentes en la red sin controlar dado que al agregar los supervisores estos invariantes sufren modificaciones y hasta en ocasiones desaparecen de la misma.
\end{itemize}


\subsubsection{Secuencia de ejecución}
En primera instancia se extraen los archivos necesarios del software (Petrinator) y una vez que se obtuvieron se procede de la siguiente manera: \\
Llevar a cabo el “Primer análisis de la red” y luego colocar uno de los supervisores que sugiere el algoritmo.  \\
En caso de que la red aún presente deadlock proseguir con la ejecución bajo la elección de “Análisis de red con supervisores” y continuar colocando algún supervisor sugerido hasta que la red esté controlada. \\


\noindent En caso de que el algoritmo continúe indicando algún supervisor:
\begin{itemize}
    \begin{enumerate}
        \item Con marcado 0 (cero) ó 
        \item Con solo arcos de salida y sin arcos de entrada ó 
        \item Algún supervisor que ya existe.
    \end{enumerate}
\end{itemize}
Se lleva a cabo el análisis “Red con supervisores, tratamiento de conflictos y t\_idle”, el cual indica que arcos agregar/quitar para lograr alcanzar el control de la red. \\
Si la red no se controla puede ser que la elección de los supervisores no haya sido la correcta y debería volver a ejecutar el “Primer análisis de la red” con la red original, eligiendo otro supervisor.
\bigskip

\subsubsection{Ejecución en diferentes escenarios}
Se llevó a cabo un nuevo análisis en cada una de las redes mediante esta nueva re-versión del algoritmo, verificando su funcionamiento y en busca de obtener mejores resultados a través de la generalización del mismo. \\
Este nuevo algoritmo se ejecuta sobre la totalidad de la red, permitiendo así obtener los supervisores necesarios para resolver el deadlock de la misma. \\
Si bien el análisis se realizó sobre todas las redes mencionadas anteriormente, en está sección sólo se visualizan aquellas que en la versión previa fueron necesario dividirlas para analizarlas.
\bigskip

%-------------------------------------------------------------
% CASO EZPELETA
%-------------------------------------------------------------
\newpage
\paragraph{Caso Ezpeleta}
\subparagraph{Análisis estructural}
\hfill \break

\begin{figure}[H]
	\centering
	\includegraphics[width=\textwidth]{Figures/algoritmo4/ezpeleta_imag1.png}
	\caption[RdP Ezpeleta y sus T-invariantes.]{RdP Ezpeleta \footnotemark \ y sus T-invariantes.}
	\label{fig:Rdp-Ezpeletav4}
\end{figure} \footnotetext{Figura adaptada del paper publicador por \textit{Ezpeleta} et al. \cite{paperezpeleta} .}
\bigskip

En la figura \ref{fig:Rdp-Ezpeletav4}, se observan los T-invariantes y la plaza P2 que forma parte de un conflicto.
\bigskip

\subparagraph{Control de la red}
\hfill \break 

\begin{table}[H]
    \small
    \centering
    \begin{tabular}{|c|c|P{2.2cm}|P{2.2cm}|c|}
    \hline
    \textbf{Supervisor} & \textbf{Marcado} & \textbf{Transiciones input} & \textbf{Transiciones output} & \textbf{Bad Siphon Controlado}  \\  \hline
    $P_{17}$ & 2 & \{$T_{2}, T_{4}, T_{9}$\} & \{$T_{1}, T_{7}$\} & \{$P_7,P_{8},P_{10},P_{19}$\} \\ 
    \hline
    $P_{16}$ & 2 & \{$T_{2}, T_{5}, T_{8}$\} & \{$T_{1}, T_{7}$\} & \{$P_{10},P_{11},P_{13},P_{14}$\} \\ 
    \hline
    \end{tabular}
    \caption{Supervisores: RdP Ezpeleta - Análisis 1 y 2.}
    \label{tab:Ezpeleta12-v4}
\end{table}

\begin{figure}[H]
	\centering
	\includegraphics[width=\textwidth]{Figures/algoritmo4/ezpeleta_deadlock_false.png}
	\caption{RdP Ezpeleta controlada.}
	\label{fig:Rdp-Ezpeleta-Contv4}
\end{figure}
\bigskip

%-------------------------------------------------------------
% CASO POPN
%-------------------------------------------------------------
\paragraph{Caso POPN}
\subparagraph{Análisis estructural}
\hfill

\begin{figure}[H]
	\centering
	\includegraphics[scale=0.7]{Figures/algoritmo4/popn_imag1.png}
	\caption[RdP POPN y sus T-invariantes.]{RdP POPN \footnotemark \ y sus T-invariantes.}
	\label{fig:Rdp-POPNv4}
\end{figure} \footnotetext{Figura adaptada del libro \textit{System Modeling and Control with Resource-OrientedPetri Nets} \cite{libropopn} .}

En la imagen \ref{fig:Rdp-POPNv4}, la plaza $P_{16}$ forma parte de un conflicto, dado que el marcado de la misma determina la ejecución de un subcircuito de la red u otro, pudiendo seguir dos T-invariantes diferentes. 
\bigskip

\subparagraph{Control de la red}
\hfill \break

\begin{table}[H]
    \small
    \centering
    \begin{tabular}{|c|c|P{2cm}|P{2.3cm}|c|}
    \hline
    \textbf{Supervisor} & \textbf{Marcado} & \textbf{Transiciones input} & \textbf{Transiciones output} & \textbf{Bad Siphon Controlado}  \\  \hline
    $P_{27}$ & 1 & \{$T_{15}, T_{19}$\} & \{$T_{6}, T_{12}, T_{17}$\} & \{$P_4, P_{12}, P_{14}, P_{20}, P_{22}, P_{25}$\} \\ 
    \hline
    $P_{28}$ & 1 & \{$T_{3}, T_{8}$\} & \{$T_{6}, T_{12}, T_{17}$\} & \{$P_{3},P_{8},P_{12},P_{14}, P_{19}, P_{23}, P_{25}$\} \\ 
    \hline
    $P_{29}$ & 1 & \{$T_{4}, T_{9}$\} & \{$T_{6}, T_{12}, T_{17}$\} & \{$P_{4},P_{9},P_{13},P_{14}, P_{19}, P_{23}, P_{25}$\} \\ 
    \hline
    $P_{30}$ & 1 & \{$T_{5}, T_{10}$\} & \{$T_{6}, T_{12}, T_{17}$\} & \{$P_{5},P_{9},P_{10},P_{15}$\} \\ 
    \hline
    \end{tabular}
    \caption{Supervisores: RdP POPN - Análisis 1 y 2.}
    \label{tab:POPN12-v4}
\end{table}
\hfill
\bigskip

\begin{figure}[H]
	\centering
	\includegraphics[width=\textwidth]{Figures/algoritmo4/popn_imag2.png}
	\caption{RdP POPN ejecutando el análisis 1 y 2.}
	\label{fig:Rdp-POPN-1y2v4}
\end{figure}
\bigskip

Al agregar los supervisores en la red la misma aún presentaba deadlock, lo que se hizo fue ejecutar la parte 3 del algoritmo para resolver tanto el problema de conflicto y de t\_idle.

\begin{figure}[H]
	\centering
	\includegraphics[width=\textwidth]{Figures/algoritmo4/terminal-POPN-Ejec3.png}
	\caption{Resultado al ejecutar el análisis 3.}
	\label{fig:Rdp-POPN-3v4}
\end{figure}

\begin{table}[H]
    \small
    \centering
    \begin{tabular}{|c|c|P{2.2cm}|P{2.2cm}|c|}
    \hline
    \textbf{Supervisor} & \textbf{Marcado} & \textbf{Transiciones input} & \textbf{Transiciones output} & \textbf{Bad Siphon Controlado}  \\  \hline
    $P_{27}$ & 1 & \{$T_{7}, T_{15}, T_{19}$\} & \{$T_{12}, T_{17}$\} & \{$P_4, P_{12}, P_{14}, P_{20}, P_{22}, P_{25}$\} \\ 
    \hline
    $P_{28}$ & 1 & \{$T_{3}, T_{8}, T_{14}$\} & \{$T_{6}, T_{12}$\} & \{$P_{3},P_{8},P_{12},P_{14}, P_{19}, P_{23}, P_{25}$\} \\ 
    \hline
    $P_{29}$ & 1 & \{$T_{4}, T_{9}, T_{14}$\} & \{$T_{6}, T_{12}$\} & \{$P_{4},P_{9},P_{13},P_{14}, P_{19}, P_{23}, P_{25}$\} \\ 
    \hline
    $P_{30}$ & 1 & \{$T_{5}, T_{10}, T_{14}$\} & \{$T_{6}, T_{12}$\} & \{$P_{5},P_{9},P_{10},P_{15}$\} \\ 
    \hline
    \end{tabular}
    \caption{Supervisores: RdP POPN - Análisis 3.}
    \label{tab:POPN3-v4}
\end{table}

\begin{figure}[H]
	\centering
	\includegraphics[width=\textwidth]{Figures/algoritmo4/popn_imag3.png}
	\caption{RdP POPN controlada.}
	\label{fig:Rdp-POPN-Contv4}
\end{figure}
\bigskip


%-------------------------------------------------------------
% CASO HOSPITAL
%-------------------------------------------------------------
\paragraph{Caso Hospital}
\subparagraph{Análisis estructural}
\hfill

\begin{figure}[H]
	\centering
	\includegraphics[width=\textwidth]{Figures/algoritmo4/hospital_imag1.png}
	\caption[RdP Hospital y sus T-invariantes.]{RdP Hospital \footnotemark \ y sus T-invariantes.}
	\label{fig:Rdp-Hospitalv4}
\end{figure} \footnotetext{Figura adaptada del paper publicado por \textit{A. Timotei y J. Colom} \cite{paperhospital}.}

En este caso particular de red en el que la plaza $P_1$ forma parte de un conflicto, dado que el marcado de la misma determina la ejecución de un subcircuito de la red u otro, pudiendo seguir dos T-invariantes diferentes.

\subparagraph{Control de la red}
\hfill

\begin{table}[H]
    \small
    \centering
    \begin{tabular}{|c|c|P{2.2cm}|P{2.2cm}|c|}
    \hline
    \textbf{Supervisor} & \textbf{Marcado} & \textbf{Transiciones input} & \textbf{Transiciones output} & \textbf{Bad Siphon Controlado}  \\  \hline
    $P_{14}$ & 3 & \{$T_{4}, T_{5}, T_{9}$\} & \{$T_{1}, T_{7}$\} & \{$P_4, P_{7}, P_{8}, P_{9}, P_{10}, P_{11}$\} \\ 
    \hline
    $P_{15}$ & 1 & \{$T_{5}, T_{8}$\} & \{$T_{1}, T_{7}$\} & \{$P_{4},P_{6},P_{9},P_{10}$\} \\ 
    \hline
    \end{tabular}
    \caption{Supervisores: RdP Hospital - Análisis 1 y 2.}
    \label{tab:Huang12-v4}
\end{table}

\begin{figure}[H]
	\centering
	\includegraphics[width=\textwidth]{Figures/algoritmo4/hospital_imag2.png}
	\caption{RdP Hospital ejecutando el análisis 1 y 2.}
	\label{fig:Rdp-Hospital1y2v4}
\end{figure}

Al agregar los supervisores mencionados en el cuadro previo en la red, la misma aún presentaba deadlock sin indicar la posibilidad de incorporar nuevos supervisores para su control por lo tanto lo que se hizo fue ejecutar la parte 3 del algoritmo para resolver el problema de conflicto.
\bigskip

\begin{figure}[H]
	\centering
	\includegraphics[width=\textwidth]{Figures/algoritmo4/terminal-Hospital-Ejec3.png}
	\caption{Resultado al ejecutar el análisis 3.}
	\label{fig:Rdp-Hospital3v4}
\end{figure}
\bigskip

\begin{table}[H]
    \small
    \centering
    \begin{tabular}{|c|c|P{2.2cm}|P{2.2cm}|c|}
    \hline
    \textbf{Supervisor} & \textbf{Marcado} & \textbf{Transiciones input} & \textbf{Transiciones output} & \textbf{Bad Siphon Controlado}  \\  \hline
    $P_{14}$ & 3 & \{$T_{4}, T_{5}, T_{9}$\} & \{$T_{1}, T_{7}$\} & \{$P_4, P_{7}, P_{8}, P_{9}, P_{10}, P_{11}$\} \\ 
    \hline
    $P_{15}$ & 1 & \{$T_{2}, T_{5}, T_{8}$\} & \{$T_{1}, T_{7}$\} & \{$P_{4},P_{6},P_{9},P_{10}$\} \\ 
    \hline
    \end{tabular}
    \caption{Supervisores: RdP Hospital - Análisis 3.}
    \label{tab:Hospital3-v4}
\end{table}
\bigskip

\begin{figure}[H]
	\centering
	\includegraphics[width=\textwidth]{Figures/algoritmo4/hospital_imag3.png}
	\caption{RdP Hospital controlada.}
	\label{fig:Rdp-Hospital-Contv4}
\end{figure}
\bigskip

%-------------------------------------------------------------
% CASO HUANG
%-------------------------------------------------------------
\paragraph{Caso Huang}
\subparagraph{Análisis estructural}
\hfill

\begin{figure}[H]
	\centering
	\includegraphics[width=\textwidth]{Figures/algoritmo4/huang_imag1.png}
	\caption[RdP Huang y sus T-invariantes.]{RdP Huang \footnotemark \ y sus T-invariantes.}
	\label{fig:Rdp-Huangv4}
\end{figure} \footnotetext{Figura adaptada del paper publicado por \textit{Yi-Sheng Huang} et al. \cite{paperhuang} .}

En la figura \ref{fig:Rdp-Huangv4}, la plaza $P_2$ forma parte de un conflicto, dado que el marcado de la misma determina la ejecución de un subcircuito de la red u otro, pudiendo seguir dos T-invariantes diferentes (rojo o verde, en este caso). 

\subparagraph{Control de la red}
\hfill

\begin{table}[H]
    \small
    \centering
    \begin{tabular}{|c|c|P{2.2cm}|P{2.2cm}|c|}
    \hline
    \textbf{Supervisor} & \textbf{Marcado} & \textbf{Transiciones input} & \textbf{Transiciones output} & \textbf{Bad Siphon Controlado}  \\  \hline
    $P_{14}$ & 1 & \{$T_{9}$\} & \{$T_{1}, T_{7}$\} & \{$P_2, P_{5}, P_{11}, P_{12}, P_{13}$\} \\ 
    \hline
    \end{tabular}
    \caption{Supervisores: RdP Huang - Análisis 1 y 2.}
    \label{tab:Huang12-v4}
\end{table}

\begin{figure}[H]
	\centering
	\includegraphics[width=\textwidth]{Figures/algoritmo4/huang_imag2.png}
	\caption{RdP Huang ejecutando el análisis 1 y 2.}
	\label{fig:Rdp-Huang3v4}
\end{figure}

Una vez que se agregó el supervisor mencionado en el cuadro anterior a la red, la misma aún presentaba deadlock sin indicar la posibilidad de incorporar nuevos supervisores para su control por lo tanto lo que se hizo fue ejecutar la parte 3 del algoritmo para resolver el problema de conflicto.
\bigskip

\begin{figure}[H]
	\centering
	\includegraphics[width=\textwidth]{Figures/algoritmo4/terminal-Huang-Ejec3.png}
	\caption{Resultado al ejecutar el análisis 3.}
	\label{fig:Rdp-Huang1y2v4}
\end{figure}
\bigskip

\begin{table}[H]
    \small
    \centering
    \begin{tabular}{|c|c|P{2.2cm}|P{2.2cm}|c|}
    \hline
    \textbf{Supervisor} & \textbf{Marcado} & \textbf{Transiciones input} & \textbf{Transiciones output} & \textbf{Bad Siphon Controlado}  \\  \hline
    $P_{14}$ & 1 & \{$T_{2}, T_{3}, T_{9}$\} & \{$T_{1}, T_{7}$\} & \{$P_2, P_{5}, P_{11},P_{12}, P_{13}$\} \\ 
    \hline
    \end{tabular}
    \caption{Supervisores: RdP Huang - Análisis 3.}
    \label{tab:Huang3-v4}
\end{table}
\bigskip

\begin{figure}[H]
	\centering
	\includegraphics[width=\textwidth]{Figures/algoritmo4/huang_imag3.png}
	\caption{RdP Huang controlada.}
	\label{fig:Rdp-Huang-Contv4}
\end{figure}
\bigskip

%-------------------------------------------------------------
% CASO EZPELETA V2.0
%-------------------------------------------------------------
\paragraph{Caso Ezpeleta v2.0}
\subparagraph{Análisis estructural}
\hfill

\begin{figure}[H]
	\centering
	\includegraphics[width=\textwidth]{Figures/algoritmo4/ezpeleta_v2_imag1.png}
	\caption[RdP Ezpeleta v2.0 y sus T-invariantes.]{RdP Ezpeleta v2.0 \footnotemark \ y sus T-invariantes.}
	\label{fig:Rdp-Ezpeletav2v4}
\end{figure} \footnotetext{Figura adaptada del paper publicado por \textit{Zhong} et al. \cite{paperezpeletav2}.}

En este caso particular la plaza $P_1$ de la red figura \ref{fig:Rdp-Ezpeletav2v4} forma parte de un conflicto, dado que el marcado de la misma determina la ejecución de un subcircuito de la red u otro.
\bigskip

\subparagraph{Control de la red}
\hfill

\begin{table}[H]
    \centering
    \begin{tabular}{|c|c|P{2.2cm}|P{2.2cm}|c|}
    \hline
    \textbf{Supervisor} & \textbf{Marcado} & \textbf{Transiciones input} & \textbf{Transiciones output} & \textbf{Bad Siphon Controlado}  \\  \hline
    $P_{21}$ & 4 & \{$T_{4}, T_{9}$\} & \{$T_{1}, T_{12}$\} & \{$P_5, P_{7}, P_{11}, P_{15}, P_{17}, P_{19}$\} \\ 
    \hline
    $P_{22}$ & 2 & \{$T_{4}, T_{10}$\} & \{$T_{1}, T_{12}$\} & \{$P_{5},P_{7},P_{10},P_{17}, P_{19}$\} \\ 
    \hline
    $P_{23}$ & 2 & \{$T_{3}, T_{9}$\} & \{$T_{1}, T_{12}$\} & \{$P_{4},P_{7},P_{11},P_{15}, P_{19}$\} \\ 
    \hline
    \end{tabular}
    \caption{Supervisores: RdP Ezpeleta v2 - Análisis 1 y 2.}
    \label{tab:Ezpeletav212-v4}
\end{table}

\begin{figure}[H]
	\centering
	\includegraphics[width=\textwidth]{Figures/algoritmo4/ezpeleta_v2_imag2.png}
	\caption{RdP Ezpeleta v2.0 ejecutando el análisis 1 y 2.}
	\label{fig:Rdp-Ezpeletav21y2v4}
\end{figure}

Una vez que se agregaron los supervisores mencionados en el cuadro anterior a la red, la misma aún presentaba deadlock sin indicar la posibilidad de incorporar nuevos supervisores para su control; por lo tanto, se ejecutó la parte 3 del algoritmo para resolver el problema de conflicto.
\bigskip
 
 
 
\begin{figure}[H]
	\centering
	\includegraphics[width=\textwidth]{Figures/algoritmo4/terminal-EzpeletaV2-Ejec3.png}
	\caption{Resultado al ejecutar el análisis 3.}
	\label{fig:Rdp-Ezpeletav23v4}
\end{figure}
\bigskip

\begin{table}[H]
    \centering
    \begin{tabular}{|c|c|P{2.2cm}|P{2.2cm}|c|}
    \hline
    \textbf{Supervisor} & \textbf{Marcado} & \textbf{Transiciones input} & \textbf{Transiciones output} & \textbf{Bad Siphon Controlado}  \\  \hline
    $P_{21}$ & 4 & \{$T_{4}, T_{9}, T_{13}$\} & \{$T_{1}, T_{12}$\} & \{$P_5, P_{7}, P_{11}, P_{15}, P_{17}, P_{19}$\} \\ 
    \hline
    $P_{22}$ & 2 & \{$T_{4}, T_{10}, T_{13}$\} & \{$T_{1}, T_{12}$\} & \{$P_{5},P_{7},P_{10},P_{17}, P_{19}$\} \\ 
    \hline
    $P_{23}$ & 2 & \{$T_{3}, T_{9}, T_{13}$\} & \{$T_{1}, T_{12}$\} & \{$P_{4},P_{7},P_{11},P_{15}, P_{19}$\} \\ 
    \hline
    \end{tabular}
    \caption{Supervisores: RdP Ezpeleta v2 - Análisis 3.}
    \label{tab:Ezpeletav23-v4}
\end{table}


\begin{figure}[H]
	\centering
	\includegraphics[width=\textwidth]{Figures/algoritmo4/ezpeleta_v2_imag3.png}
	\caption{RdP Ezpeleta v2.0 controlada.}
	\label{fig:Rdp-Ezpeletav2-Contv4}
\end{figure}
\bigskip

%-------------------------------------------------------------
% CONCLUSION
%-------------------------------------------------------------
\subsection{Conclusión}
Con la incorporación de las nuevas características al algoritmo inicial, se logró generalizar el análisis de las diferentes redes de Petri propuestas para el estudio (S³PR). \\
En cada caso sometido a estudio se logró resolver los problemas de deadlock, sin tener la necesidad de aplicar el criterio de división a cada red con el que se venía trabajando anteriormente. Incluso en algunos casos la solución se optimizó al lograr controlar la red con una menor cantidad de supervisores. \\
Al momento de la elección de los supervisores a agregar entre los propuestos por el algoritmo, se necesita tener un orden dado que la elección incorrecta de los supervisores puede llevar a divergir el problema y no haya solución. El criterio con el que se llegó a la solución, fue tomar los supervisores que más se repetían y aquellos cuyos arcos eran mínimos (es decir, si un supervisor tiene como arcos de entrada \{$T_1,T_5,T_9$\} , y otro supervisor tiene \{$T_1, T_5$\} y presentan las mismas transiciones de salida, se elige el segundo supervisor). \\
Esta versión final del algoritmo sigue teniendo en cuenta el conflicto y los \break T-invariantes pero ya no con la finalidad antes propuesta que era la división de las mismas para su análisis; sino que que se observó que en algunos casos, la incorporación de  supervisores ocasionaba la pérdida de el/los T-invariante/s de la red original (sin controlar) y esto llevaba a que la red aún permaneciera bloqueada. Para solucionar esto se llevó a cabo un análisis del conflicto y de las transiciones en estado idle, para determinar qué arcos debían agregarse/quitarse para preservar el/los T-invariante/s de la red original llevando a una red sin estados de deadlock.

\subsection{Extensión: algoritmo v4.1}
En esta extensión, se logró automatizar la tarea de incorporar/eliminar los arcos indicados por el análisis \textit{Red  con supervisores, tratamiento de conflictos y t\_idle} que en versiones previas se realizaba de manera manual.
\bigskip
%----------------------------------------------------------------------------------------
 % Chapter Template

\chapter{Conclusión} % Main chapter title

\label{Chapter7} % Change X to a consecutive number; for referencing this chapter elsewhere, use \ref{ChapterX}
En el capítulo introductorio del presente documento se especificaron los objetivos planteados para el proyecto desarrollado. Una vez finalizadas las tareas asociadas a los mismos, es posible extraer algunas conclusiones. \\

Por un lado se logró establecer una retroalimentación entre nuestro algoritmo y el software Petrinator, como ya se mencionó con anterioridad, en lo que sería la extracción manual de los archivos y su posterior procesamiento; así como también la actualización de la red con sus respectivos supervisores. \\

Por otra parte, se detectó la necesidad de extraer los T-invariantes de la red original, dado que estos debían preservarse a lo largo del análisis ya que con la incorporación de un nuevo supervisor la estructura de la red se ve afectada pudiendo perder los mismos, alterando el análisis y el objetivo es conservar el comportamiento de la red, es decir, que mantenga sus T-invariantes (que los procesos se sigan ejecutando de la misma manera) pero de una forma controlada y manteniendo la concurrencia de los procesos. \\
Esta modificación estructural que se presenta en la red es producto del nuevo supervisor, ya que al colocarlo no se tiene en cuenta si la red presenta conflictos o si las transiciones idle que le extraen el token son realmente necesarias. Para el primero se verificó que si la red presentaba conflictos antes de finalizar ese camino el token debía regresar el supervisor. Para el segundo se verificó que la transición idle le quitara token a los supervisores que realmente afectan al proceso que esta transición iba a desencadenar, de no ser así ese arco no debiera existir. \\

A partir de lo mencionado y demostrado a lo largo de la evolución del algoritmo, este se fue probando en redes con estructuras diferentes (conflictos presentes en la red, la distribución/relación entre los bad siphons y los T-invariantes) que permitió demostrar fiabilidad del mismo como también agregarle nuevas características para llegar al objetivo planteado. \\
De esta forma el algoritmo puede ser tomado como base para el estudio y posterior desarrollo en el área de control de redes de Petri.

\newpage
\section{Trabajo a futuro} \label{trabajofuturo}
A continuación se listan los posibles avances que se podría realizar para continuar este proyecto:

\begin{itemize}
    \item Integrar este algoritmo como una nueva funcionalidad del software Petrinator permitiendo la ejecución automática del mismo sin necesidad de la extracción manual de archivos.
    
    \item Posibilidad de integrar este proyecto junto con los que también se están desarrollando en el LAC con la idea de lograr tanto el control de la red como la ejecución de la misma (tanto en hardware como en software) a partir de ciertas políticas definidas. 
    
    \item Posibilidad de realizar un análisis para la determinación de un criterio de elección óptimo del orden de supervisores a colocar para el control de las diferentes redes permitiendo el mayor paralelismo a la hora de ejecutar las mismas. 
    
%    \item Automatizar la tarea de incorporar/eliminar los arcos indicados por el análisis \textit{Red  con supervisores, tratamiento de conflictos y t\_idle} que actualmente se realiza manualmente.
    
    \item Realizar un generador aleatorio de redes de Petri que sirvan de entrada para el algoritmo.
    
    \item Se propone seguir investigando y testeando con nuevos tipos de redes bloqueadas que presenten diferentes características a las ya analizadas.
    
\end{itemize}


 \include{Chapters/Chapter5}
 \include{Chapters/Chapter6} 
 \include{Chapters/Chapter7} 
% \include{Chapters/Chapter7}

%----------------------------------------------------------------------------------------
%	THESIS CONTENT - APPENDICES
%----------------------------------------------------------------------------------------

\appendix % Cue to tell LaTeX that the following "chapters" are Appendices

% Include the appendices of the thesis as separate files from the Appendices folder
% Uncomment the lines as you write the Appendices

% Appendix Template

\chapter{Tutorial para desplegar el entorno y las aplicaciones desarrolladas} % Main appendix title

\label{AppendixA} % Change X to a consecutive letter; for referencing this appendix elsewhere, use \ref{AppendixX}
En este apéndice, se detalla el procedimiento en el cual se configura el entorno sobre el que se llevan a cabo todos los desarrollos del trabajo. Luego se explica el procedimiento que se debe seguir para instalar las aplicaciones e iniciar la interfaz gráfica.


\section{Instalación del agente en el dispositivo}

A continuación se listan y explican los pasos que se deben seguir para poder instalar e iniciar el agente \textit{NETCONF} en el dispositivo.

\begin{itemize}   
    \item \textbf{Compilar el agente Yuma123:} En primer lugar, se deberá compilar el agente para la arquitectura deseada. Para ello, se usarán los \textit{scripts} realizados con \textit{Dockerfile}. Un ejemplo de compilación para la arquitectura \textit{NIOS II} (arquitectura del procesador del \textit{muxponder} de 40GB) se muestra a continuación.
	
	\begin{lstlisting}[language=SHELXL]
		$ cd .../NETCONF-SDN/compile_yuma123/
		$ make all TARGET=nios2
    \end{lstlisting}

    \item \textbf{Instalar el agente en el dispositivo:} Una vez compilado exitosamente, se deberá instalar el agente. Para ello se hace uso de un \textit{script} en \textit{bash}, al cual se le especifica un usuario \textit{SSH}, dirección IP y arquitectura deseada. Un ejemplo de uso se puede ver a continuación.
    
	\begin{lstlisting}[language=SHELXL]
		$ cd .../NETCONF-SDN/compile_yuma123/utils_scripts/
		$ bash ./remote_install_yuma.sh @user @host @arch
    \end{lstlisting}
	
    \item \textbf{Instalar el módulo \textit{YANG} y librería desarrollada:} Luego, se debe instalar tanto el módulo \textit{YANG} como la librería en C desarrollada. Existe otro \textit{script} en \textit{bash} que facilitará esta tarea. Para su uso, se debe especificar un usuario \textit{SSH}, dirección IP, nombre del módulo a instalar y la arquitectura deseada. Dicho script se encarga no solo de pasar a la carpeta requerida tanto el módulo como la librería, sino que también las compila previamente. Un ejemplo de su uso es el que se observa:
	
	\begin{lstlisting}[language=SHELXL]
		$ cd .../NETCONF-SDN/examples_modules/utils_scripts/
		$ bash ./remote-install-module.sh @user @host @module @arch
    \end{lstlisting}

    \item \textbf{Inicio del agente en el dispositivo:} Una vez instalado el agente y la librería, se puede iniciar el servidor \textit{netconfd} en el dispositivo. Para ello, indicamos el módulo a iniciar junto con el servidor, el nivel de debug deseado y el \textit{container target} de las operaciones por defecto.
	\begin{lstlisting}[language=SHELXL]
		$ cd ~/usrapp/sbin/
		$ ./netconfd --module=cli-mxp --log-level="debug2" --target=running --superuser=root --with-startup=true
    \end{lstlisting}
  
\end{itemize}

\section{Inicio del controlador \textit{ONOS}}

En esta sección se muestra como iniciar el controlador \textit{ONOS} que se comunicará con los dispositivos.

\begin{itemize}   
    \item \textbf{Compilación del controlador:} En primer lugar, se deberá compilar el controlador. Para ello, se ejecutará el comando que se muestra a continuación. El controlador cargará de forma automática el \textit{driver} desarrollado. 
	
	\begin{lstlisting}[language=SHELXL]
		$ cd $ONOS_ROOT && tools/build/onos-buck run onos-local -- clean debug
	\end{lstlisting}
	
	\item \textbf{Instalación de la interfaz REST en el controlador:} Luego, se deberá compilar e instalar en el controlador la aplicación que provee una interfaz REST para la interfaz gráfica desarrollada. Los pasos a seguir para este objetivo se muestran a continuación. 
	
	\begin{lstlisting}[language=SHELXL]
		$ cd .../NETCONF-SDN/onos/app-altura/altura/ && mvn clean install && onos-app localhost reinstall! target/altura-1.0-SNAPSHOT.oar
	\end{lstlisting}

	\item Algunos comandos útiles:
	\begin{lstlisting}[language=SHELXL]
		http://localhost:8181/onos/ui/login.html#/topo    //interfaz gráfica de ONOS
		$ cd $ONOS_ROOT && tools/test/bin/onos karaf@localhost // To attach to the ONOS CLI console
	\end{lstlisting}
	
\end{itemize}


\section{Interfaz gráfica}

Por último, se muestra cómo iniciar la interfaz gráfica desarrollada.

\begin{itemize}   
    \item \textbf{Entorno virtual de Python:} Para no interferir con el binario de Python instalado en el host local, primeramente se deberá instalar un entorno virtual de Python junto con las librerías requeridas por la interfaz gráfica. Para ello se ejecuta: 
	
	\begin{lstlisting}[language=SHELXL]
		$ cd .../NETCONF-SDN/python-app/
		$ virtualenv2 altura-gui
		$ ./altura-gui/bin/pip2.7 install -r requirements.txt  #(or in Windows - sometimes python -m pip install -r requirements.txt )
	\end{lstlisting}
	
	\item \textbf{Inicio de la interfaz:} Una vez preparado el entorno virtual, se ejecuta la aplicación WEB de la siguiente forma: 
	
	\begin{lstlisting}[language=SHELXL]
		$ ./altura-gui/bin/python2.7 altura.py
	\end{lstlisting}

	\item Algunos comandos útiles:
	\begin{lstlisting}[language=SHELXL]
		http://127.0.0.1:5000/    //Para acceder a la interfaz gráfica
	\end{lstlisting}
	
\end{itemize}


\section{Código fuente de la aplicación}
El código fuente de las aplicaciones y el controlador, requeridos para las pruebas mencionadas anteriormente, puede obtenerse clonando los repositorios
\begin{center}
    $ \emph {git clone https://github.com/ragnar-l/NETCONF-SDN}$
\end{center}

\begin{center}
    $ \emph {git clone https://github.com/ragnar-l/onos-fork}$
\end{center}

% % Appendix Template

\chapter{Ejecución del algoritmo completa y detallada} % Main appendix title

\label{AppendixB} % Change X to a consecutive letter; for referencing this appendix elsewhere, use \ref{AppendixX}
En este apéndice, se detalla el procedimiento de ejecución de la última versión del algoritmo (sección \ref{sec:algoritmo4}) para un caso específico. 


\section{Ejecución del caso POPN}

Siguiendo los pasos mencionados en el apéndice \ref{AppendixA} se carga la red POPN previamente creada. 

\begin{figure}[H]
	\centering
	\includegraphics[scale=0.45]{Figures/apendiceB/POPN_DEADLOCK.png}
	\caption[RdP POPN]{RdP POPN \cite{libropopn}.}
	\label{fig:popndeadlocktrue}
 \end{figure}

En primera instancia se verifica la presencia de deadlock como se muestra en la figura \ref{fig:clasificacion-red}; al verificar que la red presenta deadlock \textbf{true} se prosigue con la extracción de los otros archivos necesarios:
\begin{itemize}
    \item Análisis de invariantes.
    \item Matrices.
    \item Grafo de alcanzabilidad.
    \item Sifones y trampas.
\end{itemize}

\noindent Se realiza la ejecución del algoritmo con \textit{Python v3}:
\begin{lstlisting}[language=SHELXL]
    		$ python3  algoritmo.py
\end{lstlisting}
\bigskip

\noindent Se solicita el ingreso de los archivos exportados del Petrinator (.html). Siendo:

\begin{itemize}
    \item go.html : Grafo de alcanzabilidad.
    \item mo.html : Matrices
    \item so.html : Sifones y trampas.
    \item io.html : Análisis de invariantes.
\end{itemize}
\bigskip

\noindent Al ser la primera vez que se ingresan los archivos se lleva a cabo la elección de la opción 1: \textit{Primer análisis de la red}. El cual solicita ingresar el nombre de la red con la extensión \textit{.pflow} a la que se le colocará el supervisor correspondiente.
Una vez procesada la información y llevado a cabo el análisis, el algoritmo arroja los siguientes resultados:
\bigskip

\begin{figure} [H]
    \centering
    \includegraphics[scale=0.8]{Figures/apendiceB/Py-POPN1.png}
    \caption{Primer iteración: ejecución del primer análisis de la red.}
    \label{fig:b-popn1}
\end{figure}

\begin{figure} [H]
    \centering
    \includegraphics[scale=0.7]{Figures/apendiceB/Py-POPN2.png}
    \caption{Lista de supervisores.}
    \label{fig:b-popn2}
\end{figure}

\begin{figure} [H]
    \centering
    \includegraphics[scale=0.65]{Figures/apendiceB/Py-POPN3.png}
    \caption{Elección del supervisor.}
    \label{fig:b-popn3}
\end{figure}

El mismo arroja la cantidad de estados que presentan deadlock, el número de sifones vacíos asociados y los correspondientes supervisores para controlarlos cada uno con su respectivo id, marcado y transiciones input/output. \\
Una vez seleccionado el supervisor a colocar, el propio algoritmo se encarga de agregar la plaza con su marcado y arcos correspondientes al mismo sobre el archivo \textit{.pflow} especificado con anterioridad. \\
Luego se vuelve al Petrinator y se selecciona la opción \textit{Reload} como se muestra en la figura \ref{fig:b-reload}.

\begin{figure} [H]
    \centering
    \includegraphics[width=\textwidth]{Figures/apendiceB/POPN_reload.png}
    \caption{Reload de la red.}
    \label{fig:b-reload}
\end{figure}

\noindent Como se muestra en la siguiente figura \ref{fig:b-supervisor}, la plaza resaltada $P_{27}$ es el supervisor agregado.

\begin{figure} [H]
    \centering
    \includegraphics[scale=0.45]{Figures/apendiceB/Primer-supervisor.png}
    \caption{Supervisor agregado.}
    \label{fig:b-supervisor}
\end{figure}

Se debe verificar si el supervisor colocado soluciona el problema de deadlock realizando nuevamente el análisis de la red. En caso de no ser así, siguen existiendo el deadlock, se deberán extraer nuevamente los archivos correspondientes a la nueva red. Para seguir ejecutando el análisis se debe ingresar la opción 1 como se muestra al final de la figura \ref{fig:b-popn3}. \\
\par
Nuevamente, se deben ingresar los archivos para realizar un nuevo análisis. En este caso la segunda opción \textit{Análisis de red con supervisores}. 
\bigskip

\begin{figure} [H]
    \centering
    \includegraphics[width=\textwidth]{Figures/apendiceB/Py-POPN4.png}
    \caption{Segunda Iteración: análisis de la red con supervisores.}
    \label{fig:b-popn4}
\end{figure}
\bigskip

Como se puede observar en la figura \ref{fig:b-popn4} la lista de supervisores sugeridos presentan marcado 0, por este motivo no se coloca ninguno de estos y se vuelve a ejecutar el algoritmo pero haciendo uso de la opción \textit{Red con supervisores, tratamiento de conflicto y t\_idle}. Se deben eliminar/colocar los arcos indicados por el algoritmo como se muestra en la figura siguiente.

\begin{figure} [H]
    \centering
    \includegraphics[width=\textwidth]{Figures/apendiceB/Py-POPN5.png}
    \caption{Tercera Iteración: red con supervisores, tratamiento de conflicto y t\_idle.}
    \label{fig:b-popn5}
\end{figure}

\noindent El resultado de lo indicado con anterioridad se observa en la siguiente imagen.

\begin{figure} [H]
    \centering
    \includegraphics[width=\textwidth]{Figures/apendiceB/eliminacion-arco.png}
    \caption{Red resultante de agregar/eliminar arcos}
    \label{fig:b-supervisor1}
\end{figure}

Se verifica nuevamente si la red presenta deadlock y se extraen los archivos correspondientes. Para continuar con la ejecución iterativa hasta lograr el control total de la red, como se ejemplifica en las siguientes imágenes:  

\begin{figure} [H]
    \centering
    \includegraphics[width=\textwidth]{Figures/apendiceB/Py-POPN6.png}
    \caption{Cuarta iteración: análisis de la red con supervisores.}
    \label{fig:b-popn6}
\end{figure}

\begin{figure} [H]
    \centering
    \includegraphics[width=\textwidth]{Figures/apendiceB/Py-POPN7.png}
    \caption{Elección del supervisor.}
    \label{fig:b-popn7}
\end{figure}

En este caso, se selecciona el supervisor de id = 0. Como se puede observar en la siguiente figura.

\begin{figure} [H]
    \centering
    \includegraphics[width=\textwidth]{Figures/apendiceB/supervisor2.png}
    \caption{Segundo supervisor agregado.}
    \label{fig:b-supervisor2}
\end{figure}

De manera iterativa\footnote{Para cada nueva iteración fue necesario verificar si la red presentaba deadlock y exportar los archivos correspondientes, como en las iteraciones anteriores.} se colocan los demás supervisores, hasta que el análisis resulta con deadlock \textbf{false} o con supervisores de marcado 0. En el segundo caso, se debe hacer la ejecución de red con supervisores,tratamiento de conflicto y t\_idle.

\begin{figure} [H]
    \centering
    \includegraphics[width=\textwidth]{Figures/apendiceB/supervisor3.png}
    \caption{Tercer supervisor agregado.}
    \label{fig:b-supervisor3}
\end{figure}
\bigskip

\begin{figure} [H]
    \centering
    \includegraphics[width=\textwidth]{Figures/apendiceB/supervisor4.png}
    \caption{Cuarto supervisor agregado.}
    \label{fig:b-supervisor4}
\end{figure}

\begin{figure} [H]
    \centering
    \includegraphics[width=\textwidth]{Figures/apendiceB/Py-POPN8.png}
    \caption{Última Iteración.}
    \label{fig:b-popn8}
\end{figure}

Por último, se ejecuta el algoritmo con la opción 3 como se observa en la figura \ref{fig:b-popn9}, indicando cuales arcos deben agregarse y cuales deben eliminarse. 

\begin{figure} [H]
    \centering
    \includegraphics[width=\textwidth]{Figures/apendiceB/Py-POPN9.png}
    \caption{Arcos a eliminar/agregar.}
    \label{fig:b-popn9}
\end{figure}

Se agregan/eliminan los arcos y se comprueba la clasificación de la red observando que el deadlock es false. 

\begin{figure}[H]
	\centering
	\includegraphics[width=\textwidth]{Figures/algoritmo4/popn_imag3.png}
	\caption{RdP POPN controlada.}
	\label{fig:Rdp-POPN-Contv4}
\end{figure}
\bigskip

Una vez que se verifica que la red no presenta deadlock, para finalizar la ejecución del programa, se debe ingresar \textit{0} en el menú de opciones que se muestra al final de la figura \ref{fig:b-popn3}. 
%\include{Appendices/AppendixC}

%----------------------------------------------------------------------------------------
%	BIBLIOGRAPHY
%----------------------------------------------------------------------------------------

\printbibliography[heading=bibintoc]
% 
%----------------------------------------------------------------------------------------

\end{document} 